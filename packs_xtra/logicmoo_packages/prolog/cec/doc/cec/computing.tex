\section{Computing with Completed Specifications}

One may regard the completed specification as a first step towards an operational
implementation. 
Terms can be efficiently normalized, identities efficiently proved or disproved.
Moreover the solutions of equations can be found through narrowing.

\subsection{Normalization of Terms}
\label{NormCommand}

Computation in specifications is realized by term reduction with the
rules of the completed specification. The result of such computations are unique
normal forms. \bigskip

\begin{command}[\com{norm}{\comArg{Expression}}]
normalizes the input expression \comArg{Expression}. Three kinds of normalization are
provided:
\begin{itemize}
\item
if \comArg{Expression} is a signature term, the current rules are used to compute the
normalform of \comArg{Expression}.
\item
if \comArg{Expression} has the form \condRule{\comArg{conjunction}}{\comArg{term}} 
where \comArg{conjunction} is
a conjunction of equations, CEC will try to orient these equations into
rules before normalizing \comArg{Term} with the current rules and these oriented
condition equations.
\item
if \comArg{Expression} is a {\em let-expression}, the definitions (which
are equations between constructor terms and let-expressions) 
are evaluated first,
binding variables to the evaluated terms, before normalizing the body
of the let-expression using the current set of rules. This feature
should be used in the order-sorted case to avoid typing problems.
\end{itemize}
Let-expressions are given according to the following syntax:

\begin{syntax}
\nt{let-expression} \IS \kw{let} \nt{definitions} \kw{in} \nt{let-expression}\END
\\
\nt{definitions}    \IS \nt{definition} \{ \kw{and} \nt{definition} \} \END
\nt{definition}     \IS \nt{pattern} \kw{=} \nt{let-expression}
		    \OR \nt{pattern} \kw{=} \nt{signatureTerm} \END
\\
\nt{pattern}        \IS $<$ \rm well typed term with variables build up from
                    \GETON \hspace{2ex} constructors of the user-specified signature$>$
\end{syntax}

\noindent 
It is possible to trace the application of all rules or some selected rules:\medskip

\kw{trace:=on}

\noindent enables the trace mechanism.\medskip

\kw{rulesToTrace:=[}\comArg{RuleIndex1}\ad\ldots\ad\comArg{RuleIndexN}\kw{]}

\noindent restricts the trace mechanism to applications of the
rules \comArg{RuleIndex1},\ldots,\comArg{RuleIndexN}.
\end{command}

\begin{command}[\com{applyRule}{\comArg{Term}\ad\comArg{RuleIndex}\ad\comArg{ReducedTerm}}]
attempts to apply the rule with the given index once to the given term.
Different redexes are tried upon backtracking. If successful, the reduced
term is computed.
\end{command}

%\noindent
%Given the following confluent and Noetherian specification: \bigskip
%
%\begin{screen}
%| ?- {\bf show.}
%
%Current equations
%
%
%
%Current rules:
%
%  1  append([],l) -> l
%  2  append([e|l1], l2) -> [e|append(l1,l2)]
%  3  rev([]) -> []
%  4  rev([e|l]) -> append(rev(l),[e])
%
%Current nonoperational equations
%
%
%All axioms reduced.
%All critical pairs computed.
%All superpositions on nonoperational equations computed.
%No more equations, the system is complete.
%yes
%| ?-
%\end{screen}
%
%\noindent
%Then we can compute the reverse of a given list by computing it's normal form.
%
%\begin{screen}
%| ?- {\bf norm(rev([a, b, c, d]), NormalForm).}
%
%NormalForm = [ d, c, b, a]
%
%| ?-
%\end{screen}
%
%\noindent
%With applyRule we can apply one specific rule to reduce a term:
%
%\begin{screen}
%| ?- {\bf applyRule(rev([a, b, c, d]), 4, ReducedForm).}
%
%ReducedForm = append(rev([b, c, d]), [a])
%
%| ?- {\bf applyRule(rev([a, b, c, d]), 3, ReducedForm).}
%
%no
%
%| ?-
%\end{screen}

\noindent
To improve the computation of the normal form of a term, one can {\it compile}
the current set of rewrite rules into compiled Prolog and
use this compiled set of rewrite rules to normalize given terms.\bigskip

\begin{command}[\comName{compile}]
compiles the current set of rewrite rules into \comRef{compiled} Prolog.
The compiled rules are used when calling \comRef{eval}. 
Later changes to the set of rewrite rules
have no effect on \comRef{eval} unless a new call to \comRef{compile} is
performed. The predicate \comRef{norm} always uses the current set of
rewrite rules.
\end{command}

\begin{command}[\com{eval}{\comArg{Expression}}]
computes the normalform of a expression using the most recently compiled set of rewrite
rules. \comRef{eval} fails, if \comRef{compile} has not been called yet, cf. \comRef{compile}.
The trace mechanism applies also to \comRef{eval}.
\end{command}

\subsection{Proving Theorems in the Equational Theory}
\label{ProveCommand}

If general confluence can be achieved equational theorems become
decidable, e.g. it is decidable if two terms are equivalent with respect to the
equations in the specification: simply reduce the two terms to their unique normal
form and check if they are identical. This can be done using the
\comRef{prove}-command:\bigskip

\begin{command}[\com{prove}{\comArg{ConditionalEquation}}]
proves or disproves the conditional equation \comArg{ConditionalEquation} by 
rewriting the conclusion to normalforms, using the equations in the 
condition as additional rewrite rules. The method is incomplete for
nonempty conditions and/or noncanonical systems.
\end{command}

%In the following example a theorem of the equational theory is proved.
%
%\begin{screen}
%| ?- {\bf prove((append(rev([x, y]), [c, d]) = [y, x, c, d])).}
%
%Normal forms are : [y,x,c,d] and [y,x,c,d]
%yes
%| ?-
%\end{screen}


\subsection{Solving Equations in the Equational Theory}
\label{NarrowCommand}

Narrowing can be used to solve equations in an equational theory if a
canonical set of rewrite rules equivalent to the set of equations  exists.\bigskip

\begin{command}[\com{solve}{\comArg{Goal}\ad\comArg{Solution}}]
tries to solve \comArg{Goal} and to return an answer-substitution 
if successful.
The set of all answer-substitutions can be obtained by backtracking
(enter '\kw{;}'\nt{return} at the user level). Enter \nt{return} if
no more solutions are wanted.

\noindent
Goals are given using the following syntax:

\begin{syntax}
\nt{Goal} \IS \nt{condition} \nt{equation} 
          \OR \nt{equation} \{ \kw{and} \nt{equation} \} \END
\\
\nt{condition} \IS \nt{equation} \{ \kw{and} \nt{equation} \} \kw{=>} \END
\nt{equation}  \IS \nt{signatureTerm} \kw{=} \nt{signatureTerm} \END
\end{syntax}
\end{command}
