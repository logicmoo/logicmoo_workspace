\section{Listing of all CEC commands}
\label{commands}

Remember that commands must be followed by a full stop `\user{.}'.\bigskip

\begin{command}[\comName{??}\nt{space}]
lists the available CEC commands and refers to the \comRef{?}-command for further
informations.
\end{command}

\begin{command}[\com{??}{\comArg{Keyword}}]
Lists only topics which contain \comArg{Keyword} as a substring.
\comArg{Keyword} must be a Prolog atom, e.g.\ if it contains any special
characters it must be enclosed in single quotes.
\end{command}

\begin{command}[\com{?}{\comArg{Function}}]
prints a short description for the CEC command \comArg{Function}. 
\comRef{?} is specified as prefix operator.
\end{command}

\begin{command}[\com{applyRule}{\comArg{Term}\ad\comArg{RuleIndex}\ad\comArg{ReducedTerm}}]
attempts to apply the rule with the given index once to the given term.
Different redexes are tried upon backtracking. If successful, the reduced
term is computed.
\end{command}

\begin{command}[\comName{c}]
calls the Knuth-Bendix completion procedure. This executes a fixed strategy
of applications of the ``completion inference'' predicates \comRef{orderEq}, 
\comRef{cp}, \comRef{superpose}, \comRef{redRule}, \comRef{redEq} 
and \comRef{redNOpEq}.
A manual guidance of the process is possible by explicitly calling these
predicates. The \comRef{repeat} predicate can be used to execute a predicate 
repeatedly until no more instances of it can be applied. An arbitrary
interleaving of manual and automatic completion is supported. Also,
completion --- manual or automatic --- can always safely be restarted after 
any abortion caused by answering ``\user{A.}'' to some query of the system.
The resulting system is not necessarily a reduced system. In the conditional
case it is anyway not clear what a reduced equation is. On the other hand,
a user may always call any of the \comRef{red}-predicates after completion
to force reduction of the axioms. This may, however, lead to an incomplete
system.
\end{command}

\begin{command}[cResume]
restarts the completion procedure after the completion process was 
aborted by answering ``\user{A.}'' to some query of the system.
\end{command}

\begin{command}[\com{cd}{\comArg{Path}}]
Changes, as the cd command in UNIX, the directory for all following
file-related CEC-commands. It is declared as prefix operator.\\
The path is given in form of a Prolog-atom, hence don't forget the
quotes, if the path contains `\kw{/}', `\kw{.}', and `\kw{..}' and 
other special characters.

\comRef{cd} does not implement file name generation using patterns.
Hence, `\kw{*}', `\kw{?}' and '\kw{[}' do not receive any special treatment.\\
Without an argument \comRef{cd} resets the current directory to the one in
which the CEC-system was invoked initially.
\end{command}


\begin{command}[\com{combineSpecs}{\comArg{StateName1}\ad\comArg{StateName2}\ad\comArg{CombinedSpec}}]
The specifications  stored (\refArrow \comRef{store}) 
in specification variables \comArg{StateName1} and \comArg{StateName2}
will  be combined, if possible, by forming
the union of the signature, axioms and pragmas.
If \comArg{CombinedSpec} $\neq$ \cec{user}, the result will be stored in
\comArg{CombinedSpec}, and the current specification will not be affected.
Otherwise, the combined specification becomes the new current
specification.
\end{command}

\begin{command}[\comName{compile}]
compiles the current set of rewrite rules into \comRef{compiled} Prolog.
The compiled rules are used when calling \comRef{eval}. 
Later changes to the set of rewrite rules
have no effect on \comRef{eval} unless a new call to \comRef{compile} is
performed. The predicate \comRef{norm} always uses the current set of
rewrite rules.
\end{command}
 
\begin{command}[\com{compileRules}{\comArg{File}}]
compiles current set of rewrite rules to Prolog and writes the Prolog clauses into
the file \comArg{File}\suffix{.rules}. \comArg{File}\suffix{.rules} may later be consulted or compiled 
(cf. \comRef{loadRules}) to produce a new definition of \comRef{eval}, cf. \comRef{compile}.
\end{command}
 
\begin{command}[\com{cp}{\comArg{RuleIndex1}\ad\comArg{RuleIndex2}}]
computes all critical pairs of rule \comArg{RuleIndex1} on rule \comArg{RuleIndex2}. 
The predicate fails, if no nontrivial critical pair can be found.
\end{command}

\begin{command}[\comName{constructor}]
asks the user to enter an operator and declares this operator to be
a constructor, if this is consistent with the current specification.
\end{command}

% \newpage

\begin{command}[\com{delete}{\comArg{ModuleName}\ad\comArg{OrderName}}]
deletes the specification variable named
\comArg{ModuleName}\kw{.}\comArg{OrderName}. If \comRef{delete} is used only
with argument \comRef{ModuleName}, the specification stored under this name is
deleted.
\end{command}

\begin{command}[\com{enrich}{\comArg{ModuleName}\ad\comArg{OrderName}}]
reads in additional parts of a specification from the files 
\comArg{ModuleName}\suffix{.eqn} and \comArg{ModuleName}\suffix{.}\comArg{OrderName}\suffix{.eqn}
after saving the current state for later checks for consistency of the enrichment. 
These additional parts must form an enrichment (cf. chapter~\ref{enrichment}).
\comArg{ModuleName} and \comArg{OrderName} can be arbitrary Prolog atoms.
Leaving out \comArg{OrderName} or even both arguments yields the same
effect as for \comRef{in}. 
%Specify \comArg{ModuleName} = \kw{user} if input from terminal is wanted.
\end{command}
 

\begin{command}[\comName{equal}]
asks the user to enter a list of the following form: 
\begin{center}
\kw{[}\kw{[}\comArg{a}\kw{,}\comArg{b}\kw{,}\comArg{c}\kw{,} \ldots \kw{],}
\kw{[}\comArg{g}\kw{,}\comArg{h}\kw{,}\comArg{i}\kw{,} \ldots \kw{],} \ldots \kw{]}
\end{center}
and declares operators to have equivalent precedences 
--- only allowed for \kw{kns}.\\
Meaning: \comArg{a} = \comArg{b} = \comArg{c} = \ldots and \comArg{g} = \comArg{h} = \comArg{i} = \ldots.
\end{command}


\begin{command}[\com{eval}{\comArg{Expression}}]
computes the normalform of a expression using the most recently compiled set of rewrite
rules. \comRef{eval} fails, if \comRef{compile} has not been called yet, cf. \comRef{compile}.
The trace mechanism applies also to \comRef{eval}.
\end{command}

\begin{command}[\comName{forget}]
forgets the complete undo history.
\end{command}


\begin{command}[\com{freeze}{\comArg{ModuleName}\ad\comArg{OrderName}}]
writes the state of the current specification to the file
\comArg{ModuleName}.\comArg{OrderName}\suffix{.q2.0} and
updates the content of the specification variable
\comArg{ModuleName}.\comArg{OrderName}.
If freeze is used without \comArg{OrderName} the state is just written
to \comArg{ModuleName}\suffix{.q2.0}. 
If freeze is used without any argument the module name of the current 
specification is used for \comArg{ModuleName} and the current order name
is used for \comArg{OrderName}. In the last two cases the specification 
variable determined by the current module name and the current order name
will be updated (If current order name is \kw{noorder} the specification
variable associated with the current module name and the current termination
ordering will be updated too).
The specification may later be reused by thawing it from this file, cf. the 
\comRef{thaw}-command.
The state of CEC remains unchanged by this operation.
\end{command}

 
\begin{command}[\comName{greater}]
asks the user to enter a list of the following form: 
\begin{center}
\kw{[}\kw{[}\comArg{a}\kw{,}\comArg{b}\kw{,}\comArg{c}\kw{,} \ldots \kw{],}
\kw{[}\comArg{g}\kw{,}\comArg{h}\kw{,}\comArg{i}\kw{,} \ldots \kw{],} \ldots \kw{]}
\end{center}
and adds ordered pairs of operators to the precedence.\\
Meaning: \comArg{a} $>$ \comArg{b} $>$ \comArg{c} $>$ \ldots and 
\comArg{g} $>$ \comArg{h} $>$ \comArg{i} $>$ \ldots
\end{command}


\begin{command}[\comName{halt}]
ends a CEC session.
\end{command}


\begin{command}[\com{in}{\comArg{ModuleName}\ad\comArg{OrderName}}]
reads in a specification from the file \comArg{ModuleName}\suffix{.eqn}
and the associated order specification from the file
\comArg{ModuleName}\suffix{.}\comArg{OrderName}\suffix{.ord}.
As log-files are ``enriched'' order specifications, any log-file can
be used as an order file.
If not stated otherwise these files are assumed to be in the current 
directory. Before the specification is read in, CEC will be re-initialized, 
e.g. the current specification will be deleted. Specifications saved in 
variables will not be affected. \comArg{ModuleName} and \comArg{OrderName}
must be Prolog atoms. \comArg{OrderName} becomes the
current order name for the specification.\\
But \com{in}{\comArg{ModuleName}\ad\cec{noorder}}
has the effect that no order specification is consulted.
The termination ordering for the new
specification will be initialized to a default value 
(\kw{neqkns} or \kw{poly1}, depending on the presence of AC-operators).
Using \comRef{in} only with the parameter \comArg{ModuleName} yields the same
effect. \comArg{ModuleName} = \kw{user} expects input from terminal.
(For Quintus-Prolog2.x under EMACS: \comRef{in} without
parameter reads from \cec{Scratch.pl}).
\end{command}


\begin{command}[\comName{interpretation}]
displays all operator interpretations, provided \kw{poly}\nt{N} is the chosen termination
ordering and asks the user if he wants to change any. If so all rules will be turned 
back into equations.
\end{command}


\begin{command}[\com{load}{\comArg{ModuleName}\ad\comArg{OrderName}}]
loads the system which is currently the value of the variable
\comArg{ModuleName}\kw{.}\comArg{OrderName}, cf. the \comRef{store}-command. 
If \comRef{load} is used only with argument \comArg{ModuleName}, this actual parameter 
completely specifies the name of the variable.
Specification variables remain unchanged.
\comArg{StateName} = \cec{'$initial'} re-initializes the system.
\end{command}


\begin{command}[\com{loadLog}{\comArg{ModuleName}\ad\comArg{OrderName}}]
reads in the file \comArg{ModuleName}\suffix{.}\comArg{OrderName}\suffix{.@.ord}. 
%This file contains all the answer given during
%the completion process and the final termination ordering saved using
%the \comRef{saveLog}-command. 
If the completion process is started again, % all
questions whose answers are already contained in 
\comArg{ModuleName}\suffix{.}\comArg{OrderName}\suffix{.@.ord} 
will be suppressed. 
If \comRef{loadLog} is used without the argument \comArg{OrderName} the 
information will be taken from the file \comArg{ModuleName}\suffix{.@.ord}.
If \comRef{loadLog} is used without any argument
the name of the current specification together with the current order name
will be used.
%The current order name is the
%order name of the order specification for the current specification
%or the current termination ordering if no order specification was used.
\end{command}

\begin{command}[\com{loadOrder}{\comArg{ModuleName}\ad\comArg{OrderName}}]
reads in the file \comArg{ModuleName}\suffix{.}\comArg{OrderName}\suffix{.ord}. 
This file should contain informations concerning the termination ordering.
If \comRef{loadLog} is used without the argument \comArg{OrderName} the 
information will be taken from the file \comArg{ModuleName}\suffix{.ord}.
If \comRef{loadLog} is used without any argument
the name of the current specification together with the current order name
will be used.
\end{command}

\begin{command}[\com{loadRules}{\comArg{File}}]
compiles the set of rewrite rules which has been stored previously in the file
\comArg{File}\suffix{.rules}, cf. \comRef{compileRules}.
\end{command}

\begin{command}[\comName{moduleName}]
displays the name of the current specification.
\end{command}


\begin{command}[\com{nopEq}{\comArg{EquationIndex}}]
declares equation with index \comArg{EquationIndex} as nonoperational. 
The predicate fails if equation \comArg{EquationIndex} does not exist 
or if the equation is trivial.
\end{command}

\begin{command}[\com{norm}{\comArg{Expression}}]
normalizes the input expression \comArg{Expression}.
See also \comRef{eval}
\end{command}


\begin{command}[\comName{operators}]
displays all precedences and stati in \kw{kns} or \kw{neqkns} or
all polynomial interpretations in \kw{poly}\nt{N} respectively.
\end{command}

\begin{command}[\comName{order}]
indicates the current termination ordering and asks the user whether he wants to 
change it.
If a new ordering is selected and if the previous termination
ordering is incompatible with the new ordering, all rules are turned back into
equations and the completion must be repeated from the beginning.
\end{command}

\begin{command}[\com{orderEq}{\comArg{EquationIndex}}]
orients equation with index \comArg{EquationIndex}. The predicate fails if equation 
\comArg{EquationIndex} does
not exist or if the equation cannot be oriented or turned into a nonoperational
equation or if the equation is eliminated during reduction.
\end{command}

\begin{command}[\comName{orderName}]
displays the name of the order specification associated with
the current specification.
\end{command}

\begin{command}[\com{polGreater}{\comArg{Interpretation1}\ad\comArg{Interpretation2}}]
Only useful with ordering \kw{poly}\nt{N}.
If it succeeds, \comArg{Interpretation1} $>$ \comArg{Interpretation2} holds true
(if $>$ is the ordering on tuples of polynomials).
Interpretations (i.e. tupels of polynomials) of terms can be generated
via \comRef{polynomial}.
\end{command}

\begin{command}[\com{polynomial}{\comArg{Term}\ad\comArg{Interpretation}}]
yields the polynomial interpretation of \comArg{Term}. It
fails, if the ordering is not \kw{poly}\nt{N}. 
If there are operators in \comArg{Term},
for which no polynomial interpretation is known, the user is asked for
such an interpretation (and the given interpretation is stored).
\end{command}


\begin{command}[\comName{preregular}]
succeeds if the current (order-sorted) signature is preregular,
and fails otherwise.
The preregularity condition is the regularity of 
Smolka/Nutt/Goguen/Meseguer 87 (\refArrow \cite{SNGM87}).
\end{command}


\begin{command}[\com{prove}{\comArg{ConditionalEquation}}]
proves or disproves the conditional equation \comArg{ConditionalEquation} by 
rewriting the conclusion to normalforms, using the equations in the 
condition as additional rewrite rules. The method is incomplete for
nonempty conditions and/or noncanonical systems.
\end{command}

\begin{command}[\comName{pwd}]
prints out the current path.
\end{command}

\begin{command}[\com{redEq}{\comArg{EquationIndex}}]
reduces equation with index \comArg{EquationIndex}. 
The predicate fails if equation \comArg{EquationIndex} does
not exist or if the equation cannot be reduced or if the equation is eliminated
during reduction.
\end{command}

\begin{command}[\com{redNopEq}{\comArg{NopEqIndex}}]
reduces nonoperational equation with index \comArg{NopEqIndex}. 
The predicate fails if equation
\comArg{NopEqIndex} does not exist or if the equation cannot be reduced or if the 
equation is eliminated during reduction.
\end{command}
 
\begin{command}[\com{redRule}{\comArg{RuleIndex}}]
reduces rule with index \comArg{RuleIndex}. The predicate fails if rule 
\comArg{RuleIndex} does not
exist or if the rule cannot be reduced or if the rule is eliminated during 
reduction.
\end{command}

\begin{command}[\comName{regular}]
succeeds if the current (order-sorted) signature is regular,
and fails otherwise.
The regularity condition is the one of Goguen/Meseguer 87
(\refArrow \cite{GM87}).
\end{command}

\begin{command}[\com{renameSpec}{\kw{[}\comArg{OldSort1} \kw{<-} \comArg{NewSort1}, \ldots , 
\comArg{OldSortN} \kw{<-} \comArg{NewSortN},\\ \hspace*{6.2em}
\comArg{OldOperator1} \kw{<-} \comArg{NewOperator1}, \ldots , 
\comArg{OldOperatorM} \kw{<-} \comArg{NewOperatorM}\/\kw{]}}]
renames the current specification according to the given lists of sort associations
and operator associations. Only injective renamings of operators are allowed.
Sorts may be renamed arbitrarily. Sorts and operators which are not mentioned remain 
unchanged.
\end{command}

\begin{command}[\com{repeat}{\comArg{Predicate}}]
causes repeated backtracking of \comArg{Predicate} until \comArg{Predicate} fails.
\end{command}
 
\begin{command}[\com{resetOrient}{\comArg{Index}}]
turns the rule Index back into an equation. If \comRef{resetOrient}
is used without an argument, all rules are turned back into equations.
\end{command}


\begin{command}[\com{restoreCEC}{\comArg{FileName}}]
restores the CEC state (prolog state) in \comArg{FileName}. 
\end{command}

\begin{command}[\com{saveCEC}{\comArg{FileName}}]
saves the whole CEC state (prolog state) in \comArg{FileName}. The CEC state
can be used by simply invoking \comArg{FileName} instead of CEC or 
using the \comRef{restoreCEC}-command.
If \comRef{saveCEC} is used without argument, the current CEC-name,
i.e. the actual parameter for the last use of the \comRef{saveCEC}-command,
is used for \comArg{FileName}. If no current CEC-name is known, the name
\kw{cec} will be taken (so be careful).
\end{command}

\begin{command}[\comName{setInterpretation}]
asks the user to enter a list of the following form:
\begin{center}
\kw{[}\comArg{Operator(Arguments)} : \comArg{Interpretation}\kw{,} \ldots\kw{]}
\end{center}
Provided \kw{poly}\nt{N} is the current termination ordering a new interpretation
\comArg{Interpretation} for an operator \comArg{Operator} is added to the current state.
The interpretation
may be a polynomial over the variables in \comArg{Arguments} (if
N = 1) or a list with N polynomials.
The new interpretation must be compatible
with all C- or AC-declarations in the current state.
\end{command}

 
\begin{command}[\comName{show}]
shows the sets of equations, rules and nonoperational equations of the 
current specification in order-sorted notation.
(Usually there exists more than one many-sorted representation of
an order-sorted axiom.)

Rules \condRule{C}{\rewRule{l}{r}} which are marked by an asterix 
\kw{*} have an associated auxiliary rule of form
\condRule{C}{\rewRule{l+X}{r+X}}
where $+$ is the AC-operator on top of $l$ and $X$ is a new
variable of appropriate sort.
Auxiliary rules are automatically generated when needed during
completion modulo AC.
\end{command}
 
\begin{command}[\com{show}{\comArg{SpecificationVariable}}]
shows the sets of equations, rules and nonoperational equations 
of the specification which is stored in the variable 
\comArg{SpecificationVariable}.
\end{command}

 
\begin{command}[\comName{showCStatus}]
displays the current completion status.
\end{command}

\begin{command}[\comName{showms}]
shows the set of equations, rules and nonoperational equations of the 
current specification in many-sorted notation.
\end{command}
 
\begin{command}[\comName{sig}]
displays the signature of the current specification.
\end{command}

\begin{command}[\com{solve}{\comArg{Goal}\ad\comArg{Solution}}]
tries to solve \comArg{Goal} and to return an answer-substitution if
successful. The set of all answer-substitutions can be obtained by
backtracking (enter `\user{;}'\nt{return} at the user level). Enter
\nt{return}, if no more solutions are wanted.
\end{command}

\begin{command}[\com{sp}{\comArg{Index}\ad\comArg{NopEqIndex}}]
same as \com{superpose}{\comArg{Index}\ad\comArg{NopEqIndex}\ad\kw{left}\ad\kw{condition(1)}}.
\end{command}

\begin{command}[\comName{specifications}]
lists the module and order names of all specifications that are currently
saved in variables.
\end{command}

\begin{command}[\comName{status}]
asks the user to enter a list of the following form:
\begin{center}
\kw{[}\comArg{Operator} : \comArg{Status}\kw{,} \ldots \kw{]}
\end{center}
and declares that the operators should have the desired stati provided \kw{kns}
or \kw{neqkns}
is the type of the current termination ordering. \comArg{Status} can be
\kw{lr} for {\em left-to-right},
\kw{rl} for {\em right-to-left} or
\kw{ms} for {\em multiset}.
\end{command}

\begin{command}[\com{store}{\comArg{ModuleName}\ad\comArg{OrderName}}]
saves the current specification in a specification variable named
\comArg{ModuleName}\kw{.}\comArg{OrderName}. If \comRef{store} is used only with
argument \comArg{ModuleName} the specification is saved in a variable with this name,
if \comRef{store} is used without any argument, the name is created as 
\nt{moduleName}\kw{.}\nt{orderName}, with names as they are currently associated 
with the specification. For later restoring 
use the command \comArg{load}. The system remains unchanged except for this variable 
containing afterwards the current specification.
\end{command}

\begin{command}[\com{storeLog}{\comArg{ModuleName}\ad\comArg{OrderName}}]
creates the log-file.
The name of the log-file is 
\comArg{ModuleName}\suffix{.}\comArg{OrderName}\suffix{.@.ord}.
It has the format of an order specification file which can be used
with the \comRef{in}-command or the \comRef{loadLog}-command. 
If \comRef{storeLog} is used only with argument \comArg{ModuleName}
the log-file is named \comArg{ModuleName}\suffix{.@.ord},
if \comRef{storeLog} is used without any argument, the file is
created as \nt{moduleName}\suffix{.}\nt{orderName}\suffix{.@.ord}.
\end{command}

\begin{command}[\com{storeOrder}{\comArg{ModuleName}\ad\comArg{OrderName}}]
creates a file named
\comArg{ModuleName}\suffix{.}\comArg{OrderName}\suffix{.ord}.
It has the format of an order specification file which can be used
with the \comRef{in}-command or the \comRef{loadOrder}-command. 
If \comRef{storeOrder} is used only with argument \comArg{ModuleName},
the name \comArg{ModuleName}\suffix{.ord} is used.
If the command \comRef{storeOrder} is used without any argument, the file created
is named \nt{moduleName}\suffix{.}\nt{orderName}\suffix{.ord}.
\end{command}

\begin{command}[\com{superpose}{\comArg{RuleIndex}\ad\comArg{NopEqIndex}\ad\comArg{Literal}\ad\comArg{LiteralSide}}]
superposes the left-side of the rule \comArg{RuleIndex} on the 
\comArg{LiteralSide}-side of the
literal \comArg{Literal} of the nonoperational equation with index 
\comArg{NopEqIndex}.
It fails if no nontrivial superpositions can be found, if
any of the two axioms can be reduced, or if
superpositions of the specified type need not be computed to achieve 
fairness.\\
To denote the \comArg{LiteralSide} \kw{left} and \kw{right} are used. 
\comArg{NopEqIndex} must be the index of a nonoperational equation.
Considering superposition with
\condEq{L_1\condAnd\ldots\condAnd L_n}{L}, 
we use \kw{conclusion} to denote $L$, and \com{condition}{\comArg{$i$}}
to denote $L_i$ in \comArg{Literal}.
\end{command}

\begin{command}[\com{superpose}{\kw{reflexivity}\ad\comArg{NopEqIndex}\ad
\comArg{Literal}\ad\_}]
superposes \rewRule{x = x}{true} on the literal \comArg{Literal} of the 
nonoperational equation with index \comArg{NopEqIndex}.
It fails if no nontrivial superpositions can be found, if
any of the two axioms can be reduced, or if
superpositions of the specified type need not be computed to achieve 
fairness.
\end{command}

\begin{command}[\com{'superpose!'}{\comArg{RuleIndex}\ad\comArg{NopEqIndex}\ad\comArg{Literal}\ad\comArg{LiteralSide}}]
same effect as \comRef{superpose}, except that the specified superposition
will be performed in any case, even if not necessary for fairness of
completion.
\end{command}

\begin{command}[\com{thaw}{\comArg{ModuleName}\ad\comArg{OrderName}\ad\comArg{SpecificationVariable}}]
This command is the inverse operation of \comRef{freeze} and restores the specification
previously frozen in \comArg{ModuleName}\suffix{.}\comArg{OrderName}\suffix{.q2.0}
under the name \comArg{SpecificationVariable}. The current specification and other variables 
will be not affected by this operation.
If \comRef{thaw} is used without argument \comArg{StateName} 
the current specification is overwritten by the thawed specification. 
Specifications saved in variables will still not be affected by this operation.
If \comRef{thaw} is used only with argument \comArg{ModuleName} the frozen 
specification will be taken from the file \comArg{ModuleName}\suffix{.q2.0}.
\end{command}

\begin{command}[\comName{undo}]
can be entered at the system's top level to set the system to the state
before the last command that has caused a state change, if there was
any. \comRef{undo} can be used repeatedly to undo several steps of state changes.
It also undoes \comRef{undoUndo}-calls. At the moment, there is no way to backtrack
from single decisions that have been taken while running the completion
process.
\end{command}

\begin{command}[\comName{undoUndo}]
allows to undo the last \comRef{undo}-command at the system's top level. Repeated use of
this command undoes sequences of \comRef{undo}-commands. Chains of undo-calls begin at 
the last user interaction different from an \comRef{undo} or \comRef{undoUndo}.
\end{command}

\noindent 
The CEC-commands \comRef{equal}, \comRef{greater}, \comRef{status}, 
\comRef{setInterpretation} and \comRef{constructor} can be used as 
order pragmas, cf. chapter \ref{OrderPragmas}.
