\documentstyle[german]{article}

\textwidth 17cm
\addtolength{\baselineskip}{1.0ex}
\addtolength{\parsep}{0.5ex}
\oddsidemargin -0.4cm
\evensidemargin -0.4cm
\topmargin 0pt
\headheight 0pt
\headsep 0pt
\footheight 0pt
\footskip 40pt
\textheight 660pt

%----------------------------------------------------------------------
% Macros

% Theorems with roman font
% To declare a theorem-environment using roman font style just say
% \newtheorem{NAME}{TEXT}[COUNTER]
% \def\NAME{\Thm{NAME}{TEXT}}
% or
% \newtheorem{NAME}[OLDNAME]{TEXT}
% \def\NAME{\Thm{OLDNAME}{TEXT}}
\catcode`\@=11
\def\Thm#1#2#3{\refstepcounter
{#2}\@ifnextchar[{\@Ythm{#1}{#2}{#3}}{\@Xthm{#1}{#2}{#3}}}
\def\@Xthm#1#2#3{\@Begintheorem{#1}{#3}{\csname the#2\endcsname}\ignorespaces}
\def\@Ythm#1#2#3[#4]{\@Opargbegintheorem{#1}{#3}{\csname
       the#2\endcsname}{#4}\ignorespaces}
\def\@Begintheorem#1#2#3{\trivlist \item[\hskip \labelsep{\bf #2\ #3}]#1}
\def\@Opargbegintheorem#1#2#3#4{\trivlist
      \item[\hskip \labelsep{\bf #2\ #3\ (#4)}]#1}
\catcode`\@=12

%---------- layout ----------

\newcommand{\nl}{\hfill\\}

%\def\definitionname{Definition}
%\def\theoremname{Theorem}
%\def\propositionname{Proposition}
%\def\corollaryname{Corollary}
%\def\lemmaname{Lemma}
%\def\conjecturename{Conjecture}
%\def\remarkname{Remark}
%\def\examplename{Example}
%\def\proofname{Proof}

\def\definitionname{Definition}%
\def\theoremname{Satz}%
\def\propositionname{Proposition}%
\def\corollaryname{Korollar}%
\def\lemmaname{Lemma}%
\def\conjecturename{Vermutung}%
\def\remarkname{Bemerkung}%
\def\examplename{Beispiel}%
\def\proofname{Beweis}%

\newtheorem{definition}{}
  \def\definition{\Thm{\rm}{definition}{\definitionname}}
\newtheorem{theorem}[definition]{\theoremname}
  \def\theorem{\Thm{\it}{definition}{\theoremname}}
\newtheorem{proposition}[definition]{\propositionname}
  \def\proposition{\Thm{\it}{definition}{\propositionname}}
\newtheorem{corollary}[definition]{\corollaryname}
  \def\corollary{\Thm{\it}{definition}{\corollaryname}}
\newtheorem{lemma}[definition]{\lemmaname}
  \def\lemma{\Thm{\it}{definition}{\lemmaname}}
\newtheorem{conjecture}[definition]{\conjecturename}
  \def\conjecture{\Thm{\it}{definition}{\conjecturename}}
\newtheorem{remark}[definition]{\remarkname}
  \def\remark{\Thm{\rm}{definition}{\remarkname}}
\newtheorem{example}[definition]{\examplename}
  \def\example{\Thm{\rm}{definition}{\examplename}}
\newenvironment{proof}{\smallskip{\em \proofname:}}{\hfill\(\Box\)\\\smallskip}

%---------- internal ----------
\newcommand{\inMath}[1]{\relax\ifmmode{#1}\else${#1}$\fi}

%---------- math ----------

% decl : name * type -> declaration
\newcommand{\decl}[2]{\inMath{{#1}:{#2}}}

% fun : domain * codomain -> function type
\newcommand{\fun}[2]{\inMath{{#1}\rightarrow{#2}}}

% fundecl : function name * domain * codomain -> function declaration
\newcommand{\fundecl}[3]{\decl{#1}{\fun{#2}{#3}}}

% argpos : underscore to indicate argument positions for infix etc.
\newcommand{\argpos}{\underline{\ }}


\newcommand{\implies}{\;\Rightarrow\;}
\newcommand{\dfeq}{\;:=\;}
\newcommand{\dfiff}{\;\:\Longleftrightarrow\;}

% eset : elements -> enumerated set
\newcommand{\eset}[1]{\inMath{\{{#1}\}}}

% pset : variable * predicate -> set
\newcommand{\pset}[2]{\inMath{\{{#1}|{#2}\}}}

% family : element of family * variable in set -> family
\newcommand{\family}[2]{\inMath{({#1})_{#2}}}

% union of sets
\newcommand{\union}[2]{\inMath{{#1}\,\cup\,{#2}}}

% intersection of sets
\newcommand{\intersection}[2]{\inMath{{#1}\,\cap\,{#2}}}

% difference of sets
\newcommand{\difference}[2]{\inMath{{#1}\backslash{#2}}}

% disjointness of sets
\newcommand{\disjoint}[2]{\inMath{\intersection{#1}{#2}=\emptyset}}


% card : set -> its cardinality
\newcommand{\card}[1]{\inMath{|{#1}|}}

\newcommand{\tuple}[1]{\inMath{\langle{#1}\rangle}}
% pair : sth * sth -> pair
\newcommand{\pair}[2]{\tuple{{#1},{#2}}}
% triple : sth * sth * sth -> triple
\newcommand{\triple}[3]{\tuple{{#1},{#2},{#3}}}

% index * from * to -> from =< index =< to
\newcommand{\inrange}[3]{\inMath{{#1}={#2}\ldots{#3}}}

%---------- categories ----------

% Obj : collection of objects in a category
% Mor : collection of morphisms in a category
\newcommand{\Obj}{{\rm Obj}}
\newcommand{\Mor}{{\rm Mor}}

% ObjOf : category -> collection of objects in category
% MorOf : category -> collection of morphisms in category
\newcommand{\ObjOf}[1]{{\Obj\;{#1}}}
\newcommand{\MorOf}[1]{{\Mor\;{#1}}}

% dom : morphism -> its domain
% cod : morphism -> its codomain
% id : object -> identity morphism
% comp : morphism * morphism -> composition of morphisms
\newcommand{\id}[1]{\inMath{\dot{#1}}}
\newcommand{\dom}[1]{\inMath{{#1}^{-}}}
\newcommand{\cod}[1]{\inMath{{#1}^{+}}}
\newcommand{\comp}[2]{\inMath{{#1};{#2}}}

\newcommand{\Cat}{\mbox{\rm CAT}}
\newcommand{\Set}{\mbox{\rm SET}}
\newcommand{\dual}[1]{\inMath{{#1}^{\rm op}}}

%---------- signatures ----------

\newcommand{\emptysig}{\inMath{\emptyset}}

\newcommand{\opdecl}[3]{\fundecl{#1}{#2}{#3}}
\newcommand{\preddecl}[2]{\decl{#1}{#2}}
\newcommand{\vardecl}[2]{\decl{#1}{#2}}
\newcommand{\sigtermdecl}[3]{\inMath{{#2}:_{#1}{#3}}}
\newcommand{\termdecl}[2]{\sigtermdecl{}{#1}{#2}}

\newcommand{\Sign}{\inMath{\rm SIGN}}
\newcommand{\EqSign}{\inMath{\rm EQSIGN}}

\newcommand{\Spec}{\inMath{\rm SPEC}}

\newcommand{\sort}[1]{{\it {#1}}}
\newcommand{\pred}[1]{{\it {#1}}}

\newcommand{\varsOf}[1]{{\it var({#1})}}

%---------- structures ----------

\newcommand{\AlgOf}[1]{\inMath{\rm {#1}\mbox{-}ALG}}
\newcommand{\StructOf}[1]{\inMath{\rm {#1}\mbox{-}STRUCT}}

\newcommand{\reduct}[1]{\inMath{\bar{#1}}}

\newcommand{\T}[1]{\inMath{T_{#1}}}
\newcommand{\TSig}{\T{\Sigma}}
\newcommand{\TSigX}{\T{\Sigma(X)}}
\newcommand{\TSigY}{\T{\Sigma(Y)}}

\newcommand{\Atoms}[1]{\inMath{At_{#1}}}

\newcommand{\HerbStruct}[2]{\inMath{HS({#1},{#2})}}

\newcommand{\evalAss}[1]{\inMath{{#1}^{\#}}}
\newcommand{\evalGnd}[1]{\inMath{\#_{#1}}}

%---------- specifications ----------

\newcommand{\eq}{\approx}
\newcommand{\eqS}{\dot{\approx}}
\newcommand{\clause}[2]{\inMath{{#1}\rightarrow{#2}}}
\newcommand{\vclause}[3]{\inMath{\forall{#1}.{#2}\rightarrow{#3}}}
\newcommand{\vuclause}[2]{\inMath{\forall{#1}.{#2}}}

%---------- models ----------

\newcommand{\InitOf}[1]{\inMath{{\cal I}({#1})}}

%---------- theories ----------

\newcommand{\ThOf}[1]{\inMath{{\rm Th}({#1})}}
\newcommand{\IThOf}[1]{\inMath{{\rm ITh}({#1})}}

\newcommand{\entails}{\inMath{\;\models\;}}
\newcommand{\indEntails}{\inMath{\;\models_I\;}}

%---------- terms ----------

\newcommand{\subterm}[2]{\inMath{{#1}/{#2}}}

%---------- rewriting ----------

\newcommand{\rew}{\rightarrow}
\newcommand{\rewR}{\stackrel{\epsilon}{\rightarrow}}
\newcommand{\rewT}{\stackrel{+}{\rightarrow}}
\newcommand{\rewRT}{\stackrel{*}{\rightarrow}}
\newcommand{\rewI}{\leftarrow}
\newcommand{\rewIR}{\stackrel{\epsilon}{\leftarrow}}
\newcommand{\rewIT}{\stackrel{+}{\leftarrow}}
\newcommand{\rewIRT}{\stackrel{*}{\leftarrow}}
\newcommand{\rewS}{\leftrightarrow}
\newcommand{\rewSR}{\stackrel{\epsilon}{\leftrightarrow}}
\newcommand{\rewST}{\stackrel{+}{\leftrightarrow}}
\newcommand{\rewSRT}{\stackrel{*}{\leftrightarrow}}

\newcommand{\rewG}{\diamondsuit}
\newcommand{\rewGRT}{\stackrel{*}{\diamondsuit}}

\newcommand{\rewD}{\downarrow}

%---------- proof ----------

\newcommand{\ProofsOf}[1]{{\rm Proofs}({#1})}
\newcommand{\trans}{\inMath{\vdash}}

\newcommand{\Proofs}{\inMath{{\cal P}}}
\newcommand{\iProofs}{\inMath{\Proofs_i}}
\newcommand{\diProofs}{\inMath{\Proofs_d}}

\newcommand{\proofGT}{\succ^P}
\newcommand{\proofGE}{\succeq^P}
\newcommand{\proofLT}{\prec^P}
\newcommand{\proofLE}{\preceq^P}

\newcommand{\cplGT}{\succ^c}
\newcommand{\cplGE}{\succeq^c}
\newcommand{\cplLT}{\prec^c}
\newcommand{\cplLE}{\preceq^c}

%---------- type predicates ----------

\newcommand{\tp}[1]{\inMath{\mbox{\it is-}{#1}}}
\newcommand{\tc}[1]{\inMath{{\it tc}({#1})}}
\newcommand{\TC}[1]{\inMath{{\it TC}({#1})}}
\newcommand{\TS}{\inMath{\it TS}}
\newcommand{\TF}[1]{\inMath{{\TS_{#1}}}}

% end of macros
%----------------------------------------------------------------------

\newcommand{\cand}{\wedge}

\begin{document}

\section{Induktionsbeweise mit CEC}

\subsection{Spezifikationserweiterung}

Um Induktionsbeweise f"uhren zu k"onnen, wird die Spezifikation erweitert.
\begin{itemize}
\item F"ur jede Sorte $s$ wird ein Typpr"adikat\footnote{
      Pr"adikate werden hier als Operatoren kodiert, d.h.\ ein Pr"adikat
      \preddecl{\pi}{s_1 \times \ldots \times s_n} wird als Operator
      \opdecl{\pi}{s_1 \times \ldots \times s_n}{\sort{pred}} kodiert, und
      ein Atom \( \pi(t_1,\ldots,t_n) \) wird zu der Gleichung
      \( \pi(t_1,\ldots,tn) = \pred{true} \),
      wobei \sort{pred} eine neue Sorte und 
      \opdecl{\pred{true}}{}{\sort{pred}} ein neuer Operator ist.
      }
      \( \preddecl{\tp{s}}{s} \),
\item f"ur jede Untersortenbeziehung\footnote{
      Im CEC wird diese Gleichung f"ur die Injektion
      erzeugt, die bei der "Ubersetzung der Untersortenbeziehung 
      in eine many-sorted Spezifikation entsteht.
      }
      $s'<s$ wird eine bedingte Gleichung
      \[ \clause{\tp{s'}(x)}{\tp{s}(x)} \]
      und
\item f"ur jeden Operator \opdecl{\omega}{s_1 \times \ldots \times s_n}{s}
      wird eine Typgleichung
      \[ \clause{\tp{s_1}(x_1) \cand \ldots \cand \tp{s_n}(x_n)}
                {\tp{s}(\omega(x_1,\ldots,x_n))}
      \]
\end{itemize}
hinzugef"ugt.

Diese Erweiterung hat die folgenden Eigenschaften:
\begin{itemize}
\item Die Erweiterung ist persistent, d.h.\ die G"ultigkeit von Gleichungen
      in der alten Spezifikation wird nicht ver"andert, und
\item \( \tp{s}(t) \) ist genau dann beweisbar, wenn $t$ zu einem Grundterm 
      "aquivalent ist.
\end{itemize}


\subsection{Basissignatur}

In der Spezifikation mu"s eine Teilsignatur angegeben werden, genannt
Basissignatur, so da"s es zu jedem Grundterm in der urspr"unglichen
Signatur einen "aquivalenten Grundterm in der Basissignatur gibt. F"ur
das erzeugte TES mu"s sogar gelten, da"s sich jeder Grundterm auf einen
Basisgrundterm reduzieren l"a"st, also m"ussen alle Grundnormalformen
Basisterme sein.

Die Basissignatur wird mit Hilfe von Deklarationen der Form
\[\rm
\begin{array}{l@{\;:==\;}l}
<declaration> & <sort> {\tt generatedBy} <operator>,\ldots,<operator>
\end{array}
\]
angegeben.

Es mu"s zu jeder Sorte eine Obersorte geben, die eine Deklaration
besitzt.
\[\rm s \quad{\tt generatedBy}\; \ldots, Op, \ldots \]
bedeutet, da"s alle Operatordeklarationen f"ur $Op$ mit einer Zielsorte,
die Untersorte von $s$ ist, in der Basissignatur enthalten sind.
Im order-sorted Fall kann eine Signatur mehrere Deklarationen f"ur einen
Operator enthalten. Es ist hier m"oglich, f"ur einen Operator nur einen Teil
dieser Deklarationen in der Basissignatur zu haben.

Daraus ergibt sich f"ur den reinen many-sorted Fall, da"s es zu jeder
Sorte genau eine solche Deklaration geben mu"s, und die Operatoren in
der Basissignatur sind dann genau die in den Deklarationen aufgef"uhrten.

\medskip
{\bf Beispiel:}
\begin{verse}\tt
stack generatedBy empty, push.       \\
                                     \hfill\\
nestack < stack.                     \\
                                     \hfill\\
op empty : stack.                    \\
op push : (elem * stack -> nestack). \\
op top : (nestack -> stack).
\end{verse}
Die Basissignatur sieht dann so aus:
\begin{verse}\tt
nestack < stack.                     \\
                                     \hfill\\
op empty : stack.                    \\
op push : (elem * stack -> nestack). \\
\end{verse}

F"ur das erzeugte TES wird gefordert, da"s sich jeder Grundterm auf
einen Basisgrundterm reduzieren l"a"st.
Da"s die Forderung erf"ullt wird, wird bei der Vervollst"andigung durch
eine spezielle Strategie garantiert. Es werden nur Typgleichungen der
Form
\[ \clause{\tp{s_1}(x_1) \cand \ldots \cand \tp{s_n}(x_n)}
          {\tp{s}(\omega(x_1,\ldots,x_n))} \]
zu Regeln orientiert, deren Operatoren $\omega$ in der Basissignatur
liegen. Alle anderen Typgleichungen werden nichtoperational. 

{\em Beachte:}
\( \tp{s}(\omega(x_1,\ldots,x_n)) \)
ist ein order-sorted Term. In der many-sorted Form k"onnen noch
Injektionen oberhalb von $\omega$ stehen.


\subsection{Induktive Beweise}

Sei \clause{\Gamma}{\varphi} eine bedingte Gleichung, und
\varsOf{\clause{\Gamma}{\varphi}}=\eset{x_1,\ldots,x_n}
Definiere
\[ \tau(\clause{\Gamma}{\varphi}) 
   \dfeq \clause{\tp{s_1}(x_1) \cand \ldots \cand \tp{s_n}(x_n) \cand \Gamma}
                {\varphi}
\]

Eine bedingte Gleichung $C$ ist genau dann im initialen Modell g"ultig,
wenn $\tau(C)$ im freien Modell g"ultig ist.

{\em Beachte:}
$C$ ist nicht notwendig in allen minimalen Modellen g"ultig.

Eine bedingte Gleichung $C$ ist genau dann im freien Modell \T{D(X)} einer
Spezifikation $D$ g"ultig, wenn sich die deduktive Theorie\footnote{
Die deduktive Theorie einer Menge von bedingten Gleichungen ist die Menge der
{\em unbedingten} Gleichungen, die sich daraus ableiten lassen.
}
von $E$ durch Hinzuf"ugen von $C$ nicht "andert, formal

\[ \T{D(X)} \models C  \iff  \ThOf{E}=\ThOf{\union{D}{\eset{C}}} \]

{\em Beachte:}
$C$ ist dann nicht notwendig in allen Modellen g"ultig.

Die Bedingung
\( \ThOf{E}=\ThOf{\union{D}{\eset{C}}} \)
l"a"st sich mit Hilfe von CEC wie folgt nachweisen:
\begin{itemize}
\item $E$ vervollst"andigen. Dann ist \ThOf{E} durch Rewriting entscheidbar.
\item $C$ als neue Gleichung hinzuf"ugen.
\item vervollst"andigen, dabei werden alle bedingten Gleichungen nichtoperational
      gemacht. Treten dabei keine neuen unbedingten Gleichungen auf, so gilt
      \( \ThOf{E}=\ThOf{\union{D}{\eset{C}}} \),
      da weiterhin die gleichen Regeln vorhanden sind.
\item Tritt eine neue unbedingte Gleichung $\varphi$ auf, die nicht durch
      Rewriting eliminiert werden kann, so gilt
      \( \ThOf{E} \neq \ThOf{\union{D}{\eset{C}}} \),
      denn \( \varphi \in \ThOf{\union{E}{\eset{C}}} \),
      aber \( \varphi \not\in \ThOf{E} \).
\end{itemize}
Das Verfahren ist refutationsvollst"andig. Angenommen die
Vervollst"andigung terminiert nicht. Dann ist die Menge der
persistenten Regeln $R$ ein vollst"andiges TES f"ur \union{E}{\eset{C}},
$R$ "andert sich aber nicht, daher gilt
\( \ThOf{E}=\ThOf{\union{D}{\eset{C}}} \).


\subsection{Bedienung}

Um induktive Beweise mit dem CEC-System zu f"uhren sind folgende Schritte
notwendig.
\begin{enumerate}
\item {\em Deklarationen f"ur Basissignatur in der Spezifikation erg"anzen}.

\item {\em Spezifikationserweiterung einschalten}.\hfill\\
      Dazu das Prolog-Goal
      \begin{verse}
         {\tt domainConstraints:==on.}
      \end{verse}
      auswerten.
\item {\em Spezifikation einlesen}.\hfill\\
      Dabei wird die Spezifikation automatisch um Typisierungsinformationen
      erweitert.

\item {\em Vervollst"andigen}.\hfill\\
      Die Gleichungen der urspr"unglichen Spezifikation k"onnen wie zuvor
      behandelt werden. Die Typgleichungen werden entsprechend der in der
      Spezifikation angegebenen Basissignatur behandelt. Der Benutzer mu"s
      bei nichtoperationalen Gleichungen nur noch die Bedingung zum Super-
      positionieren ausw"ahlen. Ist die Basissignatur nicht korrekt gew"ahlt,
      tritt eine unbedingte Gleichung auf, die nichtoperational gemacht
      werden mu"s. Das System ist dann nicht vollst"andig.\footnote{
      Zur Zeit wird das nicht deutlich angezeigt.}

\item {\em Beweisen}.\hfill\\
      Es k"onnen mehrere bedingte Gleichungen simultan bewiesen werden.
      Die Kommandos lauten:
      \begin{verse}
        {\tt prove([ $E_1$, $\ldots$, $E_n$]).} \\
        {\tt prove($E$).}
      \end{verse}
      \( E_1, \ldots, E_n,E \) sind bedingte Gleichungen. 
      F"ur jede Gleichung $E$ wird \( \tau(E) \) zur Spezifikation hinzugef"ugt,
      dann wird die Vervollst"andigung gestartet.
      Alle bedingten Gleichungen werden automatisch nichtoperational,
      der Benutzer mu"s nur noch die Bedingung zum Superpositionieren w"ahlen.
      Die Wahl einer Bedingung
        \( \tp{s}(x) \)
      kann man als Wahl von $x$ als Induktionsvariable betrachten.
      \begin{itemize}
      \item Tritt eine unbedingte Gleichung auf, ist dies ein Fehlschlag, die
            Gleichungen gelten im initialen Modell nicht.
      \item Terminiert die Vervollst"andigung, sind die Gleichungen
            \( \tau(E) \) im freien Modell g"ultig. Sie bleiben als 
            nichtoperationale Gleichungen erhalten und werden bei weiteren
            Beweisen als Lemmata benutzt.
      \end{itemize}
\end{enumerate}


\subsection{Spezifikationen aus mehreren Modulen}

Es ist erlaubt, Spezifikationen wie bisher zu erweitern
oder zu kombinieren.
Beim Erweitern darf die Basisspezifikation
nur um solche Operatoren erweitert werden, deren Zielsorte nicht in
der urspr"unglichen Spezifikation liegt.
Dadurch wird erzwungen, da"s die Erweiterung der Spezifikation
hinreichend vollst"andig ist.
Beim Kombinieren m"ussen die Deklarationen bez"uglich der
Basissignatur zueinander passen, d.h.\ solche Operatoren, deren
Zielsorte in beiden Signaturen enthalten ist, m"ussen entweder in
beiden Basissignaturen oder in keiner enthalten sein.

Induktive Beweise d"urfen erst in der endg"ultigen Spezifikation gef"uhrt
werden. Sonst besteht die M"oglichkeit, da"s die bein Beweisen hinzugef"ugten
nichtoperationalen Gleichungen beim vervollst"andigen der Erweiterung
neue Regeln erzeugen, die die Gleichungstheorie ver"andern.

\medskip
{\bf Beispiel:} Vervollst"andige
\begin{verse}\tt
module test1.            \\
                         \hfill\\
nat generatedBy 0,s.     \\
op 0 : nat.              \\
op s : (nat -> nat).     \\
\end{verse}

Beweise die Injektivit"at von {\tt s} im initialen Modell
\begin{verse}\tt
prove(s(x)=s(y)=>x=y).
\end{verse}

Nun vervollst"andige
\begin{verse}\tt
module test2 using test1. \\
                          \hfill\\
s(s(x))=s(x).             \\
\end{verse}

Dann wird die neue Gleichung
\begin{verse}\tt
s(x)=x
\end{verse}
erzeugt, obwohl sie nicht in {\tt test2} gilt.

\end{document}
