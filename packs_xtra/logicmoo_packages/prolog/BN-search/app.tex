%& latex
% this is both a latex file and a quintus prolog file
% the only difference is that the latex version comments out the following
% line:
 /* 
\documentstyle[12pt,fullpage]{article}
%\setlength{\textwidth}{15.5cm}
%\setlength{\textheight}{22cm}
\title{Probabilistic conflicts in searching  
Bayesian networks\\Prolog Code}
\author{David Poole\thanks{Scholar, Canadian Institute for Advanced Research}\\
Department of Computer Science,\\ 
University of British Columbia,\\
2366 Main Mall,\\
Vancouver, B.C., Canada V6T 1Z4\\ 
poole@cs.ubc.ca}
\begin{document}
\maketitle
\appendix
\section{A Prolog Implementation} \label{code}

This code implements a bottom-up depth-first depth-bounded search to
find the most likely possible worlds that are consistent with the
observations. Here we exploit the pattern-matching of Prolog, without
utilizing the declarative nature or the search of the Language.  All
of the code here is deterministic; this program does not backtrack.
This is done here by the use of the Prolog cut (!). N.B. we could also
write a program that uses Prolog search (and no cuts) to do the
depth-first search. This was not done as the following code is more
efficient (by our tests), and can be more directly translated into
committed-choice parallel logic programming languages or more traditional
languages such as Lisp.

\subsection{Network Representation}
The Bayesian network is described using the relations:
\begin{description}
\item $\em nextvar(I1,I2)$ means that variable $\em I2$ is the next variable 
after $\em I1$ in the total order of variables. The first (dummy) variable
is $\em init$. $\em prevvar(I2,I1)$ is the inverse of $\em nextvar$ (needed 
because Prolog is not a real logic programming language).
\item $\em vals(I,VI)$ is true if $\em VI$ is the list of variables of variable $I$.
\item $\em parents(I,A,F)$ is true if $F$ is the set of values of parents of 
variable $I$, given the list $A$ of values of the ancestors of $I$ in
reverse order (i.e, the first element of the list is the value of the
variable before $V$).
\item $\em prob(E,F,P)$ is true if proposition $E$ given parents $F$ has 
probability $P$. That is if $\em p(E|F)=P$. A proposition $E$ is written
as an equality $\em Variable=Value$.
\item $\em observed(E)$ is true if proposition $E$ of the form 
{\em Variable=Value} is observed.
\item $\em miniscule(P)$ is true if $P$ is below the threshold for searching 
in the depth-bounded search.
\item $\em interesting(E)$ is true if proposition $E$ of the form 
{\em Variable=Value} is an interesting value to print out when printing
possible worlds.
\end{description}

\subsection{Search Procedure}
A possible world is represented as $\em pw(P,Vs)$ which represents the
possible world with values given by the list $\em Vs$ and probability $P$.
$\em Vs$ is the list of the values of the variables in reverse order.

The found possible worlds that are consistent with the observations
are represented as $\em wf(IM,FM,MW,FND)$ where $\em IM$ is the mass of the
possible worlds pruned by inconsistency, $\em FM$ is the sum of the
probabilities of the possible worlds found (these are all consistent
with the observations) $\em MW$ is the mass of pruned miniscule worlds and
$\em FND$ is a list of these worlds.

The state of the computation is represented as the tuple
\[\em state(D,H,C)\]
where
\begin{description}
\item[$D$] is the worlds found so far.
\item[$H$] is the heuristic function to be used.
\item[$C$] is the conflict set.
\end{description}


The top level procedure is to search through the values $\em VIs$ of
variable $I$ and all subsequent variables:
\[\em chain(I,VIs,VAs,PAs,S0,S1)\]where 
\begin{description}
\item[$I$] is
the current variable; 
\item[$\em VIs$] is the list remaining values of $I$ to test, 
\item[$\em VAs$] is the list of the values of the ancestors of $I$ (in reverse 
order, i.e., the first element of the list is the value of the
immediate predecessor of $I$);
\item[$\em PAs=P(VAs)$];
\item[$\em S0$] is the state before.
\item[$\em S1$] is the state after.
\end{description}
N.B. the first clause is when we are backtracking to a previous
variable. In this case the heuristic function should be updated, if
$I$ is a pivot variable.
\index{chain}
\begin{verbatim} */
chain(I,[],_,_,state(D,H0,C),state(D,H1,C)) :-
   hpivot(I,C,P), !,
   H1 is H0*P.
chain(_,[],_,_,S,S) :- !.
chain(I,VIs,VAs,PAs,S0,S1) :-
   parents(I,VAs,Parents),
   remove_max(I,Parents,V1,VIs,VRs,PV1),
   P1 is PAs*PV1, !,
   test(I,V1,P1,VRs,VAs,PAs,S0,S1).
/* \end{verbatim}

\[\em remove\_max(I,Parents,Vmax,VIs,VRs,Pmax)\]
is true if $\em Vmax$ is the element of $\em VIs$ such that
$\em P(I=Vmax|Parents)=Pmax$ is maximal. $\em VRs$ is the remaining elements
of $\em VIs$ once $\em Vmax$ is removed.

\index{chain}
\begin{verbatim} */
remove_max(I,Parents,V1,[V1],[],MaxProb) :-
   prob(I=V1,Parents,MaxProb),!.
remove_max(I,Parents,VMax,[V1|VIs],VRs,MaxProb) :-
   prob(I=V1,Parents,PV1),
   remove_max(I,Parents,V2,VIs,V2Rs,PV2),
   ( PV1 > PV2 -> VMax=V1, MaxProb=PV1, VRs=VIs
    ; VMax=V2, MaxProb=PV2, VRs=[V1|V2Rs] 
    ).
/* \end{verbatim}

The procedure $\em test$ is to test one of the values of the variable
being tested, to add it to the collection of inconsistent values,
elements that remain on the (pseudo) queue or recurse to find all of
the possible worlds that are its descendents.
\[\em test(I,V1,P1,VIs,VAs,PAs,S0,S1)\]
is true when
\begin{description}
\item[$I$] is
the current variable; 
\item[$\em V1$] is a possible value for variable $I$;
\item[$\em P1$] is the probability $I$ having value $V$ and all 
other variables having values given by $\em VAs$.
\item[$\em VIs$] is the list remaining values of $I$ to test (after $\em V1$), 
\item[$\em VAs$] is the list of the values of the ancestors of $I$ (in 
reverse order, i.e., the first element of the list is the value of the
immediate predecessor of $I$);
\item[$\em PAs=P(VAs)$];
\item[$\em S0$] is the state before.
\item[$\em S1$] is the state after.
\end{description}

\index{start}
\begin{verbatim} */
test(I,V1,P1,VIs,VAs,PAs,state(wf(IM0,FM,MW,FND),H0,C0),S2) :-
   inconsis_obs(I=V1), !,
   (miniscule(P1*H0) -> C0=C1
   ; expand_conflict_set(I,VAs,C0,C1) ),
   IM1 is IM0 + P1,
   chain(I,VIs,VAs,PAs,state(wf(IM1,FM,MW,FND),H0,C1),S2).
test(I,_,P1,VIs,VAs,PAs,state(wf(IM0,FM,MW0,FND),H0,C0),S2) :-
   miniscule(P1*H0), !,
   MW1 is MW0+P1*H0,
   IM1 is IM0+P1*(1-H0),
   chain(I,VIs,VAs,PAs,state(wf(IM1,FM,MW1,FND),H0,C0),S2).
test(I,V1,P1,VIs,VAs,PAs,state(D0,H0,C0),S2) :-
   nextvar(I,I1), !,
   ( hpivot(I1,C0,PP) -> H1 is H0/PP
   ; H1=H0 ),
   vals(I1,VI1), !,
   chain(I1,VI1,[V1|VAs],P1,state(D0,H1,C0),S1), !,
   chain(I,VIs,VAs,PAs,S1,S2).
test(I,V1,P1,VIs,VAs,PAs,state(wf(IM0,FM,MW,FND),H0,C0),S1) :-
   FM1 is FM+P1, !,
   chain(I,VIs,VAs,PAs,
         state(wf(IM0,FM1,MW,[pw(P1,[V1|VAs])|FND]),H0,C0),S1).
/* \end{verbatim}

$\em inconsis\_obs(E)$ is true if proposition $E$ is inconsistent with the
observations (it may have probability zero, and not be caught by this
test; this means directly contradicted by the observation,
independently of the background knowledge stated as probabilities).
\index{inconsis\_obs}
\begin{verbatim} */
inconsis_obs(I=V) :-
    observed(I=VO),
    \+ V=VO.
/* \end{verbatim}

\subsection{Conflict Sets}
A conflict set is a list of conflicts. Each conflict is represented as
\[\em conflict(MProb,LeastVar,MaxVar,Conflict)\]
where $\em Conflict$ is the set of variables that are in conflict.
$\em LeastVar$ is the least (in the total ordering of the variables)
variable in the conflict. $\em MaxVar$ is the largest (in the total
ordering of the variables) variable in the conflict. $\em MProb$ is a
bound on the maximum probability that an assignment of values to
variables in the conflict can take.

\[\em expand\_conflict\_set(I,VAs,C0,C1)\]
where $I$ is a variable whose value is unexpected given ancestor's values 
$\em VAs$, expands the conflict set from $\em C0$ to $\em C1$.

N.B. we have to make sure that each variable is only in one conflict
(otherwise the conflicts are not independent and we cannot multiple
the heuristic values of the conflicts). This is the role of the first
clause.
\index{expand\_conflict\_set}
\begin{verbatim} */
expand_conflict_set(I,_,C0,C0) :-
   inconflict(I,C0),!.
expand_conflict_set(I,VAs,C0,C1) :-
   observed(I=Vobs),
   writeln(['looking for conflict: ',I=Vobs]),
   extract_counter(I,Vobs,VAs,C0,[],C,MP,I,LV),
   (MP >= 1.0 -> C1=C0 ,
     writeln(['Observed ',I=Vobs,' Conflict ',C,' rejected. MP=',MP])
   ; C1 = [conflict(MP,LV,I,C)|C0],
     writeln(['*** Conflict found: ',C]),
     writeln(['*** Heuristic value = ',MP,'; Minvar=',LV,'; Maxvar =',I]) ).
/* \end{verbatim}

The following us used to test the case where there are no conflicts.
\begin{verbatim}
expand_conflict_set(_,_,C0,C0).
\end{verbatim}

$\em hpivot(I,CS,P)$ is true if $I$ is the least variable in a conflict in
$\em CS$. $P$ is the heuristic value of the conflict. Analogously,
$\em upivot(I,CS,P)$ is true if $I$ is the largest variable in a conflict
in $\em CS$, with probability $P$.
\index{hpivot}
\index{upivot}
\begin{verbatim} */
hpivot(I,CS,P) :-
   member(conflict(P,I,_,_),CS).
upivot(I,CS,P) :-
   member(conflict(P,_,I,_),CS).
/* \end{verbatim}

$\em inconflict(I,CS)$ is true if variable $I$ is in a conflict in 
conflict set $\em CS$.
\index{inconflict}
\begin{verbatim} */
inconflict(I,CS) :-
   member(conflict(_,_,_,C),CS), member(I,C),!.
/* \end{verbatim}
\subsection{Finding Conflicts}

\[\em extract\_counter(I,Vobs,VAs,PCon,C0,C1,Prob,LV0,LV1)\]
is true if
$\em I=Vobs$ is unexpected given ancestor values
$\em VAs$ and so the conflict grows from $\em C0$
to $\em C1$.  $\em Prob$ the the probability
bound of the counter found. $\em LV1$ is the least variable in the
conflict. $\em PCon$ is the list of previously found conflicts.

\index{extract\_counter}
\begin{verbatim} */
extract_counter0(I,Vobs,_,_,C0,C0,0,LV0,LV0) :-
   inconsis_obs(I=Vobs),!.
extract_counter0(I,Vobs,VAs,PCon,C0,C1,Prob,LV0,LV1) :-
   extract_counter(I,Vobs,VAs,PCon,C0,C1,Prob,LV0,LV1).

extract_counter(I,Vobs,VAs,PCon,C0,C2,Prob,LV0,LV1) :-
   allof(cp(PV,P),prob(I=Vobs,PV,P),Vpars),
   counter_parents(I,Vpars,VAs,PCon,C0,C1,0,Prob,LV0,LV1),
   insert(I,C1,C2).
/* \end{verbatim}
   
\[\em counter\_parents(I,Vpars,VAs,PCon,C0,C1,P0,P1,LV0,LV1)\]
is true if $\em Vpars$ is the list of $cp(PV,P)$ where $PV$ is a
parent assignment and $PP$ is the conditional probability of the value
of $I$ given $Pars$, and $P1-P0$ the corresponding probability that
could have produced the observed value.  This results in conflict $\em
C1-C0$.

\index{counter\_parents}
\begin{verbatim} */
counter_parents(_,[],_,_,L,L,P,P,LV,LV).
counter_parents(I,[cp(_,0)|R],VAs,PCon,C0,C1,P0,P2,LV0,LV1) :- !,
   counter_parents(I,R,VAs,PCon,C0,C1,P0,P2,LV0,LV1).
counter_parents(I,[cp(PV,PP)|R],VAs,PCon,C0,C1,P0,P2,LV0,LV1) :-
   parents(I,VAs,PV), !, % values predicted by current partial description
   P1 is P0+PP,
   counter_parents(I,R,VAs,PCon,C0,C1,P1,P2,LV0,LV1).
counter_parents(I,[cp(PV,PP)|R],VAs,PCon,C0,C2,P0,P2,LV0,LV2) :-
   parents(I,Anc,PV),
   find_mismatch(I,Anc,PP,VAs,PCon,C0,C1,P0,P1,LV0,LV1),
   counter_parents(I,R,VAs,PCon,C1,C2,P1,P2,LV1,LV2).
/* \end{verbatim}
   
\[\em find\_mismatch(I,Anc,VAs,PCon,C0,C1,P0,P1,LV0,LV1)\]
is true if $\em VAs$ is the list of
variables that correspond to the first mismatch between expected and
found values.  This results in conflict $\em C1-C0$. N.B. other (perhaps
smaller) conflicts can be found by choosing other mismatches other than the first found.

\index{find\_mismatch}
\begin{verbatim} */
find_mismatch(_,_,_,_,_,C0,C0,1,1,I,I) :- !.
find_mismatch(I,[A|P],PP,[A|Vs],PCon,C0,C1,P0,P1,I,LV1) :- !,
 % predicted value; least variable must be updated
   prevvar(I,I1),
   find_mismatch(I1,P,PP,Vs,PCon,C0,C1,P0,P1,I1,LV1).
find_mismatch(I,[A|P],PP,[A|Vs],PCon,C0,C1,P0,P1,LV0,LV1) :- !,
 % predicted value; least variable not updated
   prevvar(I,I1),
   find_mismatch(I1,P,PP,Vs,PCon,C0,C1,P0,P1,LV0,LV1).
find_mismatch(I,_,_,_,PCon,C0,C0,_,1,_,I) :- 
   inconflict(I,PCon),!.
 % found a clash with a previous conflict
find_mismatch(I,[A|_],PP,[_|Vs],PCon,C0,C2,P0,P1,I,LV1) :- !,
 % found mismatch; least value needs to be updated
   prevvar(I,I1),
   extract_counter0(I1,A,Vs,PCon,C0,C1,Prob,I1,LV1),
   (Prob < 1 ->
        P1 is P0+ PP*Prob, C2=C1
   ; P1 is P0+PP, C2=C0).
find_mismatch(I,[A|_],PP,[_|Vs],PCon,C0,C1,P0,P1,LV0,LV1) :-
   prevvar(I,I1),
   extract_counter0(I1,A,Vs,PCon,C0,C1,Prob,LV0,LV1),
   (Prob < 1 ->
        P1 is P0+ PP*Prob, C2=C1
   ; P1 is P0+PP, C2=C0).
/* \end{verbatim}

\subsection{Information Seeking}

$\em make\_worlds(Wlds)$ searches and returns all the consistent worlds
with probability above the threshold.
\index{start}
\begin{verbatim} */
make_worlds(Wlds) :- 
   nextvar(init,I0),
   vals(I0,VI0),
   chain(I0,VI0,[],1,state(wf(0,0,0,[]),1,[]),state(Wlds,_,_)).
/* \end{verbatim}

To start the search for the possible worlds, you call $\em start$.
\index{start}
\begin{verbatim} */
start :-
   statistics(runtime,_),
   make_worlds(wf(IM,Pfnd,MW,Fnd)),
   statistics(runtime,[_,T]),
   writeln(['Runtime: ',T,' msec.']),
   Qmass is 1-IM-Pfnd,
   writeln(['Prob found = ',Pfnd]),
   writeln(['Inconsistency mass = ', IM]),
   writeln(['Pruned mass =',MW]),
   writeln(['Queue Mass = ',Qmass]),
   Err is Qmass / (2 * (Pfnd+Qmass)),
   writeln(['Error in posterior estimation = ',Err]),
   print_worlds(Pfnd,Qmass,Fnd,_).

print_worlds(_,_,[],0).
print_worlds(Pfnd,Qmass,[pw(P,Vs)|R],N) :-
   print_worlds(Pfnd,Qmass,R,N1),!,
   N is N1+1,
   writeln(['World #',N]),
   writeln(['   Prior Probability = ',P]),
   LB is P / (Pfnd + Qmass),
   UB is P  / Pfnd,
   writeln(['   Posterior Probability = [',LB,',',UB,']']),
   writeln(['   Interesting Values:']),!,
   print_vals(Vs,_).

print_vals([],init).
print_vals([V1|R],NV) :-
   print_vals(R,V),
   nextvar(V,NV),
   (interesting(NV=V1) -> writeln(['      ',NV=V1]) ; true).
/* \end{verbatim}

% this is both a latex file and a quintus prolog file
% the only difference is that the latex version comments out the following
% line:
 /* 
\subsection{Representation of a one thousand bit adder}

The following represents a 100 bit cascaded ripple adder. See
Section \ref{code} for a description of the relations used.
\begin{verbatim} */
nextvar(init,i1(1)).
nextvar(i1(N),i2(N)).
nextvar(i2(N),i3(N)).
nextvar(i3(N),x1ok(N)).
nextvar(x1ok(N),x1(N)).
nextvar(x1(N),x2ok(N)).
nextvar(x2ok(N),x2(N)).
nextvar(x2(N),a1ok(N)).
nextvar(a1ok(N),a1(N)).
nextvar(a1(N),a2ok(N)).
nextvar(a2ok(N),a2(N)).
nextvar(a2(N),o1ok(N)).
nextvar(o1ok(N),o1(N)).
nextvar(o1(N),i1(N1)) :-
   N < 1000,
   N1 is N+1.

prevvar(i1(N1),o1(N)) :- 
   N1>1, !,
   N is N1-1.
prevvar(V1,V2) :- nextvar(V2,V1),!.

vals(i1(_),[on,off]).
vals(i2(_),[on,off]).
vals(i3(_),[on,off]).
vals(x1ok(_),[ok,stuck1,stuck0]).
vals(x1(_),[on,off]).
vals(x2ok(_),[ok,stuck1,stuck0]).
vals(x2(_),[on,off]).
vals(a1ok(_),[ok,stuck1,stuck0]).
vals(a1(_),[on,off]).
vals(a2ok(_),[ok,stuck1,stuck0]).
vals(a2(_),[on,off]).
vals(o1ok(_),[ok,stuck1,stuck0]).
vals(o1(_),[on,off]).

parents(i1(_),_,[]).
parents(i2(_),_,[]).
parents(i3(1),_,[]).
parents(i3(N),[_,_,Vo1|_],[Vo1]) :- N>1.
parents(x1ok(_),_,[]).
parents(x1(_),[X1ok,_,Vi2,Vi1|_],
              [X1ok,Vi2,Vi1]).
parents(x2ok(_),_,[]).
parents(x2(_),[X2ok,Vx1,_,Vi3|_],
              [X2ok,Vx1,Vi3]).
parents(a1ok(_),_,[]).
parents(a1(_),[A1ok,_,_,_,_,_,Vi2,Vi1|_],
              [A1ok,Vi2,Vi1]).
parents(a2ok(_),_,[]).
parents(a2(_),[A2ok,_,_,_,_,Vx1,_,Vi3|_],
              [A2ok,Vx1,Vi3]).
parents(o1ok(_),_,[]).
parents(o1(_),[O1ok,Va2,_,Va1|_],
              [O1ok,Va2,Va1]).


prob(i1(_)=on,[],0).
prob(i1(_)=off,[],1).
prob(i2(_)=on,[],0).
prob(i2(_)=off,[],1).
prob(i3(1)=on,[],0).
prob(i3(1)=off,[],1).
prob(i3(N)=V,[V],1) :- N>1.
prob(i3(N)=on,[off],0) :- N>1.
prob(i3(N)=off,[on],0) :- N>1.

prob(x1ok(_)=V,[],P) :-
   prob_ok(V,P).
prob_ok(ok,0.99999).
prob_ok(stuck1,0.000005).
prob_ok(stuck0,0.000005).

prob(x1(_)=V,Par,Prob) :-
   prob_xorgate(V,Par,Prob).

prob(x2ok(_)=V,[],P) :-
   prob_ok(V,P).
prob(x2(_)=V,Par,Prob) :-
   prob_xorgate(V,Par,Prob).
prob(a1ok(_)=V,[],P) :-
   prob_ok(V,P).
prob(a1(_)=V,Par,Prob) :-
   prob_andgate(V,Par,Prob).
prob(a2ok(_)=V,[],P) :-
   prob_ok(V,P).
prob(a2(_)=V,Par,Prob) :-
   prob_andgate(V,Par,Prob).
prob(o1ok(_)=V,[],P) :-
   prob_ok(V,P).
prob(o1(_)=V,Par,Prob) :-
   prob_orgate(V,Par,Prob).

prob_xorgate(on,[ok,on,on],0).
prob_xorgate(off,[ok,on,on],1).
prob_xorgate(on,[ok,on,off],1).
prob_xorgate(off,[ok,on,off],0).
prob_xorgate(on,[ok,off,on],1).
prob_xorgate(off,[ok,off,on],0).
prob_xorgate(on,[ok,off,off],0).
prob_xorgate(off,[ok,off,off],1).
prob_xorgate(on,[stuck1,_,_],1).
prob_xorgate(off,[stuck1,_,_],0).
prob_xorgate(on,[stuck0,_,_],0).
prob_xorgate(off,[stuck0,_,_],1).

prob_andgate(on,[ok,on,on],1).
prob_andgate(off,[ok,on,on],0).
prob_andgate(on,[ok,on,off],0).
prob_andgate(off,[ok,on,off],1).
prob_andgate(on,[ok,off,_],0).
prob_andgate(off,[ok,off,_],1).
prob_andgate(on,[stuck1,_,_],1).
prob_andgate(off,[stuck1,_,_],0).
prob_andgate(on,[stuck0,_,_],0).
prob_andgate(off,[stuck0,_,_],1).

prob_orgate(on,[ok,on,_],1).
prob_orgate(off,[ok,on,_],0).
prob_orgate(on,[ok,off,on],1).
prob_orgate(off,[ok,off,on],0).
prob_orgate(on,[ok,off,off],0).
prob_orgate(off,[ok,off,off],1).
prob_orgate(on,[stuck1,_,_],1).
prob_orgate(off,[stuck1,_,_],0).
prob_orgate(on,[stuck0,_,_],0).
prob_orgate(off,[stuck0,_,_],1).

%observed(x2(N)=off) :- N =\= 300, N=\=700.
%observed(x2(300)=on).
%observed(x2(700)=on).

 observed(x2(N)=off) :- N =\= 500.
 observed(x2(500)=on).

interesting(_=stuck1).
interesting(_=stuck0).

miniscule(P) :- 
   % P < 0.0000000000001. % appropriate bound for double errors
   P < 0.000001.     % appropriate bound for single errors
/* \end{verbatim} 
*/

% this is both a latex file and a quintus prolog file
% the only difference is that the latex version comments out the following
% line:
 /* 
\section{Auxiliary Definitions}

$member(X,L)$ is true if $X$ is a member of list $L$.
\begin{verbatim} */
member(X,[X|_]).
member(X,[_|L]) :-
   member(X,L).
/* \end{verbatim}

$insert(X,L1,L2)$ is true if $L2$ is the result of inserting $X$ into $L1$.
\begin{verbatim} */
insert(X,[],[X]).
insert(X,[X|L],[X|L]) :- !.
insert(X,[A|L],[A|L1]) :-
   insert(X,L,L1).

intersect([],_,[]).
intersect([A|L],B,[A|C]) :-
   member(A,B),!,
   intersect(L,B,C).
intersect([_|L],B,C) :-
   intersect(L,B,C).
/* \end{verbatim}

$min(A,B,C)$ is true if $C$ is the minimum of $A$ and $B$.
\index{start}
\begin{verbatim} */
min(A,B,A) :- A =< B.
min(A,B,B) :- A > B.
/* \end{verbatim}

$max(A,B,C)$ is true if $C$ is the maximum of $A$ and $B$.
\index{start}
\begin{verbatim} */
max(A,B,A) :- A >= B.
max(A,B,B) :- A < B.
/* \end{verbatim}

$power(X,N,P)$ is true if $N$ is a postive integer, and $P$ is $X$ multiplies by itself $N$ times (i.e. $P=X^N$).
\index{start}
\begin{verbatim} */
power(X,1,X).
power(X,N,V) :- 
   even(N),
   N>1,
   X2 is X*X,
   N2 is N //2,
   power(X2,N2,V).
power(X,N,V) :-
   odd(N),
   N>1,
   N1 is N-1,
   power(X,N1,V1),
   V is V1 * X.
% odd(N) is true if N evaluates to an odd integer
odd(N) :-
   1 is abs(N) mod 2.
% even(N) is true if N evaluates to an even integer
even(N) :-
   0 is abs(N) mod 2.
% ? odd(77*3).
/* \end{verbatim}

$newindex(N)$ returns a new value for $N$ each time it is called.
\index{start}
\begin{verbatim} */
:- dynamic previndex/1.
previndex(0).

newindex(I1) :-
   retract(previndex(I)),
   I1 is I+1,
   assert(previndex(I1)).
/* \end{verbatim}

$accumulate(Goal,Init,Prev,Acc,Next,Result)$ accumulates information for each success of Goal. Init is the start. Acc is a predicate that shows how to derive Next accumulation from Prev accumulation. Result is the final accumulation.

\index{start}
\begin{verbatim} */
accumulate(G,I,P,CN,N,Res) :-
   newindex(Ind),
   accumulate(Ind,G,I,P,CN,N,Res).

accumulate(Ind,G,I,P,CN,N,_) :-
   assert(result(Ind,I)),
   G,
   retract(result(Ind,P)),
   CN,
   assert(result(Ind,N)),
   fail.
accumulate(Ind,_,_,_,_,_,Res) :-
   retract(result(Ind,Res)).

:- dynamic p/1.
p(3).
p(5).
p(7).
p(1).
% ? accumulate(p(X),0,P,N is P+X,N,Res).
% ? accumulate(p(X),1,P,N is P*X,N,Res).
% ? accumulate(p(X),[],P,N=[X|P],N,Res).
% ? accumulate(p(X),L-L,P1-P2,P2=[X|P],P1-P,Res-[]).
/* \end{verbatim}

$allof(X,G,Res)$ is true if $Res$ is the list of $X$ such that $G$ succeeds.
\index{start}
\begin{verbatim} */
allof(X,G,Res) :-
   accumulate(G,L-L,P1-[X|P],true,P1-P,Res-[]).
%? allof(X,p(X),L).
%? allof(together(X,Y),append(_,[X,Y|_],[a,b,c,d]),L).
%---------
/* \end{verbatim}

$writeln(L)$ is true if $L$ is a list of items to be written on a line, followed by a newline.
\index{start}
\begin{verbatim} */
writeln([]) :- nl.
writeln([H|T]) :- write(H), writeln(T).
/* \end{verbatim}
*/

\end{document}
*/
