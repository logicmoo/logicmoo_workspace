\documentclass[a4paper,10pt]{scrartcl}
\RequirePackage[hyperindex]{hyperref}
\usepackage[pdftex]{graphicx}
\usepackage{amssymb}
\usepackage{amsmath}
\newcommand\cI{{\cal I}}
\newcommand\cT{{\cal T}}
\newcommand\cA{{\cal A}}
\newcommand\cK{{\cal K}}
\newcommand\cC{{\cal C}}
\newcommand\cL{{\cal L}}
\newcommand\fluffy{{\mathit{fluffy}}}

\newenvironment{example}{}{}
\newenvironment{definition}{}{}

\begin{document}
\title{TRILL on SWISH Manual}
\maketitle

\section{Syntax}
\label{syn}

Description Logics (DLs) are knowledge representation formalisms that are at the basis of the Semantic Web \cite{DBLP:conf/dlog/2003handbook,dlchap} and are used for modelling ontologies.
They are represented using a syntax based on concepts, basically sets of individuals of the domain, and roles, sets of pairs of individuals
of the domain. A more formal description can be found in the Appendix \ref{app:dl}.

TRILL allows the use of two different syntaxes used together or individually:
\begin{itemize}
 \item RDF/XML
 \item Prolog syntax 
\end{itemize}

RDF/XML syntax can be used by exploiting the predicate \verb|owl_rdf/1|. For example:
\begin{small}
\begin{verbatim}
owl_rdf('
<?xml version="1.0"?>

<!DOCTYPE rdf:RDF [
    <!ENTITY owl "http://www.w3.org/2002/07/owl#" >
    <!ENTITY xsd "http://www.w3.org/2001/XMLSchema#" >
    <!ENTITY rdfs "http://www.w3.org/2000/01/rdf-schema#" >
    <!ENTITY rdf "http://www.w3.org/1999/02/22-rdf-syntax-ns#" >
]>
<rdf:RDF xmlns="http://here.the.IRI.of.your.ontology#"
     xmlns:rdf="http://www.w3.org/1999/02/22-rdf-syntax-ns#"
     xmlns:owl="http://www.w3.org/2002/07/owl#"
     xmlns:xsd="http://www.w3.org/2001/XMLSchema#"
     xmlns:rdfs="http://www.w3.org/2000/01/rdf-schema#">
    <owl:Ontology rdf:about="http://here.the.IRI.of.your.ontology"/>

    <!-- 
    Axioms
    -->

</rdf:RDF>
').
\end{verbatim}
\end{small}

For a brief introduction on RDF/XML syntax see \textit{RDF/XML syntax and tools} section below (Sec. \ref{rdfxml-syn}).

Note that each single \verb|owl_rdf/1| must be self contained and well formatted, it must start and end with \verb|rdf:RDF| tag and contain all necessary declarations (namespaces, entities, ...).


An example of the combination of both syntaxes is shown the example \href{http://trill-sw.eu/example/trill/johnEmployee.pl}{\texttt{johnEmployee.pl}}. It models that \textit{john} is an \textit{employee} and that employees are \textit{workers}, which are in turn people (modeled by the concept \textit{person}).
\begin{small}
\begin{verbatim}
owl_rdf('<?xml version="1.0"?>
<rdf:RDF xmlns="http://example.foo#"
     xml:base="http://example.foo"
     xmlns:rdf="http://www.w3.org/1999/02/22-rdf-syntax-ns#"
     xmlns:owl="http://www.w3.org/2002/07/owl#"
     xmlns:xml="http://www.w3.org/XML/1998/namespace"
     xmlns:xsd="http://www.w3.org/2001/XMLSchema#"
     xmlns:rdfs="http://www.w3.org/2000/01/rdf-schema#">
    <owl:Ontology rdf:about="http://example.foo"/>

    <!-- Classes -->
    <owl:Class rdf:about="http://example.foo#worker">
        <rdfs:subClassOf rdf:resource="http://example.foo#person"/>
    </owl:Class>

</rdf:RDF>').

subClassOf('employee','worker').

owl_rdf('<?xml version="1.0"?>
<rdf:RDF xmlns="http://example.foo#"
     xml:base="http://example.foo"
     xmlns:rdf="http://www.w3.org/1999/02/22-rdf-syntax-ns#"
     xmlns:owl="http://www.w3.org/2002/07/owl#"
     xmlns:xml="http://www.w3.org/XML/1998/namespace"
     xmlns:xsd="http://www.w3.org/2001/XMLSchema#"
     xmlns:rdfs="http://www.w3.org/2000/01/rdf-schema#">
    <owl:Ontology rdf:about="http://example.foo"/>
    
    <!-- Individuals -->
    <owl:NamedIndividual rdf:about="http://example.foo#john">
        <rdf:type rdf:resource="http://example.foo#employee"/>
    </owl:NamedIndividual>
</rdf:RDF>').
\end{verbatim}
\end{small}

\subsection{Prolog Syntax}
\label{trill-syn}
\subsubsection{Declarations}


Prolog syntax allows, as in standard OWL, the declaration of classes, properties, etc.
\begin{verbatim}
class("classIRI").
datatype("datatypeIRI").
objectProperty("objectPropertyIRI").
dataProperty("dataPropertyIRI").
annotationProperty("annotationPropertyIRI").
namedIndividual("individualIRI").
\end{verbatim}
However, TRILL properly works also in their absence.

Prolog syntax allows also the declaration of aliases for namespaces by using the \verb|kb_prefix/2| predicate.
\begin{verbatim}
kb_prefix("foo","http://example.foo#").
\end{verbatim}
After this declaration, the prefix \verb|foo| is available, thus, instead of \verb|http://example.foo#john|, one can write \verb|foo:john|.
It is possible to define also an empty prefix as
\begin{verbatim}
kb_prefix("","http://example.foo#").
\end{verbatim}
or as
\begin{verbatim}
kb_prefix([],"http://example.foo#").
\end{verbatim}
In this way \verb|http://example.foo#john| can be written only as \verb|john|.

\textbf{Note:} Only one prefix per alias is allowed. Aliases defined in OWL/RDF part have the precedence, in case more than one prefix was assigned to the same alias, TRILL keeps only the first assignment.


\subsubsection{Axioms}


Axioms are modeled using the following predicates
\begin{verbatim}
subClassOf("subClass","superClass").
equivalentClasses([list,of,classes]).
disjointClasses([list,of,classes]).
disjointUnion([list,of,classes]).

subPropertyOf("subPropertyIRI","superPropertyIRI").
equivalentProperties([list,of,properties,IRI]).
propertyDomain("propertyIRI","domainIRI").
propertyRange("propertyIRI","rangeIRI").
transitiveProperty("propertyIRI").
inverseProperties("propertyIRI","inversePropertyIRI").
symmetricProperty("propertyIRI").

sameIndividual([list,of,individuals]).
differentIndividuals([list,of,individuals]).

classAssertion("classIRI","individualIRI").
propertyAssertion("propertyIRI","subjectIRI","objectIRI").
annotationAssertion("annotationIRI",axiom,literal('value')).
\end{verbatim}
For example, for asserting that \textit{employee} is subclass of \textit{worker} one can use
\begin{verbatim}
subClassOf(employee,worker).
\end{verbatim}
while the assertion \textit{worker} is equal to the intersection of \textit{person} and not \textit{unemployed}
\begin{verbatim}
equivalentClasses([worker,
           intersectionOf([person,complementOf(unemployed)])]).
\end{verbatim}


Annotation assertions can be defined, for example, as
\begin{verbatim}
annotationAssertion(foo:myAnnotation,
    subClassOf(employee,worker),'myValue').
\end{verbatim}


In particular, an axiom can be annotated with a probability which defines the degree of belief in the truth of the axiom. See Section \ref{semantics} for details.


Below, an example of an probabilistic axiom, following the Prolog syntax.
\begin{verbatim}
annotationAssertion('disponte:probability',
    subClassOf(employee,worker),literal('0.6')).
\end{verbatim}

\subsubsection{Concepts descriptions}
Complex concepts can be defined using different operators:

\noindent
Existential and universal quantifiers
\begin{verbatim}
someValuesFrom("propertyIRI","classIRI").
allValuesFrom("propertyIRI","classIRI").
\end{verbatim}
Union and intersection of concepts
\begin{verbatim}
unionOf([list,of,classes]).
intersectionOf([list,of,classes]).
\end{verbatim}
Cardinality descriptions
\begin{verbatim}
exactCardinality(cardinality,"propertyIRI").
exactCardinality(cardinality,"propertyIRI","classIRI").
maxCardinality(cardinality,"propertyIRI").
maxCardinality(cardinality,"propertyIRI","classIRI").
minCardinality(cardinality,"propertyIRI").
minCardinality(cardinality,"propertyIRI","classIRI").
\end{verbatim}
Complement of a concept
\begin{verbatim}
complementOf("classIRI").
\end{verbatim}
Nominal concept
\begin{verbatim}
oneOf([list,of,classes]).
\end{verbatim}

For example, the class \textit{workingman} is the intersection of \textit{worker} with the union of \textit{man} and \textit{woman}. It can be defined as:
\begin{verbatim}
equivalentClasses([workingman,
    intersectionOf([worker,unionOf([man,woman])])]).
\end{verbatim}

\subsection{RDF/XML syntax and tools}
\label{rdfxml-syn}
As said before, TRILL is able to automatically translate RDF/XML knowledge bases when passed as a string using 
the preticate \verb|owl_rdf/1|.

Consider the following axioms 

\begin{verbatim}
classAssertion(Cat,fluffy)
subClassOf(Cat,Pet)
propertyAssertion(hasAnimal,kevin,fluffy)
\end{verbatim}

The first axiom states that \textit{fluffy} is a \textit{Cat}. The second states that every \textit{Cat} is also a \textit{Pet}. The third states that the role \textit{hasAnimal} links together \textit{kevin} and \textit{fluffy}.

RDF (Resource Descritpion Framework) is a standard W3C. See the \href{http://www.w3.org/TR/REC-rdf-syntax/}{syntax specification} for more details.
RDF is a standard XML-based used for representing knowledge by means of triples.
A representations of the three axioms seen above is shown below.
\begin{verbatim}
<owl:NamedIndividual rdf:about="fluffy">
  <rdf:type rdf:resource="Cat"/>
</owl:NamedIndividual>

<owl:Class rdf:about="Cat">
  <rdfs:subClassOf rdf:resource="Pet"/>
</owl:Class>

<owl:ObjectProperty rdf:about="hasAnimal"/>
<owl:NamedIndividual rdf:about="kevin">
 <hasAnimal rdf:resource="fluffy"/>
</owl:NamedIndividual>
\end{verbatim}

Annotations are assertable using an extension of RDF/XML. For example the annotated axiom below, defined using the Prolog sintax
\begin{verbatim}
annotationAssertion('disponte:probability',
    subClassOf('Cat','Pet'),literal('0.6')).
\end{verbatim}
is modeled using RDF/XML syntax as
\begin{verbatim}
<owl:Class rdf:about="Cat">
 <rdfs:subClassOf rdf:resource="Pet"/>
</owl:Class>
<owl:Axiom>
 <disponte:probability rdf:datatype="&amp;xsd;decimal">
     0.6
 </disponte:probability>
 <owl:annotatedSource rdf:resource="Cat"/>
 <owl:annotatedTarget rdf:resource="Pet"/>
 <owl:annotatedProperty rdf:resource="&amp;rdfs;subClassOf"/>
</owl:Axiom>
\end{verbatim}
If you define the annotated axiom in the RDF/XML part, the annotation must be declared in the knowledge base as follow
\begin{verbatim}
<!DOCTYPE rdf:RDF [
 ...
 <!ENTITY disponte "https://sites.google.com/a/unife.it/ml/disponte#" >
]>

<rdf:RDF
 ...
 xmlns:disponte="https://sites.google.com/a/unife.it/ml/disponte#"
 ...>

 ...
 <owl:AnnotationProperty rdf:about="&amp;disponte;probability"/>
 ...
</rdf:RDF>
\end{verbatim}

There are many \href{http://www.w3.org/2001/sw/wiki/Category:Editor}{editors} for developing knowledge bases.

\section{Semantics}
\label{semantics}

Finding the explanations for a query is important for probabilistic inference. In the following we briefly describe the DISPONTE semantics~\cite{RigBelLamZes15-SW-IJ}, which requires the set of all the justifications to compute the probability of the queries.

DISPONTE \cite{RigBelLamZes15-SW-IJ,Zese17-SSW-BK} applies the distribution semantics \cite{DBLP:conf/iclp/Sato95} to Probabilistic Description Logic KBs.
In DISPONTE, a \emph{probabilistic knowledge base} $\cK$ contains a set of \emph{probabilistic axioms} which take the form
\begin{align}
& p::E\label{pax} &
\end{align}
where $p$ is a real number in $[0,1]$ and $E$ is a DL axiom. 
The probability $p$ can be interpreted as the degree of our belief in the truth of axiom $E$. 
For example, a probabilistic concept membership axiom
$
p::a:C
$
means that we have degree of belief $p$ in $C(a)$.
A probabilistic concept inclusion axiom of the form
$
p::C\sqsubseteq D
$
represents the fact that  we believe in the truth of $C \sqsubseteq D$ with probability $p$. 

For more detail about probabilistic inference with the TRILL framework, we refer the interested reader to Appendix~\ref{app:disponte} and to~\cite{ZesBelCot18-TPLP-IJ}.

The following example illustrates inference under the DISPONTE semantics.
\begin{example}
	
	\begin{align*}
	&& (E_1)\ 0.5 ::\ & \exists hasAnimal.Pet \sqsubseteq PetOwner\\
	&& 				& \fluffy: Cat \\
	&& 				& tom: Cat \\
	&& (E_2)\ 0.6 ::\ & Cat\sqsubseteq Pet\\
	&& 				& (kevin,\fluffy):hasAnimal \\
	&& 				& (kevin,tom):hasAnimal
	\end{align*}
	It indicates that the individuals that own an animal which is a pet are pet owners with a 50\% probability and that $kevin$ owns the animals $\fluffy$ and $tom$, which are cats.
	Moreover, cats are pets with a 60\% probability.

	The query axiom $Q=kevin:PetOwner$ is true with probability $P(Q)=0.5\cdot 0.6=0.3$.
\end{example}

the translation of this KB into the TRILL syntax is:
\begin{verbatim}
subClassOf(someValuesFrom(hasAnimal, pet), petOwner).
annotationAssertion(disponte:probability,
                    subClassOf(someValuesFrom(hasAnimal, pet), petOwner),
                    literal('0.5'))
classAssertion(cat, fluffy).
classAssertion(cat, tom).
subClassOf(cat, pet).
annotationAssertion(disponte:probability, subClassOf(cat, pet), literal('0.6'))
propertyAssertion(hasAnimal, kevin, fluffy).
propertyAssertion(hasAnimal, kevin, tom).
\end{verbatim}
Optionally, the KB can also contain the following axioms
\begin{verbatim}
namedIndividual(fluffy).
namedIndividual(kevin).
namedIndividual(tom).
objectProperty(hasAnimal).
annotationProperty('http://ml.unife.it/disponte#probability').
class(petOwner).
class(pet).
\end{verbatim}



\section{Inference}
\label{inf}

Traditionally, a reasoning algorithm decides  whether an axiom is entailed or not by a KB by refutation: the  axiom $E$ is entailed if $\neg E$ has no model
in the KB.
Besides deciding whether an axiom is entailed by a KB, we want to find also explanations for the axiom.

The problem of finding  explanations for a query
has been investigated by various authors \cite{DBLP:conf/ijcai/SchlobachC03,DBLP:journals/ws/KalyanpurPSH05,DBLP:conf/semweb/KalyanpurPHS07,Kalyanpurphd,extended_tracing}.
 It was called  \emph{axiom pinpointing} in 
\cite{DBLP:conf/ijcai/SchlobachC03}  and considered as a non-standard reasoning service useful for tracing derivations and debugging ontologies. 
In particular, in \cite{DBLP:conf/ijcai/SchlobachC03} the authors define \emph{minimal axiom sets}  (\emph{MinAs} for short).
\begin{definition}[MinA]
 Let $\cK$ be a knowledge base and $Q$ an
axiom that follows from it, i.e., 
$\cK \models Q$. We call a set 
$M\subseteq \cK$ a
\emph{minimal axiom set} or \emph{MinA} for $Q$ in $\cK$ if 
$M \models Q$ and it is minimal
w.r.t. set inclusion.
\end{definition}  
\noindent The problem of enumerating all MinAs is called \textsc{min-a-enum}.
\textsc{All-MinAs($Q,\cK$)} is the set of all MinAs for query $Q$ in knowledge base $\cK$.

A \emph{tableau} is a graph where each node represents an
individual $a$ and is labeled with the set of concepts $\cL(a)$ it belongs to. Each
edge $\langle a, b\rangle$ in the graph is labeled with the set of roles to which the couple
$(a, b)$ belongs. Then, a set of  consistency preserving tableau
expansion rules are repeatedly applied until a clash (i.e., a contradiction) is detected or a clash-free
graph is found to which no more rules are applicable. A clash is for example a
couple $(C, a)$ where $C$ and $\neg C$ are present in the label of a node, i.e. ${C, \neg C} \subseteq \cL(a)$.

Some expansion rules are non-deterministic, i.e., they generate
a finite set of tableaux. Thus the algorithm keeps a set of tableaux that is
consistent if there is any tableau in it that is consistent, i.e., that is clash-free.
Each time a clash is detected in a tableau $G$, the algorithm stops applying rules
to $G$. Once every tableau in $T$ contains a clash or no more expansion rules
can be applied to it, the algorithm terminates. If all the tableaux in the final
set $T$ contain a clash, the algorithm returns unsatisfiable as no model can be
found. Otherwise, any one clash-free completion graph in $T$ represents a possible
model for the concept and the algorithm returns satisfiable.

\textsc{min-a-enum} is required to answer queries to KBs following the DISPONTE semantics. To
compute the probability of a query, the explanations must be made mutually exclusive, so
that the probability of each individual explanation is computed and summed
with the others. To do that we assign independent Boolean random variables to the axioms contained in the explanations and defining 
the Disjunctive Normal Form (DNF) Boolean formula $f_K$ which models the set of explanations. Thus
$
f_K(\mathbf{X})=\bigvee_{\kappa\in K}\bigwedge_{(E_i,1)}X_{i}\bigwedge_{(E_i,0)}\overline{X_{i}}
$
where $\mathbf{X}=\{X_{i}|(E_i,k)\in\kappa,\kappa\in K\}$ is the set of Boolean random variables.
We can now translate $f_K$ to a Binary Decision Diagram (BDD), from which we can compute the probability of the query with a dynamic programming algorithm that is linear in the size of the BDD.

\subsection{Possible Queries}
\label{queries}

TRILL can compute the probability or find an explanation of the following queries:
\begin{itemize}
  \item Concept membership queries.
  \item Property assertion queries.
  \item Subsumption queries.
  \item Unsatifiability of a concept.
  \item Inconsistency of the knowledge base.
\end{itemize}
All the input arguments must be atoms or ground terms.
Note that it is necessary to specify which algorithm, TRILL, TRILL$^P$ or TORNADO, has to be loaded for performing inference. This is done by using at the beginning of the input file the directive
\begin{verbatim}
:- trill.
\end{verbatim}
for loading TRILL,
\begin{verbatim}
:- trillp.
\end{verbatim}
for TRILL$^P$ or
\begin{verbatim}
:- tornado.
\end{verbatim}
for TORNADO.

\subsubsection{Probabilistic Queries}
TRILL can be queried for computing the probability of queries. A resulting 0 probaility means that the query is false w.r.t. the knowledge base, while a probability value 1 that the query is certainly true.

The probability of an individual to belong to a concept can be asked using TRILL with the predicate
\begin{verbatim}
prob_instanceOf(+Concept:term,+Individual:atom,-Prob:double)
\end{verbatim}
as in (\href{http://trill.lamping.unife.it/example/trill/peoplePets.pl}{\texttt{peoplePets.pl}})
\begin{verbatim}
?- prob_instanceOf(cat,'Tom',Prob).
\end{verbatim}

The probability of two individuals to be related by a role can be computed with
\begin{verbatim}
prob_property_value(+Prop:atom,+Individual1:atom,
                    +Individual2:atom,-Prob:double)
\end{verbatim}
as in (\href{http://trill.lamping.unife.it/example/trill/peoplePets.pl}{\texttt{peoplePets.pl}})
\begin{verbatim}
?- prob_property_value(has_animal,'Kevin','Tom',Prob).
\end{verbatim}

If you want to know the probability with which a class is a subclass of another you have to use
\begin{verbatim}
prob_sub_class(+Concept:term,+SupConcept:term,-Prob:double)
\end{verbatim}
as in (\href{http://trill.lamping.unife.it/example/trill/peoplePets.pl}{\texttt{peoplePets.pl}})
\begin{verbatim}
?- prob_sub_class(cat,pet,Prob).
\end{verbatim}

The probability of the unsatisfiability of a concept can be asked with the predicate
\begin{verbatim}
prob_unsat(+Concept:term,-Prob:double)
\end{verbatim}
as in (\href{http://trill.lamping.unife.it/example/trill/peoplePets.pl}{\texttt{peoplePets.pl}})
\begin{verbatim}
?- prob_unsat(intersectionOf([cat,complementOf(pet)]),P).
\end{verbatim}
This query for example corresponds with a subsumption query, which is represented as the intersection of the subclass and the complement of the superclass.

Finally, you can ask the probability of the inconsistency of the knowledge base with
\begin{verbatim}
prob_inconsistent_theory(-Prob:double)
\end{verbatim}


\subsubsection{Non Probabilistic Queries}
In TRILL you can also ask whether a query is true or false w.r.t. the knowledge base and in case of a succesful query an explanation can be returned as well. 
Query predicates in this case differs in the number of arguments, in the second case, when we want also an explanation, an extra argument is added to unify with the list of axioms
build to explain the query.

The query if an individual belongs to a concept can be used the predicates
\begin{verbatim}
instanceOf(+Concept:term,+Individual:atom)
instanceOf(+Concept:term,+Individual:atom,-Expl:list)
\end{verbatim}
as in (\href{http://trill.lamping.unife.it/example/trill/peoplePets.pl}{\texttt{peoplePets.pl}})
\begin{verbatim}
?- instanceOf(pet,'Tom').
?- instanceOf(pet,'Tom',Expl).
\end{verbatim}
In the first query the result is \verb|true| because Tom belongs to cat, in the second case TRILL returns the explanation 
\begin{verbatim}
[classAssertion(cat,'Tom'), subClassOf(cat,pet)]
\end{verbatim}


Similarly, to ask whether two individuals are related by a role you have to use the queries
\begin{verbatim}
property_value(+Prop:atom,+Individual1:atom,+Individual2:atom)
property_value(+Prop:atom,+Individual1:atom,
               +Individual2:atom,-Expl:list)
\end{verbatim}
as in (\href{http://trill.lamping.unife.it/example/trill/peoplePets.pl}{\texttt{peoplePets.pl}})
\begin{verbatim}
?- property_value(has_animal,'Kevin','Tom').
?- property_value(has_animal,'Kevin','Tom',Expl).
\end{verbatim}

If you want to know if a class is a subclass of another you have to use
\begin{verbatim}
sub_class(+Concept:term,+SupConcept:term)
sub_class(+Concept:term,+SupConcept:term,-Expl:list)
\end{verbatim}
as in (\href{http://trill.lamping.unife.it/example/trill/peoplePets.pl}{\texttt{peoplePets.pl}})
\begin{verbatim}
?- sub_class(cat,pet).
?- sub_class(cat,pet,Expl).
\end{verbatim}

The unsatisfiability of a concept can be asked with the predicate
\begin{verbatim}
unsat(+Concept:term)
unsat(+Concept:term,-Expl:list)
\end{verbatim}
as in (\href{http://trill.lamping.unife.it/example/trill/peoplePets.pl}{\texttt{peoplePets.pl}})
\begin{verbatim}
?- unsat(intersectionOf([cat,complementOf(pet)])).
?- unsat(intersectionOf([cat,complementOf(pet)]),Expl).
\end{verbatim}
In this case, the returned explanation is the same obtained by querying if cat is subclass of pet with the \verb|sub_class/3| predicate, i.e., \verb|[subClassOf(cat,pet)]|

Finally, you can ask about the inconsistency of the knowledge base with
\begin{verbatim}
inconsistent_theory
inconsistent_theory(-Expl:list)
\end{verbatim}

The predicate above returns explanations one at a time. To collect all the explanations with a single goal you can use the predicates:
\begin{verbatim}
all_instanceOf(+Concept:term,+Individual:atom,-Expl:list)
all_property_value(+Prop:atom,+Individual1:atom,
                        +Individual2:atom,-Expl:list)
all_sub_class(+Concept:term,+SupConcept:term,-Expl:list)
all_unsat(+Concept:term,-Expl:list)
all_inconsistent_theory(-Expl:list)
\end{verbatim}

\subsection{Query Options}
The behaviour of the queries can be fine tuned using the \emph{query options}. To use them you need to use the predicates:
\begin{verbatim}
instanceOf(+Concept:term,+Individual:atom,-Expl:list,-QueryOptions:list)
property_value(+Prop:atom,+Individual1:atom,
                        +Individual2:atom,-Expl:list,-QueryOptions:list)
sub_class(+Concept:term,+SupConcept:term,-Expl:list,-QueryOptions:list)
unsat(+Concept:term,-Expl:list,-QueryOptions:list)
inconsistent_theory(-Expl:list,-QueryOptions:list)
\end{verbatim}

Options can be:
\begin{itemize}
	\item \verb|assert_abox(Boolean)| if Boolean is set to true the list of ABoxes is asserted with the predicate \verb|final_abox/1|;
	\item \verb|return_prob(Prob)| if present the probability of the query is computed and unified with \verb|Prob|;
%	\item \verb|return_single_prob(Boolean)| must be used with \verb|return_prob(Prob)|. It forces TRILL to return the probability of each single explanation;
	\item \verb|max_expl(Value)| to limit the maximum number of explanations to find. \verb|Value| must be an integer. The predicate will return a list containing at most \verb|Value| different explanations;
	\item \verb|time_limit(Value)| to limit the time for the inference. The predicate will return the explanations found in the time allowed. \verb|Value| is the number of seconds allowed for the search of explanations .
\end{itemize}

For example, if you want to find the probability of the query $Q=kevin:PetOwner$ computed on at most 2 explanations allowing at most 1 second for the explanations search you can use the goal
\begin{verbatim}
instanceOf('natureLover','Kevin',Expl,
           [time_limit(1),return_prob(Prob),max_expl(2)]).
\end{verbatim}

\subsection{TRILL Useful Predicates}
There are other predicates defined in TRILL which helps manage and load the KB.
\begin{verbatim}
add_kb_prefix(++ShortPref:string,++LongPref:string)
add_kb_prefixes(++Prefixes:list)
\end{verbatim}
They register the alias for prefixes. The firs registers \verb|ShortPref| for the prefix \verb|LongPref|, while the second register all the alias prefixes contained in Prefixes. The input list must contain pairs alias=prefix, i.e., \verb|[('foo'='http://example.foo#')]|. In both cases, the empty string \verb|''| can be defined as alias. The predicates
\begin{verbatim}
remove_kb_prefix(++ShortPref:string,++LongPref:string)
remove_kb_prefix(++Name:string)
\end{verbatim}
remove from the registered aliases the one given in input. In particular, \verb|remove_kb_prefix/1| takes as input a string that can be an alias or a prefix and removes the pair containing the string from the registered aliases.

\begin{verbatim}
add_axiom(++Axiom:axiom)
add_axioms(++Axioms:list)
\end{verbatim}
These predicates add (all) the given axiom to the knowledge base. While, to remove axioms can be similarly used the predicates
\begin{verbatim}
remove_axiom(++Axiom:axiom)
remove_axioms(++Axioms:list)
\end{verbatim}
All the axioms must be defined following the TRILL syntax.

Finally, we can interrogate TRILL to return the loaded axioms with
\begin{verbatim}
axiom(?Axiom:axiom)
\end{verbatim}
This predicate searches in the loaded knowledge base axioms that unify with Axiom.

\section{Using TRILL on your Machine}
To use TRILL without the need of an Internet connection, you must install it on your machine. Please, read the \href{https://github.com/rzese/trill/blob/master/doc/manual.pdf}{TRILL manual} for detailed information.

\section{Download Query Results through an API}
The results of queries can also be downloaded programmatically by directly
approaching the Pengine API. Example client code is 
\href{https://github.com/friguzzi/trill-on-swish/tree/master/client}{available}.  For example, the \verb|swish-ask.sh| client
can be used with bash to download the results for a query in CSV.  The call
below downloads a CSV file for the coin example.
\begin{verbatim}
$ bash swish-ask.sh --server=http://trill-sw.eu \
  example/trill/peoplePets.pl \
  Prob "prob_instanceOf('natureLover','Kevin',Prob)"
\end{verbatim}
The script can ask queries against Prolog scripts stored in 
\url{http://trill-sw.eu} by specifying
the script on the command line.  User defined files stored
in TRILL on SWISH (locations of type
\url{http://trill-sw.eu/p/johnEmployee_user.pl}) can
be directly used, for example:
\begin{verbatim}
$ bash swish-ask.sh --server=http://trill-sw.eu \
  johnEmployee_user.pl Expl "instanceOf(person,john,Expl)"
\end{verbatim}
Example programs can be used by specifying the folder portion of the url of the example,
as in the first johnEmployee example above where the url for the program is 
\url{http://trill-sw.eu/example/trill/johnEmployee.pl}.

You can also use an url for the program as in 
\begin{verbatim}
$ bash swish-ask.sh --server=http://trill-sw.eu \
  https://raw.githubusercontent.com/friguzzi/trill-on-swish/
  master/examples/trill/peoplePets.pl \
  Prob "prob_instanceOf('natureLover','Kevin',Prob)"
\end{verbatim}
Results can be downloaded in JSON using the option \verb|--json-s| or
\verb|--json-html|.
With the first the output is in a simple string format where Prolog terms are sent using quoted write, the latter serialize responses as HTML strings. E.g.
\begin{verbatim}
$ bash swish-ask.sh --json-s --server=http://trill-sw.eu \
  johnEmployee_user.pl Expl "instanceOf(person,john,Expl)"
\end{verbatim}
The JSON format can also be modified. See
\url{http://www.swi-prolog.org/pldoc/doc_for?object=pengines%3Aevent_to_json/4}.

Prolog can exploit the Pengine API directly.  For example, the above can
be called as:
\begin{verbatim}
?- [library(pengines)].
?- pengine_rpc('http://trill-sw.eu',
     prob_instanceOf('natureLover','Kevin',Prob),
     [ src_url('https://raw.githubusercontent.com/friguzzi/trill-on-swish/\
  master/example/trill/peoplePets.pl'),
     application(swish)
     ]).
Prob = 0.51.
?-
\end{verbatim}

\section{Manual in PDF}
A PDF version of this help in PDF is available at
\url{https://github.com/rzese/trill/blob/master/doc/help-trill.pdf}.
A manual for the standalone version of TRILL is available at \url{https://github.com/rzese/trill/blob/master/doc/manual.pdf}


\bibliographystyle{plain}
\bibliography{bib}

\appendix
 
\section{Description Logics}
\label{app:dl}
In this section, we recall the expressive description logic $\mathcal{ALC}$ \cite{DBLP:journals/ai/Schmidt-SchaussS91}. We refer to 
\cite{DBLP:journals/ws/LukasiewiczS08} for a detailed description of $\mathcal{SHOIN}(\mathbf{D})$ DL, that is at the basis of OWL DL.

%%ALC
Let $\mathbf{A}$, $\mathbf{R}$ and $\mathbf{I}$ be sets of \emph{atomic concepts}, \emph{roles} and \emph{individuals}.
A \emph{role} is an atomic role $R \in \mathbf{R}$. 
\emph{Concepts} are defined by induction as follows. Each $C \in \mathbf{A}$, $\bot$ and $\top$
are concepts.
If $C$, $C_1$ and $C_2$ are concepts and $R \in \mathbf{R}$, then $(C_1\sqcap C_2)$, $(C_1\sqcup C_2 )$, $\neg C$,
$\exists R.C$, and $\forall R.C$ are concepts. 
Let $C$, $D$ be concepts,  $R \in \mathbf{R}$ and $a, b \in \mathbf{I}$. 
An \emph{ABox} $\cA$ is a finite set of \textit{concept membership axioms} $a : C$ and \textit{role membership
	axioms} $(a, b) : R$, while 
a \emph{TBox} $\cT$ is a finite set of \textit{concept inclusion axioms} $C\sqsubseteq D$. $C \equiv D$ abbreviates $C \sqsubseteq D$ and $D\sqsubseteq  C$.

A \emph{knowledge base} $\cK = (\cT, \cA)$ consists of a TBox $\cT$ and an ABox $\cA$.
A KB $\cK$ is assigned a semantics in terms of set-theoretic interpretations $\cI = (\Delta^\cI , \cdot^\cI )$, where $\Delta^\cI$ is a non-empty \textit{domain} and $\cdot^\cI$ is the \textit{interpretation function} that assigns  an element in $\Delta ^\cI$ to each $a \in \mathbf{I}$, a subset of $\Delta^\cI$ to each $C \in \mathbf{A}$ and a subset of $\Delta^\cI \times \Delta^\cI$ to each $R \in \mathbf{R}$.

\section{DISPONTE}
\label{app:disponte}
In the field of Probabilistic Logic Programming (PLP for short) many proposals have been presented. 
An effective and popular approach is the Distribution Semantics \cite{DBLP:conf/iclp/Sato95}, which underlies many PLP languages such as
PRISM~\cite{DBLP:conf/iclp/Sato95,DBLP:journals/jair/SatoK01},
Independent Choice Logic  \cite{Poo97-ArtInt-IJ}, Logic Programs with Annotated Disjunctions \cite{VenVer04-ICLP04-IC} and ProbLog \cite{DBLP:conf/ijcai/RaedtKT07}.
Along this line, many reserchers proposed to combine probability theory with Description Logics (DLs for short) \cite{DBLP:journals/ws/LukasiewiczS08,DBLP:conf/rweb/Straccia08}.
DLs are at the basis of the Web Ontology Language (OWL for short), a family of knowledge representation formalisms used for modeling information
of the Semantic Web

TRILL follows the DISPONTE \cite{RigBelLamZese12-URSW12,Zese17-SSW-BK} semantics to compute the probability of queries.
DISPONTE applies the distribution semantics \cite{DBLP:conf/iclp/Sato95} of probabilistic logic programming to DLs. 
A program following this semantics defines a probability distribution over normal logic programs
called \emph{worlds}. Then the distribution is extended to queries and the probability of a query is obtained by marginalizing the joint distribution of the query and the programs.

In DISPONTE, a \emph{probabilistic knowledge base} $\cK$ is a set of \emph{certain axioms} or \emph{probabilistic axioms} in which each axiom is independent evidence.
Certain axioms take the form of regular DL axioms while probabilistic axioms are
$p::E$
where $p$ is a real number in $[0,1]$ and $E$ is a DL axiom. 

The idea of DISPONTE is to associate independent Boolean random variables to the probabilistic axioms. 
To obtain a \emph{world}, we include every formula obtained from a certain axiom. 
For each probabilistic axiom, we decide whether to include it or not in $w$.
A world therefore is a non probabilistic KB that can be assigned a semantics in the usual way.
A query is entailed by a world if it is true in every  model of the world.

The probability $p$ can be interpreted as an \emph{epistemic probability}, i.e., as the degree of our belief in axiom $E$. 
For example, a probabilistic concept membership axiom
$
p::a:C
$
means that we have degree of belief $p$ in $C(a)$.
A probabilistic concept inclusion axiom of the form
$
p::C\sqsubseteq D
$
represents our belief in the truth of $C \sqsubseteq D$ with probability $p$. 

Formally, an \emph{atomic choice} is a couple $(E_i,k)$ where $E_i$ is the $i$th probabilistic axiom  and $k\in \{0,1\}$. 
$k$ indicates whether $E_i$ is chosen to be included in a world ($k$ = 1) or not ($k$ = 0). 
A \emph{composite choice} $\kappa$ is a consistent set of atomic choices, i.e.,  $(E_i,k)\in\kappa, (E_i,m)\in \kappa$ implies $k=m$ (only one decision is taken for each formula). 
The probability of a composite choice $\kappa$  is 
$P(\kappa)=\prod_{(E_i,1)\in \kappa}p_i\prod_{(E_i, 0)\in \kappa} (1-p_i)$, where $p_i$ is the probability associated with axiom $E_i$.
A \emph{selection} $\sigma$ is a total composite choice, i.e., it contains an atomic choice $(E_i,k)$ for every 
probabilistic axiom  of the probabilistic KB. 
A selection $\sigma$ identifies a theory $w_\sigma$ called  a \emph{world} in this way:
$w_\sigma=\cC\cup\{E_i|(E_i,1)\in \sigma\}$ where $\cC$ is the set of certain axioms. Let us indicate with $\mathcal{S}_\cK$ the set of all selections and with $\mathcal{W}_\cK$ the set of all worlds.
The probability of a world $w_\sigma$  is 
$P(w_\sigma)=P(\sigma)=\prod_{(E_i,1)\in \sigma}p_i\prod_{(E_i, 0)\in \sigma} (1-p_i)$.
$P(w_\sigma)$ is a probability distribution over worlds, i.e., $\sum_{w\in \mathcal{W}_\cK}P(w)=1$.

We can now assign probabilities to queries. 
Given a world $w$, the probability of a query $Q$ is defined as $P(Q|w)=1$ if $w\models Q$ and 0 otherwise.
The probability of a query can be defined by marginalizing the joint probability of the query and the worlds, i.e.
$P(Q)=\sum_{w\in \mathcal{W}_\cK}P(Q,w)=\sum_{w\in \mathcal{W}_\cK} P(Q|w)p(w)=\sum_{w\in \mathcal{W}_\cK: w\models Q}P(w)$.

\begin{example}
	\label{people+petsxy}
	\begin{small}
		Consider the following KB, inspired by the \texttt{people+pets} ontology  \cite{ISWC03-tut}:
		{\center $0.5\ \ ::\ \ \exists hasAnimal.Pet \sqsubseteq NatureLover\ \ \ \ \ 0.6\ \ ::\ \ Cat\sqsubseteq Pet$\\
			$(kevin,tom):hasAnimal\ \ \ \ \ (kevin,\fluffy):hasAnimal\ \ \ \ \ tom: Cat\ \ \ \ \ \fluffy: Cat$\\}
		\noindent The KB indicates that the individuals that own an animal which is a pet are nature lovers with a 50\% probability and that $kevin$ has the animals 
		$\fluffy$ and $tom$.  Fluffy and $tom$ are cats and cats are pets with probability 60\%.
		We associate a Boolean variable to each axiom as follow
		%\begin{small}
		$F_1 = \exists hasAnimal.Pet \sqsubseteq NatureLover$, $F_2=(kevin,\fluffy):hasAnimal$, $F_3=(kevin,tom):hasAnimal$, $F_4=\fluffy: Cat$, $F_5=tom: Cat$ and $F_6= Cat\sqsubseteq Pet$.
		%\end{small}.
		
		
		The KB has four worlds and the query axiom $Q=kevin:NatureLover$ is true in one of them, the one corresponding to the selection 
		$
		\{(F_1,1),(F_2,1)\}
		$.
		The probability of the query is $P(Q)=0.5\cdot 0.6=0.3$.
	\end{small}
\end{example}

\begin{example}
	\label{people+pets_comb}
	\begin{small}
		Sometimes we have to combine knowledge from multiple, untrusted sources, each one with a different reliability. 
		Consider a KB similar to the one of Example \ref{people+petsxy} but where we have a single cat, $\fluffy$.
		{\center $\exists hasAnimal.Pet \sqsubseteq NatureLover\ \ \ \ \ (kevin,\fluffy):hasAnimal\ \ \ \ \ Cat\sqsubseteq Pet$\\}
		
		\noindent and there are two sources of information with different reliability that provide the information that $\fluffy$ is a cat. 
		On one source the user has a degree of belief of 0.4, i.e., he thinks it is correct with a 40\% probability,  
		while on the other source he has a degree of belief 0.3. %, i.e. he thinks it is correct with a 30\% probability. 
		The user can reason on this knowledge by adding the following statements to his KB:
		{\center$0.4\ \ ::\ \ \fluffy: Cat\ \ \ \ \ 0.3\ \ ::\ \ \fluffy: Cat$\\}
		The two statements represent independent evidence on $\fluffy$ being a cat. We associate $F_1$ ($F_2$) to the first (second) probabilistic axiom.
		
		The query axiom $Q=kevin:NatureLover$ is true in 3 out of the 4 worlds, those corresponding to the selections 
		$
		\{ \{(F_1,1),(F_2,1)\},
		\{(F_1,1),(F_2,0)\},
		\{(F_1,0),(F_2,1)\}\}
		$. 
		So 
		$P(Q)=0.4\cdot 0.3+0.4\cdot 0.7+ 0.6\cdot 0.3=0.58.$
		This is reasonable if the two sources can be considered as independent. In fact,  the probability comes from the  disjunction of two
		independent Boolean random variables with probabilities respectively 0.4 and 0.3: 
		$
		P(Q) = P(X_1\vee X_2) = P(X_1)+P(X_2)-P(X_1\wedge X_2)
		= P(X_1)+P(X_2)-P(X_1)P(X_2)
		= 0.4+0.3-0.4\cdot 0.3=0.58
		$
	\end{small}
\end{example}

\section{Inference}
\label{app:inf}
Traditionally, a reasoning algorithm decides  whether an axiom is entailed or not by a KB by refutation: the  axiom $E$ is entailed if $\neg E$ has no model
in the KB.
Besides deciding whether an axiom is entailed by a KB, we want to find also explanations for the axiom, in order to compute the probability of the axiom.

\subsection{Computing Queries Probability}
The problem of finding  explanations for a query
has been investigated by various authors \cite{DBLP:conf/ijcai/SchlobachC03,DBLP:journals/ws/KalyanpurPSH05,DBLP:conf/semweb/KalyanpurPHS07,Kalyanpurphd,extended_tracing,Zese17-SSW-BK}.
It was called  \emph{axiom pinpointing} in 
\cite{DBLP:conf/ijcai/SchlobachC03}  and considered as a non-standard reasoning service useful for tracing derivations and debugging ontologies. 
In particular, in \cite{DBLP:conf/ijcai/SchlobachC03} the authors define \emph{minimal axiom sets}  (\emph{MinAs} for short).
\begin{definition}[MinA]
	Let $\cK$ be a knowledge base and $Q$ an
	axiom that follows from it, i.e., 
	$\cK \models Q$. We call a set 
	$M\subseteq \cK$ a
	\emph{minimal axiom set} or \emph{MinA} for $Q$ in $\cK$ if 
	$M \models Q$ and it is minimal
	w.r.t. set inclusion.
\end{definition}  
\noindent The problem of enumerating all MinAs is called \textsc{min-a-enum}.
\textsc{All-MinAs($Q,\cK$)} is the set of all MinAs for query $Q$ in knowledge base $\cK$.

A \emph{tableau} is a graph where each node represents an
individual $a$ and is labeled with the set of concepts $\cL(a)$ it belongs to. Each
edge $\langle a, b\rangle$ in the graph is labeled with the set of roles to which the couple
$(a, b)$ belongs. Then, a set of  consistency preserving tableau
expansion rules are repeatedly applied until a clash (i.e., a contradiction) is detected or a clash-free
graph is found to which no more rules are applicable. A clash is for example a
couple $(C, a)$ where $C$ and $\neg C$ are present in the label of a node, i.e. ${C, \neg C} \subseteq \cL(a)$.

Some expansion rules are non-deterministic, i.e., they generate
a finite set of tableaux. Thus the algorithm keeps a set of tableaux that is
consistent if there is any tableau in it that is consistent, i.e., that is clash-free.
Each time a clash is detected in a tableau $G$, the algorithm stops applying rules
to $G$. Once every tableau in $T$ contains a clash or no more expansion rules
can be applied to it, the algorithm terminates. If all the tableaux in the final
set $T$ contain a clash, the algorithm returns unsatisfiable as no model can be
found. Otherwise, any one clash-free completion graph in $T$ represents a possible
model for the concept and the algorithm returns satisfiable.

To compute the probability of a query, the explanations must be made mutually exclusive, so
that the probability of each individual explanation is computed and summed
with the others. To do that we assign independent Boolean random variables to the axioms contained in the explanations and defining 
the Disjunctive Normal Form (DNF) Boolean formula $f_K$ which models the set of explanations. Thus
$
f_K(\mathbf{X})=\bigvee_{\kappa\in K}\bigwedge_{(E_i,1)}X_{i}\bigwedge_{(E_i,0)}\overline{X_{i}}
$
where $\mathbf{X}=\{X_{i}|(E_i,k)\in\kappa,\kappa\in K\}$ is the set of Boolean random variables.
We can now translate $f_K$ to a Binary Decision Diagram (BDD), from which we can compute the probability of the query with a dynamic programming algorithm that is linear in the size of the BDD.


In \cite{DBLP:journals/jar/BaaderP10,DBLP:journals/logcom/BaaderP10} the authors consider the problem of finding a \emph{pinpointing formula} instead of 
\textsc{All-MinAs($Q,\cK$)}.
The pinpointing formula is a monotone Boolean formula in which each Boolean variable corresponds to an axiom of the KB. This formula is built using 
the variables and the conjunction and disjunction connectives. It compactly encodes the set of all MinAs.
Let's assume that each axiom $E$ of a KB $\cK$ is associated with a propositional variable, indicated with $var(E)$. The set of all propositional variables is indicated with $var(\cK )$.
A valuation $\nu$ of a monotone Boolean formula is the set of propositional variables that are true. For a valuation $\nu \subseteq var(\cK)$, let $\cK_{\nu} := \{t \in \cK |var(t)\in\nu\}$.
\begin{definition}[Pinpointing formula]
	Given a query $Q$ and a KB $\cK$, a monotone Boolean formula $\phi$ over $var(\cK)$ is called a \emph{pinpointing formula} for $Q$ if for every valuation $\nu \subseteq var(\cK)$ it holds that $\cK_{\nu} \models Q$ iff
	$\nu$ satisfies $\phi$.
	
\end{definition}
In Lemma 2.4 of \cite{DBLP:journals/logcom/BaaderP10} the authors proved that the set of all MinAs can be obtained by transforming the pinpointing formula into a Disjunctive Normal Form Boolean formula~(DNF) and removing disjuncts implying other disjuncts. 

Irrespective of which representation of the explanations we choose, a DNF or a general pinpointing formula, we can apply knowledge compilation and \textit{transform it into a Binary Decision Diagram (BDD)}, 
from which we can compute the probability of the query with a dynamic programming algorithm that is linear in the size of the BDD.

We refer to \cite{Zese17-SSW-BK,ZesBelRig16-AMAI-IJ} for a detailed description of the two methods.
\end{document}