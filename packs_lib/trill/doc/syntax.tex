\section{Syntax}
\label{syn}

Description Logics (DLs) are knowledge representation formalisms that are at the basis of the Semantic Web \cite{DBLP:conf/dlog/2003handbook,dlchap} and are used for modelling ontologies.
They are represented using a syntax based on concepts, basically sets of individuals of the domain, and roles, sets of pairs of individuals
of the domain. A more formal description can be found in the Appendix \ref{app:dl}.

TRILL allows the use of two different syntaxes used together or individually:
\begin{itemize}
 \item RDF/XML
 \item Prolog syntax 
\end{itemize}

RDF/XML syntax can be used by exploiting the predicate \verb|owl_rdf/1|. For example:
\begin{small}
\begin{verbatim}
owl_rdf('
<?xml version="1.0"?>

<!DOCTYPE rdf:RDF [
    <!ENTITY owl "http://www.w3.org/2002/07/owl#" >
    <!ENTITY xsd "http://www.w3.org/2001/XMLSchema#" >
    <!ENTITY rdfs "http://www.w3.org/2000/01/rdf-schema#" >
    <!ENTITY rdf "http://www.w3.org/1999/02/22-rdf-syntax-ns#" >
]>
<rdf:RDF xmlns="http://here.the.IRI.of.your.ontology#"
     xmlns:rdf="http://www.w3.org/1999/02/22-rdf-syntax-ns#"
     xmlns:owl="http://www.w3.org/2002/07/owl#"
     xmlns:xsd="http://www.w3.org/2001/XMLSchema#"
     xmlns:rdfs="http://www.w3.org/2000/01/rdf-schema#">
    <owl:Ontology rdf:about="http://here.the.IRI.of.your.ontology"/>

    <!-- 
    Axioms
    -->

</rdf:RDF>
').
\end{verbatim}
\end{small}

For a brief introduction on RDF/XML syntax see \textit{RDF/XML syntax and tools} section below (Sec. \ref{rdfxml-syn}).

Note that each single \verb|owl_rdf/1| must be self contained and well formatted, it must start and end with \verb|rdf:RDF| tag and contain all necessary declarations (namespaces, entities, ...).


An example of the combination of both syntaxes is shown the example \href{http://trill-sw.eu/example/trill/johnEmployee.pl}{\texttt{johnEmployee.pl}}. It models that \textit{john} is an \textit{employee} and that employees are \textit{workers}, which are in turn people (modeled by the concept \textit{person}).
\begin{small}
\begin{verbatim}
owl_rdf('<?xml version="1.0"?>
<rdf:RDF xmlns="http://example.foo#"
     xml:base="http://example.foo"
     xmlns:rdf="http://www.w3.org/1999/02/22-rdf-syntax-ns#"
     xmlns:owl="http://www.w3.org/2002/07/owl#"
     xmlns:xml="http://www.w3.org/XML/1998/namespace"
     xmlns:xsd="http://www.w3.org/2001/XMLSchema#"
     xmlns:rdfs="http://www.w3.org/2000/01/rdf-schema#">
    <owl:Ontology rdf:about="http://example.foo"/>

    <!-- Classes -->
    <owl:Class rdf:about="http://example.foo#worker">
        <rdfs:subClassOf rdf:resource="http://example.foo#person"/>
    </owl:Class>

</rdf:RDF>').

subClassOf('employee','worker').

owl_rdf('<?xml version="1.0"?>
<rdf:RDF xmlns="http://example.foo#"
     xml:base="http://example.foo"
     xmlns:rdf="http://www.w3.org/1999/02/22-rdf-syntax-ns#"
     xmlns:owl="http://www.w3.org/2002/07/owl#"
     xmlns:xml="http://www.w3.org/XML/1998/namespace"
     xmlns:xsd="http://www.w3.org/2001/XMLSchema#"
     xmlns:rdfs="http://www.w3.org/2000/01/rdf-schema#">
    <owl:Ontology rdf:about="http://example.foo"/>
    
    <!-- Individuals -->
    <owl:NamedIndividual rdf:about="http://example.foo#john">
        <rdf:type rdf:resource="http://example.foo#employee"/>
    </owl:NamedIndividual>
</rdf:RDF>').
\end{verbatim}
\end{small}

\subsection{Prolog Syntax}
\label{trill-syn}
\subsubsection{Declarations}


Prolog syntax allows, as in standard OWL, the declaration of classes, properties, etc.
\begin{verbatim}
class("classIRI").
datatype("datatypeIRI").
objectProperty("objectPropertyIRI").
dataProperty("dataPropertyIRI").
annotationProperty("annotationPropertyIRI").
namedIndividual("individualIRI").
\end{verbatim}
However, TRILL properly works also in their absence.

Prolog syntax allows also the declaration of aliases for namespaces by using the \verb|kb_prefix/2| predicate.
\begin{verbatim}
kb_prefix("foo","http://example.foo#").
\end{verbatim}
After this declaration, the prefix \verb|foo| is available, thus, instead of \verb|http://example.foo#john|, one can write \verb|foo:john|.
It is possible to define also an empty prefix as
\begin{verbatim}
kb_prefix("","http://example.foo#").
\end{verbatim}
or as
\begin{verbatim}
kb_prefix([],"http://example.foo#").
\end{verbatim}
In this way \verb|http://example.foo#john| can be written only as \verb|john|.

\textbf{Note:} Only one prefix per alias is allowed. Aliases defined in OWL/RDF part have the precedence, in case more than one prefix was assigned to the same alias, TRILL keeps only the first assignment.


\subsubsection{Axioms}


Axioms are modeled using the following predicates
\begin{verbatim}
subClassOf("subClass","superClass").
equivalentClasses([list,of,classes]).
disjointClasses([list,of,classes]).
disjointUnion([list,of,classes]).

subPropertyOf("subPropertyIRI","superPropertyIRI").
equivalentProperties([list,of,properties,IRI]).
propertyDomain("propertyIRI","domainIRI").
propertyRange("propertyIRI","rangeIRI").
transitiveProperty("propertyIRI").
inverseProperties("propertyIRI","inversePropertyIRI").
symmetricProperty("propertyIRI").

sameIndividual([list,of,individuals]).
differentIndividuals([list,of,individuals]).

classAssertion("classIRI","individualIRI").
propertyAssertion("propertyIRI","subjectIRI","objectIRI").
annotationAssertion("annotationIRI",axiom,literal('value')).
\end{verbatim}
For example, for asserting that \textit{employee} is subclass of \textit{worker} one can use
\begin{verbatim}
subClassOf(employee,worker).
\end{verbatim}
while the assertion \textit{worker} is equal to the intersection of \textit{person} and not \textit{unemployed}
\begin{verbatim}
equivalentClasses([worker,
           intersectionOf([person,complementOf(unemployed)])]).
\end{verbatim}


Annotation assertions can be defined, for example, as
\begin{verbatim}
annotationAssertion(foo:myAnnotation,
    subClassOf(employee,worker),'myValue').
\end{verbatim}


In particular, an axiom can be annotated with a probability which defines the degree of belief in the truth of the axiom. See Section \ref{semantics} for details.


Below, an example of an probabilistic axiom, following the Prolog syntax.
\begin{verbatim}
annotationAssertion('disponte:probability',
    subClassOf(employee,worker),literal('0.6')).
\end{verbatim}

\subsubsection{Concepts descriptions}
Complex concepts can be defined using different operators:

\noindent
Existential and universal quantifiers
\begin{verbatim}
someValuesFrom("propertyIRI","classIRI").
allValuesFrom("propertyIRI","classIRI").
\end{verbatim}
Union and intersection of concepts
\begin{verbatim}
unionOf([list,of,classes]).
intersectionOf([list,of,classes]).
\end{verbatim}
Cardinality descriptions
\begin{verbatim}
exactCardinality(cardinality,"propertyIRI").
exactCardinality(cardinality,"propertyIRI","classIRI").
maxCardinality(cardinality,"propertyIRI").
maxCardinality(cardinality,"propertyIRI","classIRI").
minCardinality(cardinality,"propertyIRI").
minCardinality(cardinality,"propertyIRI","classIRI").
\end{verbatim}
Complement of a concept
\begin{verbatim}
complementOf("classIRI").
\end{verbatim}
Nominal concept
\begin{verbatim}
oneOf([list,of,classes]).
\end{verbatim}

For example, the class \textit{workingman} is the intersection of \textit{worker} with the union of \textit{man} and \textit{woman}. It can be defined as:
\begin{verbatim}
equivalentClasses([workingman,
    intersectionOf([worker,unionOf([man,woman])])]).
\end{verbatim}

\subsection{RDF/XML syntax and tools}
\label{rdfxml-syn}
As said before, TRILL is able to automatically translate RDF/XML knowledge bases when passed as a string using 
the preticate \verb|owl_rdf/1|.

Consider the following axioms 

\begin{verbatim}
classAssertion(Cat,fluffy)
subClassOf(Cat,Pet)
propertyAssertion(hasAnimal,kevin,fluffy)
\end{verbatim}

The first axiom states that \textit{fluffy} is a \textit{Cat}. The second states that every \textit{Cat} is also a \textit{Pet}. The third states that the role \textit{hasAnimal} links together \textit{kevin} and \textit{fluffy}.

RDF (Resource Descritpion Framework) is a standard W3C. See the \href{http://www.w3.org/TR/REC-rdf-syntax/}{syntax specification} for more details.
RDF is a standard XML-based used for representing knowledge by means of triples.
A representations of the three axioms seen above is shown below.
\begin{verbatim}
<owl:NamedIndividual rdf:about="fluffy">
  <rdf:type rdf:resource="Cat"/>
</owl:NamedIndividual>

<owl:Class rdf:about="Cat">
  <rdfs:subClassOf rdf:resource="Pet"/>
</owl:Class>

<owl:ObjectProperty rdf:about="hasAnimal"/>
<owl:NamedIndividual rdf:about="kevin">
 <hasAnimal rdf:resource="fluffy"/>
</owl:NamedIndividual>
\end{verbatim}

Annotations are assertable using an extension of RDF/XML. For example the annotated axiom below, defined using the Prolog sintax
\begin{verbatim}
annotationAssertion('disponte:probability',
    subClassOf('Cat','Pet'),literal('0.6')).
\end{verbatim}
is modeled using RDF/XML syntax as
\begin{verbatim}
<owl:Class rdf:about="Cat">
 <rdfs:subClassOf rdf:resource="Pet"/>
</owl:Class>
<owl:Axiom>
 <disponte:probability rdf:datatype="&amp;xsd;decimal">
     0.6
 </disponte:probability>
 <owl:annotatedSource rdf:resource="Cat"/>
 <owl:annotatedTarget rdf:resource="Pet"/>
 <owl:annotatedProperty rdf:resource="&amp;rdfs;subClassOf"/>
</owl:Axiom>
\end{verbatim}
If you define the annotated axiom in the RDF/XML part, the annotation must be declared in the knowledge base as follow
\begin{verbatim}
<!DOCTYPE rdf:RDF [
 ...
 <!ENTITY disponte "https://sites.google.com/a/unife.it/ml/disponte#" >
]>

<rdf:RDF
 ...
 xmlns:disponte="https://sites.google.com/a/unife.it/ml/disponte#"
 ...>

 ...
 <owl:AnnotationProperty rdf:about="&amp;disponte;probability"/>
 ...
</rdf:RDF>
\end{verbatim}

There are many \href{http://www.w3.org/2001/sw/wiki/Category:Editor}{editors} for developing knowledge bases.
