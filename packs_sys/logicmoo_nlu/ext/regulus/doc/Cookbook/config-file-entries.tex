
\section{{\tt alterf\_\-patterns\_\-file}}
\label{Section:alterf--patterns--file}


Relevant if you are doing Alterf processing with LF patterns. Points
to a file containing Alterf LF patterns, that can be tested using the
CHECK\_\-ALTERF\_\-PATTERNS command.



\section{{\tt alterf\_\-sents\_\-file}}
\label{Section:alterf--sents--file}




{\em [No documentation yet.]}

\section{{\tt alterf\_\-treebank\_\-file}}
\label{Section:alterf--treebank--file}




{\em [No documentation yet.]}

\section{{\tt analysis\_\-time\_\-limit}}
\label{Section:analysis--time--limit}




{\em [No documentation yet.]}

\section{{\tt answer\_\-config\_\-file}}
\label{Section:answer--config--file}




{\em [No documentation yet.]}

\section{{\tt batchrec\_\-trace}}
\label{Section:batchrec--trace}




{\em [No documentation yet.]}

\section{{\tt batchrec\_\-trace\_\-prolog}}
\label{Section:batchrec--trace--prolog}




{\em [No documentation yet.]}

\section{{\tt batchrec\_\-trace\_\-prolog\_\-with\_\-transcriptions}}
\label{Section:batchrec--trace--prolog--with--transcriptions}




{\em [No documentation yet.]}

\section{{\tt batchrec\_\-trace\_\-prolog\_\-with\_\-transcriptions(Arg1)}}
\label{Section:batchrec--trace--prolog--with--transcriptionsArg1}




{\em [No documentation yet.]}

\section{{\tt collocation\_\-rules}}
\label{Section:collocation--rules}


Relevant to translation applications. Points to a file containing
rules for post-transfer collocation processing.



\section{{\tt compiled\_\-collocation\_\-rules}}
\label{Section:compiled--collocation--rules}




{\em [No documentation yet.]}

\section{{\tt compiled\_\-ellipsis\_\-classes}}
\label{Section:compiled--ellipsis--classes}


Relevant to translation applications. Points to a file containing the
compiled form of the ellipsis processing rules.



\section{{\tt compiled\_\-from\_\-interlingua\_\-rules}}
\label{Section:compiled--from--interlingua--rules}




{\em [No documentation yet.]}

\section{{\tt compiled\_\-graphical\_\-orthography\_\-rules}}
\label{Section:compiled--graphical--orthography--rules}




{\em [No documentation yet.]}

\section{{\tt compiled\_\-lf\_\-patterns}}
\label{Section:compiled--lf--patterns}




{\em [No documentation yet.]}

\section{{\tt compiled\_\-lf\_\-rewrite\_\-rules}}
\label{Section:compiled--lf--rewrite--rules}




{\em [No documentation yet.]}

\section{{\tt compiled\_\-original\_\-script\_\-collocation\_\-rules}}
\label{Section:compiled--original--script--collocation--rules}




{\em [No documentation yet.]}

\section{{\tt compiled\_\-original\_\-script\_\-orthography\_\-rules}}
\label{Section:compiled--original--script--orthography--rules}




{\em [No documentation yet.]}

\section{{\tt compiled\_\-orthography\_\-rules}}
\label{Section:compiled--orthography--rules}




{\em [No documentation yet.]}

\section{{\tt compiled\_\-recognition\_\-orthography\_\-rules}}
\label{Section:compiled--recognition--orthography--rules}




{\em [No documentation yet.]}

\section{{\tt compiled\_\-surface\_\-constituent\_\-rules}}
\label{Section:compiled--surface--constituent--rules}




{\em [No documentation yet.]}

\section{{\tt compiled\_\-to\_\-interlingua\_\-rules}}
\label{Section:compiled--to--interlingua--rules}




{\em [No documentation yet.]}

\section{{\tt compiled\_\-to\_\-source\_\-discourse\_\-rules}}
\label{Section:compiled--to--source--discourse--rules}




{\em [No documentation yet.]}

\section{{\tt compiled\_\-transfer\_\-rules}}
\label{Section:compiled--transfer--rules}




{\em [No documentation yet.]}

\section{{\tt dcg\_\-grammar}}
\label{Section:dcg--grammar}




{\em [No documentation yet.]}

\section{{\tt default\_\-compiled\_\-ellipsis\_\-classes}}
\label{Section:default--compiled--ellipsis--classes}




{\em [No documentation yet.]}

\section{{\tt dialogue\_\-batchrec\_\-trace\_\-prolog\_\-with\_\-transcriptions}}
\label{Section:dialogue--batchrec--trace--prolog--with--transcriptions}




{\em [No documentation yet.]}

\section{{\tt dialogue\_\-batchrec\_\-trace\_\-prolog\_\-with\_\-transcriptions(Arg1)}}
\label{Section:dialogue--batchrec--trace--prolog--with--transcriptionsArg1}




{\em [No documentation yet.]}

\section{{\tt dialogue\_\-corpus}}
\label{Section:dialogue--corpus}




{\em [No documentation yet.]}

\section{{\tt dialogue\_\-corpus(Arg1)}}
\label{Section:dialogue--corpusArg1}




{\em [No documentation yet.]}

\section{{\tt dialogue\_\-corpus\_\-judgements}}
\label{Section:dialogue--corpus--judgements}




{\em [No documentation yet.]}

\section{{\tt dialogue\_\-corpus\_\-results}}
\label{Section:dialogue--corpus--results}




{\em [No documentation yet.]}

\section{{\tt dialogue\_\-corpus\_\-results(Arg1)}}
\label{Section:dialogue--corpus--resultsArg1}




{\em [No documentation yet.]}

\section{{\tt dialogue\_\-files}}
\label{Section:dialogue--files}


Relevant to dialogue applications. Points to a list of files defining
dialogue processing behaviour.



\section{{\tt dialogue\_\-processing\_\-time\_\-limit}}
\label{Section:dialogue--processing--time--limit}




{\em [No documentation yet.]}

\section{{\tt dialogue\_\-rec\_\-params}}
\label{Section:dialogue--rec--params}




{\em [No documentation yet.]}

\section{{\tt dialogue\_\-speech\_\-corpus}}
\label{Section:dialogue--speech--corpus}




{\em [No documentation yet.]}

\section{{\tt dialogue\_\-speech\_\-corpus(Arg1)}}
\label{Section:dialogue--speech--corpusArg1}




{\em [No documentation yet.]}

\section{{\tt dialogue\_\-speech\_\-corpus\_\-results}}
\label{Section:dialogue--speech--corpus--results}




{\em [No documentation yet.]}

\section{{\tt dialogue\_\-speech\_\-corpus\_\-results(Arg1)}}
\label{Section:dialogue--speech--corpus--resultsArg1}




{\em [No documentation yet.]}

\section{{\tt discard\_\-lexical\_\-info\_\-in\_\-ebl\_\-training}}
\label{Section:discard--lexical--info--in--ebl--training}




{\em [No documentation yet.]}

\section{{\tt discriminants}}
\label{Section:discriminants}


Relevant to applications using surface processing. Points to a file of
Alterf discriminants.



\section{{\tt ebl\_\-context\_\-use\_\-threshold}}
\label{Section:ebl--context--use--threshold}


Relevant to applications using grammar specialisation. Defines the
minimum number of examples of a rule that must be present if the
system is to use rule context anti-unification to further constrain
that rule.



\section{{\tt ebl\_\-corpus}}
\label{Section:ebl--corpus}


Points to the file of training examples used as input to the
EBL\_\-TREEBANK operation. Intended originally for use for grammar
specialisation, but can also be used simply to parse a set of examples
to get information about coverage. The format is sent(Atom), so for
example a typical line would be
\begin{verbatim}
sent('switch off the light').
\end{verbatim}
(note the closing period).

If the application compiles multiple top-level specialised grammars,
the grammars relevant to each example are defined in an optional
second argument. For example, if a home control domain had separate
grammars for each room, a typical line in the training file might be
\begin{verbatim}
sent('switch off the light', [bedroom, kitchen, living_room]).
\end{verbatim}



\section{{\tt ebl\_\-filter\_\-pred}}
\label{Section:ebl--filter--pred}




{\em [No documentation yet.]}

\section{{\tt ebl\_\-gemini\_\-grammar}}
\label{Section:ebl--gemini--grammar}


Relevant to applications using grammar specialisation. Specifies the
base name of the Gemini files generated by the EBL\_\-GEMINI command.



\section{{\tt ebl\_\-grammar\_\-probs}}
\label{Section:ebl--grammar--probs}


Convert the current EBL training set, defined by the {\tt ebl\_\-corpus}
config file entry, into a form that can be used as training data by
the Nuance {\tt compute-grammar-probs utility}. The output training data is
placed in the file defined by the {\tt ebl\_\-grammar\_\-probs} config file entry.



\section{{\tt ebl\_\-ignore\_\-feats}}
\label{Section:ebl--ignore--feats}


Relevant to applications using grammar specialisation. The value
should be a list of unification grammar features: these features will
be ignored in the specialised grammar. A suitable choice of value can
greatly speed up Regulus to Nuance compilation for the specialised
grammar.



\section{{\tt ebl\_\-ignore\_\-feats\_\-file}}
\label{Section:ebl--ignore--feats--file}




{\em [No documentation yet.]}

\section{{\tt ebl\_\-include\_\-lex}}
\label{Section:ebl--include--lex}


Relevant to applications using grammar specialisation. Specifies a
file or list of files containing EBL include lex declarations.



\section{{\tt ebl\_\-multiple\_\-grammar\_\-decls}}
\label{Section:ebl--multiple--grammar--decls}




{\em [No documentation yet.]}

\section{{\tt ebl\_\-nuance\_\-grammar}}
\label{Section:ebl--nuance--grammar}


Relevant to applications using grammar specialisation. Points to the specialised Nuance GSL grammar file produced by the EBL\_\-NUANCE operation.



\section{{\tt ebl\_\-operationality}}
\label{Section:ebl--operationality}


Relevant to applications using grammar specialisation. Specifies the operationality criteria.



\section{{\tt ebl\_\-rationalised\_\-corpus}}
\label{Section:ebl--rationalised--corpus}




{\em [No documentation yet.]}

\section{{\tt ebl\_\-raw\_\-regulus\_\-grammar}}
\label{Section:ebl--raw--regulus--grammar}




{\em [No documentation yet.]}

\section{{\tt ebl\_\-regulus\_\-component\_\-grammar}}
\label{Section:ebl--regulus--component--grammar}


Relevant to applications using grammar specialisation that define
multiple top-level specialised grammars. Identifies which specialised
Regulus grammar will be loaded by the EBL\_\-LOAD command.



\section{{\tt ebl\_\-regulus\_\-grammar}}
\label{Section:ebl--regulus--grammar}




{\em [No documentation yet.]}

\section{{\tt ebl\_\-regulus\_\-no\_\-binarise\_\-grammar}}
\label{Section:ebl--regulus--no--binarise--grammar}




{\em [No documentation yet.]}

\section{{\tt ebl\_\-treebank}}
\label{Section:ebl--treebank}


Parse all sentences in current EBL training set, defined by the {\tt
ebl\_\-corpus} config file entry, to create a treebank file. Sentences that
fail to parse are printed out with warning messages, and a summary
statistic is produced at the end of the run. This is very useful for
checking where you are with coverage.



\section{{\tt ellipsis\_\-classes}}
\label{Section:ellipsis--classes}


Relevant to translation applications. Points to a file defining
classes of intersubstitutable phrases that can be used in ellipsis
processing.



\section{{\tt ellipsis\_\-classes\_\-sents\_\-file}}
\label{Section:ellipsis--classes--sents--file}




{\em [No documentation yet.]}

\section{{\tt ellipsis\_\-classes\_\-treebank\_\-file}}
\label{Section:ellipsis--classes--treebank--file}




{\em [No documentation yet.]}

\section{{\tt filtered\_\-interlingua\_\-declarations\_\-file}}
\label{Section:filtered--interlingua--declarations--file}




{\em [No documentation yet.]}

\section{{\tt from\_\-interlingua\_\-rule\_\-learning\_\-config\_\-file}}
\label{Section:from--interlingua--rule--learning--config--file}




{\em [No documentation yet.]}

\section{{\tt from\_\-interlingua\_\-rules}}
\label{Section:from--interlingua--rules}


Relevant to translation applications. Points to a file, or list of
files, containing rules that transfer source language representations
into interlingual representations



\section{{\tt from\_\-interlingua\_\-translation\_\-corpus\_\-judgements}}
\label{Section:from--interlingua--translation--corpus--judgements}




{\em [No documentation yet.]}

\section{{\tt gemini\_\-grammar}}
\label{Section:gemini--grammar}


Specifies the base name of the Gemini files generated by the GEMINI
command.



\section{{\tt generation\_\-dcg\_\-grammar}}
\label{Section:generation--dcg--grammar}




{\em [No documentation yet.]}

\section{{\tt generation\_\-grammar}}
\label{Section:generation--grammar}


Relevant to applications that use generation (typically translation
applications). Points to the file containing the compiled generation
grammar.



\section{{\tt generation\_\-grammar(Arg1)}}
\label{Section:generation--grammarArg1}


Relevant to applications that use generation (typically translation
applications) and also grammar specialisation. Points to the file
containing the compiled specialised generation grammar for the
subdomain tag $\langle$Arg$\rangle$.



\section{{\tt generation\_\-incremental\_\-deepening\_\-parameters}}
\label{Section:generation--incremental--deepening--parameters}


Relevant to applications that use generation (typically translation
applications). Value should be a list of three positive numbers
[$\langle$Start$\rangle$, $\langle$Increment$\rangle$, $\langle$Max$\rangle$], such that both $\langle$Start$\rangle$ and $\langle$Increment$\rangle$
are less than or equal to $\langle$Max$\rangle$. Generation uses an iterative
deepening algorithm, which initially sets a maximum derivation length
of $\langle$Start$\rangle$, and increases it in increments of $\langle$Increment$\rangle$ until it
exceeds $\langle$Max$\rangle$.

Default value is [5, 5, 50].



\section{{\tt generation\_\-module\_\-name}}
\label{Section:generation--module--name}


Relevant to applications that use generation (typically translation
applications). Specifies the module name in the compiled generation
grammar file. Default is generator.



\section{{\tt generation\_\-preferences}}
\label{Section:generation--preferences}


Relevant to applications that use generation (typically translation
applications). Points to the file containing the generation preference
declarations.



\section{{\tt generation\_\-regulus\_\-grammar}}
\label{Section:generation--regulus--grammar}


Relevant to applications that use generation (typically translation
applications). If there is no regulus\_\-grammar entry, points to the
Regulus file, or list of Regulus files, that are to be compiled into
the generation file.



\section{{\tt generation\_\-rules}}
\label{Section:generation--rules}


Relevant to translation applications. Points to the file containing
the generation grammar. Normally this will be a Regulus grammar
compiled for generation. The translation code currently assumes that
this file will define the module generator, and that the top-level
predicate will be of the form
\begin{verbatim}
generator:generate(Representation, Tree, Words)
\end{verbatim}



\section{{\tt generation\_\-rules(Arg1)}}
\label{Section:generation--rulesArg1}




{\em [No documentation yet.]}

\section{{\tt generation\_\-time\_\-limit}}
\label{Section:generation--time--limit}




{\em [No documentation yet.]}

\section{{\tt global\_\-context}}
\label{Section:global--context}


Relevant to translation applications. Defines a value that can be
accessed by conditional-dependent transfer rules, if transfer rules
are to be shared across several applications defined by multiple
config files.



\section{{\tt gloss\_\-generation\_\-rules}}
\label{Section:gloss--generation--rules}




{\em [No documentation yet.]}

\section{{\tt grammar\_\-probs\_\-data}}
\label{Section:grammar--probs--data}




{\em [No documentation yet.]}

\section{{\tt ignore\_\-subdomain}}
\label{Section:ignore--subdomain}


Relevant to applications using grammar specialisation. Sometimes, you
will have defined multiple subdomains, but you will only be carrying
out development in one of them. In this case, you can speed up EBL
training by temporarily adding ignore\_\-subdomain declarations in the
config file. An ignore\_\-subdomain declaration has the form
\begin{verbatim}
regulus_config(ignore_subdomain, <Tag>).
\end{verbatim}
The effect is to remove all references to $\langle$Tag$\rangle$ when performing training, and not build any specialised grammar for $\langle$Tag$\rangle$. You may include any number of ignore\_\-subdomain declarations.



\section{{\tt interlingua\_\-declarations}}
\label{Section:interlingua--declarations}


Relevant to translation applications. Points to the file containing
the interlingua declarations, which define the constants that may be
used at the interlingual representation level.



\section{{\tt interlingua\_\-structure}}
\label{Section:interlingua--structure}




{\em [No documentation yet.]}

\section{{\tt lc\_\-tables\_\-file}}
\label{Section:lc--tables--file}




{\em [No documentation yet.]}

\section{{\tt lf\_\-patterns}}
\label{Section:lf--patterns}


Relevant to dialogue applications. Points to a files of LF patterns.



\section{{\tt lf\_\-patterns\_\-modules}}
\label{Section:lf--patterns--modules}


Relevant to dialogue applications. Value should be a list of modules
referenced by the compiled LF patterns.



\section{{\tt lf\_\-postproc\_\-pred}}
\label{Section:lf--postproc--pred}


Defines a post-processing predicate that is applied after Regulus
analysis. If you are using the riacs\_\-sem semantic macros, you must set
this parameter to the value riacs\_\-postproc\_\-lf.



\section{{\tt macro\_\-expanded\_\-grammar}}
\label{Section:macro--expanded--grammar}




{\em [No documentation yet.]}

\section{{\tt missing\_\-help\_\-class\_\-decls}}
\label{Section:missing--help--class--decls}




{\em [No documentation yet.]}

\section{{\tt nbest\_\-preferences}}
\label{Section:nbest--preferences}




{\em [No documentation yet.]}

\section{{\tt nbest\_\-training\_\-data\_\-file}}
\label{Section:nbest--training--data--file}




{\em [No documentation yet.]}

\section{{\tt no\_\-spaces\_\-in\_\-original\_\-script}}
\label{Section:no--spaces--in--original--script}




{\em [No documentation yet.]}

\section{{\tt nuance\_\-compile\_\-params}}
\label{Section:nuance--compile--params}


Specifies a list of extra compilation parameters to be passed to
Nuance compilation by the NUANCE\_\-COMPILE command. A typical value is
\begin{verbatim}
['-auto_pron', '-dont_flatten']
\end{verbatim}



\section{{\tt nuance\_\-grammar}}
\label{Section:nuance--grammar}


Points to the Nuance GSL grammar produced by the NUANCE command.



\section{{\tt nuance\_\-grammar\_\-for\_\-compilation}}
\label{Section:nuance--grammar--for--compilation}




{\em [No documentation yet.]}

\section{{\tt nuance\_\-grammar\_\-for\_\-pcfg\_\-training}}
\label{Section:nuance--grammar--for--pcfg--training}




{\em [No documentation yet.]}

\section{{\tt nuance\_\-language\_\-pack}}
\label{Section:nuance--language--pack}


Specifies the Nuance language pack to be used by Nuance compilation in
the NUANCE\_\-COMPILE command.



\section{{\tt nuance\_\-recognition\_\-package}}
\label{Section:nuance--recognition--package}




{\em [No documentation yet.]}

\section{{\tt only\_\-translate\_\-up\_\-to\_\-interlingua}}
\label{Section:only--translate--up--to--interlingua}




{\em [No documentation yet.]}

\section{{\tt original\_\-script\_\-collocation\_\-rules}}
\label{Section:original--script--collocation--rules}




{\em [No documentation yet.]}

\section{{\tt original\_\-script\_\-encoding}}
\label{Section:original--script--encoding}




{\em [No documentation yet.]}

\section{{\tt original\_\-script\_\-generation\_\-rules}}
\label{Section:original--script--generation--rules}




{\em [No documentation yet.]}

\section{{\tt original\_\-script\_\-orthography\_\-rules}}
\label{Section:original--script--orthography--rules}




{\em [No documentation yet.]}

\section{{\tt orthography\_\-rules}}
\label{Section:orthography--rules}


Relevant to translation applications. Points to a file containing
rules for post-transfer orthography processing.



\section{{\tt paraphrase\_\-corpus}}
\label{Section:paraphrase--corpus}




{\em [No documentation yet.]}

\section{{\tt paraphrase\_\-generation\_\-grammar}}
\label{Section:paraphrase--generation--grammar}




{\em [No documentation yet.]}

\section{{\tt parse\_\-preferences}}
\label{Section:parse--preferences}


Can be used to define default analysis preferences.



\section{{\tt parsing\_\-history\_\-file}}
\label{Section:parsing--history--file}




{\em [No documentation yet.]}

\section{{\tt pcfg\_\-training\_\-output\_\-directory}}
\label{Section:pcfg--training--output--directory}




{\em [No documentation yet.]}

\section{{\tt prolog\_\-semantics}}
\label{Section:prolog--semantics}


Relevant to generation grammars, in particular paraphrase grammars. If
value is 'yes', allow semantic values to be arbitrary Prolog
expressions (by default, only Regulus-formatted GSL expressions
are allowed).



\section{{\tt reflective\_\-dcg\_\-grammar}}
\label{Section:reflective--dcg--grammar}




{\em [No documentation yet.]}

\section{{\tt reflective\_\-dcg\_\-grammar\_\-for\_\-generation}}
\label{Section:reflective--dcg--grammar--for--generation}




{\em [No documentation yet.]}

\section{{\tt regulus\_\-grammar}}
\label{Section:regulus--grammar}


Points to the Regulus file, or list of Regulus files, that constitute
the main grammar.



\section{{\tt regulus\_\-no\_\-sem\_\-decls}}
\label{Section:regulus--no--sem--decls}


Points to a file which removes the sem feature from the main grammar.



\section{{\tt resolution\_\-preferences}}
\label{Section:resolution--preferences}




{\em [No documentation yet.]}

\section{{\tt role\_\-marked\_\-semantics}}
\label{Section:role--marked--semantics}




{\em [No documentation yet.]}

\section{{\tt stanford\_\-dcg\_\-debug\_\-grammar}}
\label{Section:stanford--dcg--debug--grammar}




{\em [No documentation yet.]}

\section{{\tt stanford\_\-dcg\_\-grammar}}
\label{Section:stanford--dcg--grammar}




{\em [No documentation yet.]}

\section{{\tt strcat\_\-semantics}}
\label{Section:strcat--semantics}




{\em [No documentation yet.]}

\section{{\tt surface\_\-constituent\_\-rules}}
\label{Section:surface--constituent--rules}


Relevant to applications using surface processing. Points to the
surface constituent rules file.



\section{{\tt surface\_\-patterns}}
\label{Section:surface--patterns}


Relevant to applications using surface processing. Points to the
surface patterns file.



\section{{\tt surface\_\-postprocessing}}
\label{Section:surface--postprocessing}


Relevant to applications using surface processing. Points to a file
that defines a post-processing predicate that can be applied to the
results of surface processing. The file should define a predicate
\begin{verbatim}
surface_postprocess(Representation, 
                    PostProcessedRepresentation).
\end{verbatim}



\section{{\tt tagging\_\-grammar}}
\label{Section:tagging--grammar}


Relevant to applications using surface processing. Points to a file
that defines a tagging grammar, in DCG form. The top-level rule should
be of the form
\begin{verbatim}
tagging_grammar(Item) --> <Body>.
\end{verbatim}



\section{{\tt target\_\-model}}
\label{Section:target--model}


Relevant to applications using surface processing. Points to a file
defining a target model. The file should define the predicates
target\_\-atom/1 and target\_\-atom\_\-excludes/2.



\section{{\tt targeted\_\-help\_\-backed\_\-off\_\-corpus\_\-file}}
\label{Section:targeted--help--backed--off--corpus--file}




{\em [No documentation yet.]}

\section{{\tt targeted\_\-help\_\-classes\_\-file}}
\label{Section:targeted--help--classes--file}




{\em [No documentation yet.]}

\section{{\tt targeted\_\-help\_\-corpus\_\-file}}
\label{Section:targeted--help--corpus--file}




{\em [No documentation yet.]}

\section{{\tt targeted\_\-help\_\-source\_\-files}}
\label{Section:targeted--help--source--files}




{\em [No documentation yet.]}

\section{{\tt test\_\-corpus}}
\label{Section:test--corpus}




{\em [No documentation yet.]}

\section{{\tt tmp\_\-ebl\_\-operational\_\-file}}
\label{Section:tmp--ebl--operational--file}




{\em [No documentation yet.]}

\section{{\tt tmp\_\-preds}}
\label{Section:tmp--preds}




{\em [No documentation yet.]}

\section{{\tt to\_\-interlingua\_\-rule\_\-learning\_\-config\_\-file}}
\label{Section:to--interlingua--rule--learning--config--file}




{\em [No documentation yet.]}

\section{{\tt to\_\-interlingua\_\-rules}}
\label{Section:to--interlingua--rules}


Relevant to translation applications. Points to a file, or list of
files, containing rules that transfer interlingual representations
into target language representations.



\section{{\tt to\_\-interlingua\_\-translation\_\-corpus\_\-judgements}}
\label{Section:to--interlingua--translation--corpus--judgements}




{\em [No documentation yet.]}

\section{{\tt to\_\-source\_\-discourse\_\-rules}}
\label{Section:to--source--discourse--rules}


Relevant to translation applications. Points to a file, or list of
files, containing rules that transfer source representations into
source discourse representations.



\section{{\tt top\_\-level\_\-cat}}
\label{Section:top--level--cat}


Defines the top-level category of the grammar.



\section{{\tt top\_\-level\_\-generation\_\-cat}}
\label{Section:top--level--generation--cat}


Relevant to applications that use generation (typically translation
applications). Defines the top-level category of the generation
grammar. Default is .MAIN.




\section{{\tt top\_\-level\_\-generation\_\-feat}}
\label{Section:top--level--generation--feat}


Relevant to applications that use generation (typically translation
applications). Defines the semantic feature in the top-level rule
which holds the semantic value. Normally, the rule will be of the form
\begin{verbatim}
'.MAIN':[gsem=[value=Sem]] --> Body
\end{verbatim}
and the value of this parameter will be value (default if not
specified).



\section{{\tt top\_\-level\_\-generation\_\-pred}}
\label{Section:top--level--generation--pred}


Relevant to applications that use generation (typically translation
applications). Defines the top-level category of the generation
grammar. For translation applications, the value should be generate
(default if not specified).



\section{{\tt transfer\_\-rules}}
\label{Section:transfer--rules}


Relevant to translation applications. Points to a file, or list of
files, containing rules that transfer source language representations
into target language representations.



\section{{\tt translate\_\-from\_\-interlingua}}
\label{Section:translate--from--interlingua}




{\em [No documentation yet.]}

\section{{\tt translation\_\-corpus}}
\label{Section:translation--corpus}


Relevant to translation applications. Points to a file of examples
used as input to the TRANSLATE\_\-CORPUS command. The format is
sent(Atom), so for example a typical line would be
\begin{verbatim}
sent('switch off the light').
\end{verbatim}
(note the closing period).



\section{{\tt translation\_\-corpus(Arg1)}}
\label{Section:translation--corpusArg1}


Relevant to translation applications. Points to a file of examples
used as input to the parameterised command TRANSLATE\_\-CORPUS
$\langle$Arg$\rangle$. The format is sent(Atom), so for example a
typical line would be
\begin{verbatim}
sent('switch off the light').
\end{verbatim}
(note the closing period).



\section{{\tt translation\_\-corpus\_\-judgements}}
\label{Section:translation--corpus--judgements}


Relevant to translation applications. Points to a file of recognition
judgements. You should not normally edit this file directly, but
update it using the command UPDATE\_\-RECOGNITION\_\-JUDGEMENTS.



\section{{\tt translation\_\-corpus\_\-recognition\_\-judgements}}
\label{Section:translation--corpus--recognition--judgements}




{\em [No documentation yet.]}

\section{{\tt translation\_\-corpus\_\-results}}
\label{Section:translation--corpus--results}


Relevant to translation applications. Points to the file containing
the result of running the TRANSLATE\_\-CORPUS command. You can then edit
this file to update judgements, and incorporate them into the
translation\_\-corpus\_\-judgements file by using the command
UPDATE\_\-TRANSLATION\_\-JUDGEMENTS.



\section{{\tt translation\_\-corpus\_\-results(Arg1)}}
\label{Section:translation--corpus--resultsArg1}


Relevant to translation applications. Points to the file containing
the result of running the parameterised command TRANSLATE\_\-CORPUS
$\langle$Arg$\rangle$. You can then edit this file to update judgements, and
incorporate them into the translation\_\-corpus\_\-judgements file by using
the parameterised command UPDATE\_\-TRANSLATION\_\-JUDGEMENTS $\langle$Arg$\rangle$.



\section{{\tt translation\_\-corpus\_\-tmp\_\-recognition\_\-judgements}}
\label{Section:translation--corpus--tmp--recognition--judgements}


Relevant to translation applications. Points to the file of new
recognition results generated by running the TRANSLATE\_\-SPEECH\_\-CORPUS
command. You can then edit this file to update the judgements, and
incorporate them into the translation\_\-corpus\_\-recognition\_\-judgements
file using the command UPDATE\_\-RECOGNITION\_\-JUDGEMENTS.



\section{{\tt translation\_\-corpus\_\-tmp\_\-recognition\_\-judgements(Arg1)}}
\label{Section:translation--corpus--tmp--recognition--judgementsArg1}


Relevant to translation applications. Points to the file of new
recognition results generated by running the TRANSLATE\_\-SPEECH\_\-CORPUS
$\langle$Arg$\rangle$ command. You can then edit this file to update the judgements,
and incorporate them into the
translation\_\-corpus\_\-recognition\_\-judgements file using the command
UPDATE\_\-RECOGNITION\_\-JUDGEMENTS $\langle$Arg$\rangle$.



\section{{\tt translation\_\-rec\_\-params}}
\label{Section:translation--rec--params}


Relevant to translation applications. Specifies the list of Nuance
parameters that will be used when carrying out recognition for the
TRANSLATE\_\-SPEECH\_\-CORPUS command. These parameters must at a minimum
specify the recognition package and the top-level Nuance grammar, for
example
\begin{verbatim}
[package=med_runtime(recogniser), grammar='.MAIN']
\end{verbatim}



\section{{\tt translation\_\-speech\_\-corpus}}
\label{Section:translation--speech--corpus}


Relevant to translation applications. Points to a file of examples
used as input to the TRANSLATE\_\-SPEECH\_\-CORPUS command. The format is
$\langle$Wavfile$\rangle$ $\langle$Words$\rangle$, so for example a typical line would be
\begin{verbatim}
C:/Regulus/data/utt03.wav switch off the light
\end{verbatim}



\section{{\tt translation\_\-speech\_\-corpus(Arg1)}}
\label{Section:translation--speech--corpusArg1}


Relevant to translation applications. Points to a file of examples
used as input to the TRANSLATE\_\-SPEECH\_\-CORPUS($\langle$Arg$\rangle$)
command. The format is $\langle$Wavfile$\rangle$
$\langle$Words$\rangle$, so for example a typical line would be
\begin{verbatim}
C:/Regulus/data/utt03.wav switch off the light
\end{verbatim}



\section{{\tt translation\_\-speech\_\-corpus\_\-results}}
\label{Section:translation--speech--corpus--results}


Relevant to translation applications. Points to the file containing
the result of running the TRANSLATE\_\-SPEECH\_\-CORPUS command. You can
then edit this file to update judgements, and incorporate them into
the translation\_\-corpus\_\-judgements file by using the command
UPDATE\_\-TRANSLATION\_\-JUDGEMENTS\_\-SPEECH.



\section{{\tt translation\_\-speech\_\-corpus\_\-results(Arg1)}}
\label{Section:translation--speech--corpus--resultsArg1}


Relevant to translation applications. Points to the file containing
the result of running the TRANSLATE\_\-SPEECH\_\-CORPUS $\langle$Arg$\rangle$ command. You
can then edit this file to update judgements, and incorporate them
into the translation\_\-corpus\_\-judgements file by using the command
UPDATE\_\-TRANSLATION\_\-JUDGEMENTS\_\-SPEECH $\langle$Arg$\rangle$.



\section{{\tt tts\_\-command}}
\label{Section:tts--command}




{\em [No documentation yet.]}

\section{{\tt wavfile\_\-directory}}
\label{Section:wavfile--directory}


Relevant to translation applications. If output speech is to be
produced using recorded wavfiles, points to the directory that holds
these files.



\section{{\tt wavfile\_\-preceding\_\-context}}
\label{Section:wavfile--preceding--context}




{\em [No documentation yet.]}

\section{{\tt wavfile\_\-preceding\_\-context(Arg1)}}
\label{Section:wavfile--preceding--contextArg1}




{\em [No documentation yet.]}

\section{{\tt wavfile\_\-recording\_\-script}}
\label{Section:wavfile--recording--script}


Relevant to translation applications. If output speech is to be
produced using recorded wavfiles, points to an automatically created
file that holds a script which can be used to create the missing
wavfiles. This script is produced by finding all the lexical items in
the file referenced by generation\_\-rules, and creating an entry for
every item not already in wavfile\_\-directory. The file is created as
part of the processing carried out by the LOAD\_\-TRANSLATE command.

Due to limitations of some operating systems the script contains some latin-1 characters translated to character sequences shown in the table below. 
\begin{verbatim}
Char   Translates to
�      a1
�      a2
�      a3
�      a4
�      a5
�      c1
�      e1
�      e2
�      e3
�      e4
�      e6
�      n1
�      o1
�      o2
�      o3
�      o4
�      u1
�      u2
�      u3
�      u4
\end{verbatim}



\section{{\tt wavfiles}}
\label{Section:wavfiles}



Show wavfiles recorded using recognition from the top-level
(the RECOGNISE command). Each file is displayed together with
a timestamp and associated text. The associated text is a 
transcription if one is available, or the recognition result
otherwise. 

The files shown by this command can be used as top-level 
recorded speech input by typing
\begin{verbatim}
WAVFILE <Wavfile>
\end{verbatim}
e.g.
\begin{verbatim}
WAVFILE c:/Regulus/recorded_wavfiles/2008-04-24_22-36-03/utt05.wav
\end{verbatim}



\section{{\tt working\_\-directory}}
\label{Section:working--directory}


Working files will have names starting with this prefix.



\section{{\tt working\_\-file\_\-prefix}}
\label{Section:working--file--prefix}




{\em [No documentation yet.]}
