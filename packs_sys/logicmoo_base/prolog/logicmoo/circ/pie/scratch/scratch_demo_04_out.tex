\documentclass[a4paper]{article}
\usepackage[utf8]{inputenc}
\usepackage[T1]{fontenc}
\usepackage{imakeidx}
\usepackage[hidelinks]{hyperref}
%% A screen friendly geometry:	    
\usepackage[paper=a5paper,scale=0.9]{geometry}
%% PPL Setup
	   
\newcommand{\assign}{\mathrel{\mathop:}=}
\newcommand{\concat}{\mathrel{+\!+}}
	    
\newcommand{\f}[1]{\mathsf{#1}}
\newcommand{\true}{\top}
\newcommand{\false}{\bot}
\newcommand{\imp}{\rightarrow}
\newcommand{\revimp}{\leftarrow}
\newcommand{\equi}{\leftrightarrow}
\newcommand{\entails}{\models}	    
\newcommand{\eqdef}{\; 
\raisebox{-0.1ex}[0mm]{$ \stackrel{\raisebox{-0.2ex}{\tiny 
\textnormal{def}}}{=} $}\; }
\newcommand{\iffdef}{\n{iff}_{\mbox{\scriptsize \textnormal{def}}}}

\newcommand{\pplmacro}[1]{\mathit{#1}}
\newcommand{\ppldefmacro}[1]{\mathit{#1}}
\newcommand{\pplparam}[1]{\mathit{#1}}
\newcommand{\pplparamidx}[2]{\mathit{#1}_{#2}}
\newcommand{\pplparamplain}[1]{#1}
\newcommand{\pplparamplainidx}[2]{#1_{#2}}
\newcommand{\pplparamsup}[2]{\mathit{#1}^{#2}}
\newcommand{\pplparamsupidx}[3]{\mathit{#1}^{#2}_{#3}}
\newcommand{\pplparamplainsup}[2]{#1^{#2}}
\newcommand{\pplparamplainsupidx}[3]{#1^{#2}_{#3}}
\newcommand{\pplparamnum}[1]{\mathit{X}_{#1}}

%%	    
%% We use @startsection just to obtain reduced vertical spacing above
%% macro headers which are immediately after other headers, e.g. of sections
%%	    
\makeatletter%
\newcounter{entry}%
\newcommand{\entrymark}[1]{}%
\newcommand\entryhead{%
\@startsection{entry}{10}{\z@}{12pt plus 2pt minus 2pt}{0pt}{}}%
\makeatother
	    
\newcommand{\pplkbBefore}
{\entryhead*{}%
\setlength{\arraycolsep}{0pt}%
\pagebreak[0]%
\begin{samepage}%
\noindent%
\rule[0.5pt]{\textwidth}{2pt}\\%
\noindent}

% \newcommand{\pplkbDefType}[1]{\hspace{\fill}{{[}#1{]}\\}}

\newcommand{\pplkbBetween}
{\setlength{\arraycolsep}{3pt}%
\\\rule[3pt]{\textwidth}{1pt}%
\par\nopagebreak\noindent Defined as\begin{center}}

\newcommand{\pplkbAfter}{\end{center}\end{samepage}\noindent}

\newcommand{\pplkbBodyBefore}{\par\noindent where\begin{center}}
\newcommand{\pplkbBodyAfter}{\end{center}}

\newcommand{\pplkbFreePredicates}[1]{\f{free\_predicates}(#1)}
% \newcommand{\pplkbRenameFreeOccurrences}[3]{\f{rename\_free\_occurrences}(#1,#2,#3)}

\newcommand{\pplIsValid}[1]{\noindent This formula is valid: $#1$\par}
\newcommand{\pplIsNotValid}[1]{\noindent This formula is not valid: $#1$\par}	    
\newcommand{\pplFailedToValidate}[1]{\noindent Failed to validate this formula: $#1$\par}

\newcounter{def}
	    
\makeindex
\usepackage{graphicx}

\begin{document}
%
% Doc at position 0
%
 \title{\textit{PIE} Example Document}
  \date{Revision: March 28, 2020; Rendered: \today}
  \maketitle
%
% Doc at position 386
%
 \section{Projection and Definientia} 
%
% Statement at position 431
%
\pplkbBefore
\index{project(S,F)@$\ppldefmacro{project}(\pplparamplain{S},\pplparamplain{F})$}$\begin{array}{lllll}
\ppldefmacro{project}(\pplparamplain{S},\pplparamplain{F})
\end{array}
$\pplkbBetween
$\begin{array}{lllll}
\exists \pplparamplainidx{S}{1} \, \pplparamplain{F},
\end{array}
$\pplkbAfter
\pplkbBodyBefore
$
\begin{array}{l}\pplparamplainidx{S}{2} \assign \pplkbFreePredicates{\pplparamplain{F}},\\
\pplparamplainidx{S}{1} \assign \pplparamplainidx{S}{2} \setminus \pplparamplain{S}.

\end{array}$\pplkbBodyAfter
%
% Statement at position 639
%
\pplkbBefore
\index{definiens(G,F,S)@$\ppldefmacro{definiens}(\pplparamplain{G},\pplparamplain{F},\pplparamplain{S})$}$\begin{array}{lllll}
\ppldefmacro{definiens}(\pplparamplain{G},\pplparamplain{F},\pplparamplain{S})
\end{array}
$\pplkbBetween
$\begin{array}{lllll}
\pplmacro{project}(\pplparamplain{S},(\pplparamplain{F} \land  \pplparamplain{G})) \imp  \lnot  \pplmacro{project}(\pplparamplain{S},(\pplparamplain{F} \land  \lnot  \pplparamplain{G})).
\end{array}
$\pplkbAfter
%
% Doc at position 712
%
 \section{Obtaining a Definiens with Interpolation} 
%
% Doc at position 771
%
We specify a background knowledge base:
%
% Statement at position 819
%
\pplkbBefore
\index{kb_1@$\ppldefmacro{kb_{1}}$}$\begin{array}{lllll}
\ppldefmacro{kb_{1}}
\end{array}
$\pplkbBetween
$\begin{array}{lllll}
\forall \mathit{x} \, ((\mathsf{q}\mathit{x} \imp  \mathsf{p}\mathit{x}) \land  (\mathsf{p}\mathit{x} \imp  \mathsf{r}\mathit{x}) \land  (\mathsf{r}\mathit{x} \imp  \mathsf{q}\mathit{x})).
\end{array}
$\pplkbAfter
%
% Doc at position 889
%
\bigskip

To obtain a definiens of $\exists x\, \f{p}x$ in terms $\{\f{q}, \f{r}\}$
within $\mathit{kb_1}$ we specify an implication with the $\mathit{definiens}$
macro:
%
% Statement at position 1067
%
\pplkbBefore
\index{ex_1@$\ppldefmacro{ex_{1}}$}$\begin{array}{lllll}
\ppldefmacro{ex_{1}}
\end{array}
$\pplkbBetween
$\begin{array}{lllll}
\pplmacro{definiens}(\exists \mathit{x} \, \mathsf{p}\mathit{x},\pplmacro{kb_{1}},{[}\mathsf{q},\mathsf{r}{]}).
\end{array}
$\pplkbAfter
%
% Doc at position 1132
%
\bigskip

We now invoke a utility predicate that computes and prints for a given valid
implication an interpolant of its antecedent and consequent:

\noindent Input: $\pplmacro{ex_{1}}.$\\
\noindent Result of interpolation:
\[\begin{array}{lllll}
\exists \mathit{x} \, \mathsf{q}\mathit{x}.
\end{array}
\]
%
% Doc at position 1368
%
The proof underlying interpolant extraction can be visualized, colors
representing the sides with respect to interpolation. The color of
the closing marks indicate the side of the connection partner:

\begin{center}
\includegraphics[width=15em]{/tmp/tmp_pie_dotgraph01}
\end{center}
%
% Doc at position 1659
%
Before we leave that example, we take a look at the expansion of
$\mathit{ex}_1$:
\[\begin{array}{lllll}
\exists \mathit{p} \, (\forall \mathit{x} \, ((\mathsf{q}\mathit{x} \imp  \mathit{p}\mathit{x}) \land  (\mathit{p}\mathit{x} \imp  \mathsf{r}\mathit{x}) \land  (\mathsf{r}\mathit{x} \imp  \mathsf{q}\mathit{x})) &&&\; \land \\
\hphantom{\exists \mathit{p} \, (} \exists \mathit{x} \, \mathit{p}\mathit{x}) &&&&\; \imp \\
\lnot  \exists \mathit{p} \, (\forall \mathit{x} \, ((\mathsf{q}\mathit{x} \imp  \mathit{p}\mathit{x}) \land  (\mathit{p}\mathit{x} \imp  \mathsf{r}\mathit{x}) \land  (\mathsf{r}\mathit{x} \imp  \mathsf{q}\mathit{x})) &&&\; \land \\
\hphantom{\lnot  \exists \mathit{p} \, (} \lnot  \exists \mathit{x} \, \mathit{p}\mathit{x}).
\end{array}
\]
%
% Doc at position 1799
%
  And we invoke an external prover (\textit{Prover9}) to validate it:

\pplIsValid{\pplmacro{ex_{1}}.}
%
% Doc at position 2119
%
 \section{Obtaining the Strongest Definiens with Elimination} 
%
% Doc at position 2188
%
The antecedent of the implication in the expansion of $\mathit{definiens}$
specifies the strongest necessary condition of $G$ on $S$ within $F$.  In case
definability holds (that is, the implication is valid), this antecedent denotes
the strongest definiens. In the example it has a first-order equivalent
that can be computed by second-order quantifier elimination.
%
% Statement at position 2563
%
\pplkbBefore
\index{ex_2@$\ppldefmacro{ex_{2}}$}$\begin{array}{lllll}
\ppldefmacro{ex_{2}}
\end{array}
$\pplkbBetween
$\begin{array}{lllll}
\pplmacro{project}({[}\mathsf{q},\mathsf{r}{]},(\pplmacro{kb_{1}} \land  \exists \mathit{x} \, \mathsf{p}\mathit{x})).
\end{array}
$\pplkbAfter

\noindent Input: $\pplmacro{ex_{2}}.$\\
\noindent Result of elimination:
\[\begin{array}{lllll}
\forall \mathit{x} \, (\mathsf{r}\mathit{x} \imp  \mathsf{q}\mathit{x}) \land  \forall \mathit{x} \, (\mathsf{q}\mathit{x} \imp  \mathsf{r}\mathit{x}) \land  \exists \mathit{x} \, \mathsf{r}\mathit{x}.
\end{array}
\]
\printindex
\end{document}
