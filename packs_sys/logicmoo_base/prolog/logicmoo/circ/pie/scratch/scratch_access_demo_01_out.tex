\documentclass[a4paper]{article}
\usepackage[utf8]{inputenc}
\usepackage[T1]{fontenc}
\usepackage{imakeidx}
\usepackage[hidelinks]{hyperref}
%% A screen friendly geometry:	    
\usepackage[paper=a5paper,scale=0.9]{geometry}
%% PPL Setup
	   
\newcommand{\assign}{\mathrel{\mathop:}=}
\newcommand{\concat}{\mathrel{+\!+}}
	    
\newcommand{\f}[1]{\mathsf{#1}}
\newcommand{\true}{\top}
\newcommand{\false}{\bot}
\newcommand{\imp}{\rightarrow}
\newcommand{\revimp}{\leftarrow}
\newcommand{\equi}{\leftrightarrow}
\newcommand{\entails}{\models}	    
\newcommand{\eqdef}{\; 
\raisebox{-0.1ex}[0mm]{$ \stackrel{\raisebox{-0.2ex}{\tiny 
\textnormal{def}}}{=} $}\; }
\newcommand{\iffdef}{\n{iff}_{\mbox{\scriptsize \textnormal{def}}}}

\newcommand{\pplmacro}[1]{\mathit{#1}}
\newcommand{\ppldefmacro}[1]{\mathit{#1}}
\newcommand{\pplparam}[1]{\mathit{#1}}
\newcommand{\pplparamidx}[2]{\mathit{#1}_{#2}}
\newcommand{\pplparamplain}[1]{#1}
\newcommand{\pplparamplainidx}[2]{#1_{#2}}
\newcommand{\pplparamsup}[2]{\mathit{#1}^{#2}}
\newcommand{\pplparamsupidx}[3]{\mathit{#1}^{#2}_{#3}}
\newcommand{\pplparamplainsup}[2]{#1^{#2}}
\newcommand{\pplparamplainsupidx}[3]{#1^{#2}_{#3}}
\newcommand{\pplparamnum}[1]{\mathit{X}_{#1}}

%%	    
%% We use @startsection just to obtain reduced vertical spacing above
%% macro headers which are immediately after other headers, e.g. of sections
%%	    
\makeatletter%
\newcounter{entry}%
\newcommand{\entrymark}[1]{}%
\newcommand\entryhead{%
\@startsection{entry}{10}{\z@}{12pt plus 2pt minus 2pt}{0pt}{}}%
\makeatother
	    
\newcommand{\pplkbBefore}
{\entryhead*{}%
\setlength{\arraycolsep}{0pt}%
\pagebreak[0]%
\begin{samepage}%
\noindent%
\rule[0.5pt]{\textwidth}{2pt}\\%
\noindent}

% \newcommand{\pplkbDefType}[1]{\hspace{\fill}{{[}#1{]}\\}}

\newcommand{\pplkbBetween}
{\setlength{\arraycolsep}{3pt}%
\\\rule[3pt]{\textwidth}{1pt}%
\par\nopagebreak\noindent Defined as\begin{center}}

\newcommand{\pplkbAfter}{\end{center}\end{samepage}\noindent}

\newcommand{\pplkbBodyBefore}{\par\noindent where\begin{center}}
\newcommand{\pplkbBodyAfter}{\end{center}}

\newcommand{\pplkbFreePredicates}[1]{\f{free\_predicates}(#1)}
% \newcommand{\pplkbRenameFreeOccurrences}[3]{\f{rename\_free\_occurrences}(#1,#2,#3)}

\newcommand{\pplIsValid}[1]{\noindent This formula is valid: $#1$\par}
\newcommand{\pplIsNotValid}[1]{\noindent This formula is not valid: $#1$\par}	    
\newcommand{\pplFailedToValidate}[1]{\noindent Failed to validate this formula: $#1$\par}

\newcounter{def}
	    
\makeindex

\begin{document}
%
% Doc at position 0
%
\title{Access Predicates Demo}
\date{Revision: March 7, 2017; Rendered: \today}
\maketitle

\noindent Illustration of the computation of access predicates for an RDF fact
base. Makes use of scratch\_forgetting and scratch\_definientia. Formalized
with the \href{http://cs.christophwernhard.com/pie/}{\textit{PIE}} system.
%
% Doc at position 621
%

\section{Introduction}

We assume a RDF knowledge base with facts like 

\begin{center}
\begin{tabular}{l}
\texttt{rdf\_triple(p1, name, 'Sulzer').}\\
\texttt{rdf\_triple(p1, yob, 1720).}
\end{tabular}
\end{center}

\section{Specifications of Access Predicates}

Specifications of access predicates for \texttt{person\_name} and
\texttt{person\_yob}. In practice, these would be automatically generated from
the schema of the knowledge base.

%
% Statement at position 1072
%
\pplkbBefore
\index{accessor_spec@$\ppldefmacro{accessor\_spec}$}$\begin{array}{lllll}
\ppldefmacro{accessor\_spec}
\end{array}
$\pplkbBetween
$\begin{array}{lllll}
\forall \mathit{p}\mathit{n} \, (\mathsf{b}\mathit{p} \imp  (\mathsf{person\_name}(\mathit{p},\mathit{n}) \equi  \mathsf{person\_name\_bf}(\mathit{p},\mathit{n}))) &&&&\; \land \\
\forall \mathit{p}\mathit{n} \, (\mathsf{b}\mathit{n} \imp  (\mathsf{person\_name}(\mathit{p},\mathit{n}) \equi  \mathsf{person\_name\_fb}(\mathit{p},\mathit{n}))) &&&&\; \land \\
\forall \mathit{p}\mathit{n} \, (\mathsf{b}\mathit{p} \imp  (\mathsf{person\_yob}(\mathit{p},\mathit{n}) \equi  \mathsf{person\_yob\_bf}(\mathit{p},\mathit{n}))) &&&&\; \land \\
\forall \mathit{p}\mathit{n} \, (\mathsf{b}\mathit{n} \imp  (\mathsf{person\_yob}(\mathit{p},\mathit{n}) \equi  \mathsf{person\_yob\_fb}(\mathit{p},\mathit{n}))) &&&&\; \land \\
\forall \mathit{p}\mathit{n} \, (\mathsf{person\_name}(\mathit{p},\mathit{n}) \imp  \mathsf{b}\mathit{p} \land  \mathsf{b}\mathit{n}) &&&&\; \land \\
\forall \mathit{p}\mathit{n} \, (\mathsf{person\_yob}(\mathit{p},\mathit{n}) \imp  \mathsf{b}\mathit{p} \land  \mathsf{b}\mathit{n}).
\end{array}
$\pplkbAfter
%
% Statement at position 1456
%
\pplkbBefore
\index{accessor_spec_1@$\ppldefmacro{accessor\_spec_{1}}$}$\begin{array}{lllll}
\ppldefmacro{accessor\_spec_{1}}
\end{array}
$\pplkbBetween
$\begin{array}{lllll}
\forall \mathit{p}\mathit{n} \, (\mathsf{b}\mathit{p} \imp  (\mathsf{person\_name}(\mathit{p},\mathit{n}) \equi  \mathsf{person\_name\_bf}(\mathit{p},\mathit{n}))) &&&&\; \land \\
\forall \mathit{p}\mathit{n} \, (\mathsf{b}\mathit{n} \imp  (\mathsf{person\_name}(\mathit{p},\mathit{n}) \equi  \mathsf{person\_name\_fb}(\mathit{p},\mathit{n}))) &&&&\; \land \\
\forall \mathit{p}\mathit{n} \, (\mathsf{person\_name}(\mathit{p},\mathit{n}) \imp  \mathsf{b}\mathit{p} \land  \mathsf{b}\mathit{n}).
\end{array}
$\pplkbAfter
%
% Doc at position 1665
%
\section{Some Example Queries, Processed by Interpolation}

$b(n)$ in the background formula effects that $\texttt{person\_name}$ is
rewritten \texttt{to person\_name\_fb}:

%
% Statement at position 1846
%
\pplkbBefore
\index{rewrite_1@$\ppldefmacro{rewrite_{1}}$}$\begin{array}{lllll}
\ppldefmacro{rewrite_{1}}
\end{array}
$\pplkbBetween
$\begin{array}{lllll}
\pplmacro{definiens}(\mathsf{person\_name}(\mathsf{p},\mathsf{n}),\\
\hphantom{\pplmacro{definiens}(} \pplmacro{accessor\_spec_{1}} \land  \mathsf{b}\mathsf{n},\\
\hphantom{\pplmacro{definiens}(} {[}\mathsf{person\_name\_bf},\mathsf{person\_name\_fb}{]}).
\end{array}
$\pplkbAfter

\noindent Input: $\pplmacro{rewrite_{1}}.$\\
\noindent Result of interpolation:
\[\begin{array}{lllll}
\mathsf{person\_name\_fb}(\mathsf{p},\mathsf{n}).
\end{array}
\]
%
% Doc at position 2222
%
\noindent
This is the macro expansion of \texttt{rewrite\_1}:
\[\begin{array}{lllll}
\exists \mathit{q}\mathit{r} \, (\forall \mathit{x}\mathit{y} \, (\mathit{q}\mathit{x} \imp  (\mathit{r}\mathit{x}\mathit{y} \equi  \mathsf{person\_name\_bf}(\mathit{x},\mathit{y}))) &&&\; \land \\
\hphantom{\exists \mathit{q}\mathit{r} \, (} \forall \mathit{x}\mathit{y} \, (\mathit{q}\mathit{y} \imp  (\mathit{r}\mathit{x}\mathit{y} \equi  \mathsf{person\_name\_fb}(\mathit{x},\mathit{y}))) &&&\; \land \\
\hphantom{\exists \mathit{q}\mathit{r} \, (} \forall \mathit{x}\mathit{y} \, (\mathit{r}\mathit{x}\mathit{y} \imp  \mathit{q}\mathit{x} \land  \mathit{q}\mathit{y}) &&&\; \land \\
\hphantom{\exists \mathit{q}\mathit{r} \, (} \mathit{q}\mathsf{n} &&&\; \land \\
\hphantom{\exists \mathit{q}\mathit{r} \, (} \mathit{r}\mathsf{p}\mathsf{n}) &&&&\; \imp \\
\lnot  \exists \mathit{q}\mathit{r} \, (\forall \mathit{x}\mathit{y} \, (\mathit{q}\mathit{x} \imp  (\mathit{r}\mathit{x}\mathit{y} \equi  \mathsf{person\_name\_bf}(\mathit{x},\mathit{y}))) &&&\; \land \\
\hphantom{\lnot  \exists \mathit{q}\mathit{r} \, (} \forall \mathit{x}\mathit{y} \, (\mathit{q}\mathit{y} \imp  (\mathit{r}\mathit{x}\mathit{y} \equi  \mathsf{person\_name\_fb}(\mathit{x},\mathit{y}))) &&&\; \land \\
\hphantom{\lnot  \exists \mathit{q}\mathit{r} \, (} \forall \mathit{x}\mathit{y} \, (\mathit{r}\mathit{x}\mathit{y} \imp  \mathit{q}\mathit{x} \land  \mathit{q}\mathit{y}) &&&\; \land \\
\hphantom{\lnot  \exists \mathit{q}\mathit{r} \, (} \mathit{q}\mathsf{n} &&&\; \land \\
\hphantom{\lnot  \exists \mathit{q}\mathit{r} \, (} \lnot  \mathit{r}\mathsf{p}\mathsf{n}).
\end{array}
\]
%
% Doc at position 2347
%
\noindent
For given name compute years in which a person with that name has been born:
%
% Statement at position 2441
%
\pplkbBefore
\index{rewrite_2@$\ppldefmacro{rewrite_{2}}$}$\begin{array}{lllll}
\ppldefmacro{rewrite_{2}}
\end{array}
$\pplkbBetween
$\begin{array}{lllll}
\pplmacro{definiens}(\exists \mathit{p} \, (\mathsf{person\_name}(\mathit{p},\mathsf{n}) \land  \mathsf{person\_yob}(\mathit{p},\mathsf{y})),\\
\hphantom{\pplmacro{definiens}(} \pplmacro{accessor\_spec} \land  \mathsf{b}\mathsf{n},\\
\hphantom{\pplmacro{definiens}(} {[}\mathsf{person\_name\_bf},\mathsf{person\_name\_fb},\mathsf{person\_yob\_bf},\mathsf{person\_yob\_fb}{]}).
\end{array}
$\pplkbAfter

\noindent Input: $\pplmacro{rewrite_{2}}.$\\
\noindent Result of interpolation:
\[\begin{array}{lllll}
\exists \mathit{x} \, (\mathsf{person\_name\_fb}(\mathit{x},\mathsf{n}) \land  \mathsf{person\_yob\_bf}(\mathit{x},\mathsf{y})).
\end{array}
\]
%
% Doc at position 2656
%
\section{Referential Constraints}

The predicate \texttt{person} can be accessed to ``enumerate''
all persons. There is a referential constraint from
\texttt{person\_name} to \texttt{person}.
%
% Statement at position 2856
%
\pplkbBefore
\index{person_spec@$\ppldefmacro{person\_spec}$}$\begin{array}{lllll}
\ppldefmacro{person\_spec}
\end{array}
$\pplkbBetween
$\begin{array}{lllll}
\forall \mathit{p} \, (\mathsf{person}(\mathit{p}) \imp  \mathsf{b}\mathit{p}) &&&&\; \land \\
\forall \mathit{p}\mathit{n} \, (\mathsf{person\_name}(\mathit{p},\mathit{n}) \imp  \mathsf{person}(\mathit{p})).
\end{array}
$\pplkbAfter
%
% Statement at position 2953
%
\pplkbBefore
\index{rewrite_3@$\ppldefmacro{rewrite_{3}}$}$\begin{array}{lllll}
\ppldefmacro{rewrite_{3}}
\end{array}
$\pplkbBetween
$\begin{array}{lllll}
\pplmacro{definiens}(\exists \mathit{p} \, \mathsf{person\_name}(\mathit{p},\mathsf{n}),\\
\hphantom{\pplmacro{definiens}(} \pplmacro{person\_spec} \land  \pplmacro{accessor\_spec},\\
\hphantom{\pplmacro{definiens}(} {[}\mathsf{person},\mathsf{person\_name\_bf},\mathsf{person\_name\_fb}{]}).
\end{array}
$\pplkbAfter

\noindent Input: $\pplmacro{rewrite_{3}}.$\\
\noindent Result of interpolation:
\[\begin{array}{lllll}
\exists \mathit{x} \, (\mathsf{person}(\mathit{x}) \land  \mathsf{person\_name\_bf}(\mathit{x},\mathsf{n})).
\end{array}
\]
%
% Doc at position 3135
%

\section{Different Proofs -- Different Interpolants}

Here only the second returned interpolant is without the redundant occurrence
of \texttt{person}.
%
% Statement at position 3296
%
\pplkbBefore
\index{rewrite_4@$\ppldefmacro{rewrite_{4}}$}$\begin{array}{lllll}
\ppldefmacro{rewrite_{4}}
\end{array}
$\pplkbBetween
$\begin{array}{lllll}
\pplmacro{definiens}(\exists \mathit{p} \, (\mathsf{person\_name}(\mathit{p},\mathsf{n}) \land  \mathsf{person\_yob}(\mathit{p},\mathsf{y})),\\
\hphantom{\pplmacro{definiens}(} \pplmacro{person\_spec} \land  \pplmacro{accessor\_spec} \land  \mathsf{b}\mathsf{n},\\
\hphantom{\pplmacro{definiens}(} {[}\mathsf{person},\mathsf{person\_name\_bf},\mathsf{person\_name\_fb},\mathsf{person\_yob\_bf},\mathsf{person\_yob\_fb}{]}).
\end{array}
$\pplkbAfter

\noindent Input: $\pplmacro{rewrite_{4}}.$\\
\noindent Result of interpolation:
\[\begin{array}{lllll}
\exists \mathit{x} \, (\mathsf{person}(\mathit{x}) \land  \mathsf{person\_name\_fb}(\mathit{x},\mathsf{n}) \land  \mathsf{person\_yob\_bf}(\mathit{x},\mathsf{y})).
\end{array}
\]

\noindent Input: $\pplmacro{rewrite_{4}}.$\\
\noindent Result of interpolation:
\[\begin{array}{lllll}
\exists \mathit{x} \, (\mathsf{person\_name\_fb}(\mathit{x},\mathsf{n}) \land  \mathsf{person\_yob\_bf}(\mathit{x},\mathsf{y})).
\end{array}
\]
\printindex
\end{document}
