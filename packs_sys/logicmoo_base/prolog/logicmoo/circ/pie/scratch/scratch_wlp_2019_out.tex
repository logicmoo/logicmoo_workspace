\documentclass[a4paper]{article}
\usepackage[utf8]{inputenc}
\usepackage[T1]{fontenc}
\usepackage{imakeidx}
\usepackage[hidelinks]{hyperref}
%% A screen friendly geometry:	    
\usepackage[paper=a5paper,scale=0.9]{geometry}
%% PPL Setup
	   
\newcommand{\assign}{\mathrel{\mathop:}=}
\newcommand{\concat}{\mathrel{+\!+}}
	    
\newcommand{\f}[1]{\mathsf{#1}}
\newcommand{\true}{\top}
\newcommand{\false}{\bot}
\newcommand{\imp}{\rightarrow}
\newcommand{\revimp}{\leftarrow}
\newcommand{\equi}{\leftrightarrow}
\newcommand{\entails}{\models}	    
\newcommand{\eqdef}{\; 
\raisebox{-0.1ex}[0mm]{$ \stackrel{\raisebox{-0.2ex}{\tiny 
\textnormal{def}}}{=} $}\; }
\newcommand{\iffdef}{\n{iff}_{\mbox{\scriptsize \textnormal{def}}}}

\newcommand{\pplmacro}[1]{\mathit{#1}}
\newcommand{\ppldefmacro}[1]{\mathit{#1}}
\newcommand{\pplparam}[1]{\mathit{#1}}
\newcommand{\pplparamidx}[2]{\mathit{#1}_{#2}}
\newcommand{\pplparamplain}[1]{#1}
\newcommand{\pplparamplainidx}[2]{#1_{#2}}
\newcommand{\pplparamsup}[2]{\mathit{#1}^{#2}}
\newcommand{\pplparamsupidx}[3]{\mathit{#1}^{#2}_{#3}}
\newcommand{\pplparamplainsup}[2]{#1^{#2}}
\newcommand{\pplparamplainsupidx}[3]{#1^{#2}_{#3}}
\newcommand{\pplparamnum}[1]{\mathit{X}_{#1}}

%%	    
%% We use @startsection just to obtain reduced vertical spacing above
%% macro headers which are immediately after other headers, e.g. of sections
%%	    
\makeatletter%
\newcounter{entry}%
\newcommand{\entrymark}[1]{}%
\newcommand\entryhead{%
\@startsection{entry}{10}{\z@}{12pt plus 2pt minus 2pt}{0pt}{}}%
\makeatother
	    
\newcommand{\pplkbBefore}
{\entryhead*{}%
\setlength{\arraycolsep}{0pt}%
\pagebreak[0]%
\begin{samepage}%
\noindent%
\rule[0.5pt]{\textwidth}{2pt}\\%
\noindent}

% \newcommand{\pplkbDefType}[1]{\hspace{\fill}{{[}#1{]}\\}}

\newcommand{\pplkbBetween}
{\setlength{\arraycolsep}{3pt}%
\\\rule[3pt]{\textwidth}{1pt}%
\par\nopagebreak\noindent Defined as\begin{center}}

\newcommand{\pplkbAfter}{\end{center}\end{samepage}\noindent}

\newcommand{\pplkbBodyBefore}{\par\noindent where\begin{center}}
\newcommand{\pplkbBodyAfter}{\end{center}}

\newcommand{\pplkbFreePredicates}[1]{\f{free\_predicates}(#1)}
% \newcommand{\pplkbRenameFreeOccurrences}[3]{\f{rename\_free\_occurrences}(#1,#2,#3)}

\newcommand{\pplIsValid}[1]{\noindent This formula is valid: $#1$\par}
\newcommand{\pplIsNotValid}[1]{\noindent This formula is not valid: $#1$\par}	    
\newcommand{\pplFailedToValidate}[1]{\noindent Failed to validate this formula: $#1$\par}

\newcounter{def}
	    
\makeindex
\usepackage{graphicx}
\usepackage{amssymb}

\begin{document}
%
% Doc at position 0
%
 \title{\textit{PIE} Document: WLP 2019 Examples}
  \date{Revision: August 29, 2019; Rendered: \today}
  \maketitle

  \noindent Examples from the DECLARE/WLP 2019 presentation \textit{PIE --
  Proving, Interpolating and Eliminating on the Basis of First-Order Logic}.
  
%
% Doc at position 965
%
 \section{Abduction with the Weakest Sufficient Condition}
%
% Statement at position 1031
%
\pplkbBefore
\index{kb1@$\ppldefmacro{kb_{1}}$}$\begin{array}{lllll}
\ppldefmacro{kb_{1}}
\end{array}
$\pplkbBetween
$\begin{array}{lllll}
(\mathsf{sprinkler\_was\_on} \imp  \mathsf{wet}(\mathsf{grass})) &&&&\; \land \\
(\mathsf{rained\_last\_night} \imp  \mathsf{wet}(\mathsf{grass})) &&&&\; \land \\
(\mathsf{wet}(\mathsf{grass}) \imp  \mathsf{wet}(\mathsf{shoes})).
\end{array}
$\pplkbAfter
%
% Doc at position 1141
%
 A knowledge base.
%
% Statement at position 1167
%
\pplkbBefore
\index{explanation(Kb,Na,Obs)@$\ppldefmacro{explanation}(\pplparam{Kb},\pplparam{Na},\pplparam{Obs})$}$\begin{array}{lllll}
\ppldefmacro{explanation}(\pplparam{Kb},\pplparam{Na},\pplparam{Obs})
\end{array}
$\pplkbBetween
$\begin{array}{lllll}
\forall \pplparam{Na} \, (\pplparam{Kb} \imp  \pplparam{Obs}).
\end{array}
$\pplkbAfter
%
% Doc at position 1224
%
 The weakest sufficient condition of observation $\begin{array}{l}
\pplparam{Obs}
\end{array}
$ on
  the complement of $\begin{array}{l}
\pplparam{Na}
\end{array}
$ as
  assumables within knowledge base $\begin{array}{l}
\pplparam{Kb}
\end{array}
$.
  \par\medskip\noindent
  The expression $\begin{array}{l}
\pplmacro{explanation}(\pplmacro{kb_{1}},{[}\mathsf{wet}{]},\mathsf{wet}(\mathsf{shoes}))
\end{array}
$ expands into:

\[\begin{array}{lllll}
\forall \mathit{p} \, ((\mathsf{sprinkler\_was\_on} \imp  \mathit{p}(\mathsf{grass})) &&&\; \land \\
\hphantom{\forall \mathit{p} \, (} (\mathsf{rained\_last\_night} \imp  \mathit{p}(\mathsf{grass})) &&&\; \land \\
\hphantom{\forall \mathit{p} \, (} (\mathit{p}(\mathsf{grass}) \imp  \mathit{p}(\mathsf{shoes})) &&&&\; \imp \\
\hphantom{\forall \mathit{p} \, (} \mathit{p}(\mathsf{shoes})).
\end{array}
\]
%
% Doc at position 1591
%
 \par\medskip\noindent Second-order quantifier elimination
    computes the abductive explanation: \par\medskip

\noindent Input: $\pplmacro{explanation}(\pplmacro{kb_{1}},{[}\mathsf{wet}{]},\mathsf{wet}(\mathsf{shoes})).$\\
\noindent Result of elimination:
\[\begin{array}{lllll}
\mathsf{rained\_last\_night} \lor  \mathsf{sprinkler\_was\_on}.
\end{array}
\]
%
% Doc at position 1775
%
 \noindent The following is shown by invoking a first-order prover, Prover9
    by default: \par\medskip
\pplIsValid{\pplmacro{kb_{1}} \land  \mathsf{rained\_last\_night} \imp  \mathsf{wet}(\mathsf{shoes}).}
%
% Doc at position 1963
%
 \section{A Simple Example of Second-Order Quantifier Elimination}

\noindent Input: $\exists \mathit{p} \, (\forall \mathit{x} \, (\mathsf{q}(\mathit{x}) \imp  \mathit{p}(\mathit{x})) \land  \forall \mathit{x} \, (\mathit{p}(\mathit{x}) \imp  \mathsf{r}(\mathit{x}))).$\\
\noindent Result of elimination:
\[\begin{array}{lllll}
\forall \mathit{x} \, (\mathsf{q}(\mathit{x}) \imp  \mathsf{r}(\mathit{x})).
\end{array}
\]
%
% Doc at position 2134
%
 \section{Predicate Circumscription}
%
% Statement at position 2180
%
\pplkbBefore
\index{circ(P,F)@$\ppldefmacro{circ}(\pplparamplain{P},\pplparamplain{F})$}$\begin{array}{lllll}
\ppldefmacro{circ}(\pplparamplain{P},\pplparamplain{F})
\end{array}
$\pplkbBetween
$\begin{array}{lllll}
\pplparamplain{F} \land  \lnot  \exists \pplparamplainsup{P}{\prime} \, (\pplparamplainsup{F}{\prime} \land  \pplparamplainidx{T}{1} \land  \lnot  \pplparamplainidx{T}{2}),
\end{array}
$\pplkbAfter
\pplkbBodyBefore
$
\begin{array}{l}\pplparamplainsup{F}{\prime} \assign \pplparamplain{F}[\pplparamplain{P} \mapsto \pplparamplainsup{P}{\prime}],\\
\pplparamplain{A} \assign \mathrm{arity\ of }\; \pplparamplain{P}\; \mathrm{ in }\; \pplparamplain{F},\\
\pplparamplainidx{T}{1} \assign \mathrm{transfer\ clauses}\; {[}\pplparamplain{P}/\pplparamplain{A}\textrm{-}\mathsf{n}{]} \rightarrow {[}\pplparamplainsup{P}{\prime}{]},\\
\pplparamplainidx{T}{2} \assign \mathrm{transfer\ clauses}\; {[}\pplparamplainsup{P}{\prime}{]} \rightarrow {[}\pplparamplain{P}/\pplparamplain{A}\textrm{-}\mathsf{n}{]}.

\end{array}$\pplkbBodyAfter
%
% Doc at position 2398
%
   Predicate circumscription of a single predicate. The formula
   $\begin{array}{l}
\pplmacro{circ}(\mathsf{p},\mathsf{p}(\mathsf{a}))
\end{array}
$, for example, expands into:
\[\begin{array}{lllll}
\mathsf{p}(\mathsf{a}) &&&&\; \land \\
\lnot  \exists \mathit{q} \, (\mathit{q}(\mathsf{a}) \land  \forall \mathit{x} \, (\mathit{q}(\mathit{x}) \imp  \mathsf{p}(\mathit{x})) \land  \lnot  \forall \mathit{x} \, (\mathsf{p}(\mathit{x}) \imp  \mathit{q}(\mathit{x}))).
\end{array}
\]
%
% Doc at position 2590
%
 \par\medskip\noindent Second-order quantifier elimination can be applied
    to to compute that circumscription:\par\medskip

\noindent Input: $\pplmacro{circ}(\mathsf{p},\mathsf{p}(\mathsf{a})).$\\
\noindent Result of elimination:
\[\begin{array}{lllll}
\mathsf{p}(\mathsf{a}) \land  \forall \mathit{x} \, (\mathsf{p}(\mathit{x}) \imp  \mathit{x}=\mathsf{a}).
\end{array}
\]
%
% Doc at position 2766
%
 \par\medskip\noindent Similarly we can compute the circumscription of
    $\begin{array}{l}
\mathsf{wet}
\end{array}
$ in $\begin{array}{l}
\pplmacro{kb_{1}}
\end{array}
$:\par\medskip

\noindent Input: $\pplmacro{circ}(\mathsf{wet},\pplmacro{kb_{1}}).$\\
\noindent Result of elimination:
\[\begin{array}{lllll}
(\mathsf{rained\_last\_night} \imp  \mathsf{wet}(\mathsf{grass})) &&&&\; \land \\
(\mathsf{sprinkler\_was\_on} \imp  \mathsf{wet}(\mathsf{grass})) &&&&\; \land \\
(\mathsf{wet}(\mathsf{grass}) \imp  \mathsf{wet}(\mathsf{shoes})) &&&&\; \land \\
\forall \mathit{x} \, (\mathsf{wet}(\mathit{x}) \imp  \mathsf{rained\_last\_night} \lor  \mathsf{sprinkler\_was\_on}) &&&&\; \land \\
\forall \mathit{x} \, (\mathsf{wet}(\mathit{x}) \land  \mathsf{wet}(\mathsf{grass}) \imp  \mathit{x}=\mathsf{grass} \lor  \mathit{x}=\mathsf{shoes}).
\end{array}
\]
%
% Doc at position 2970
%
  \section{Computing Modal Correspondences}  
%
% Doc at position 3024
%
 $\Box p \imp p$, known as axiom $M$ or $T$, corresponds to reflexivity of
   the accessibility relation:

\noindent Input: $\forall \mathit{p} \, \forall \mathit{v} \, (\forall \mathit{w} \, (\mathsf{r}(\mathit{v},\mathit{w}) \imp  \mathit{p}(\mathit{w})) \imp  \mathit{p}(\mathit{v})).$\\
\noindent Result of elimination:
\[\begin{array}{lllll}
\forall \mathit{x} \, \mathsf{r}(\mathit{x},\mathit{x}).
\end{array}
\]
%
% Doc at position 3220
%
\par\medskip\noindent
  $\Box p \imp \Box \Box p$, known as axiom $4$, corresponds to transitivity:


\noindent Input: $\forall \mathit{p} \, \forall \mathit{v} \, (\forall \mathit{w} \, (\mathsf{r}(\mathit{v},\mathit{w}) \imp  \mathit{p}(\mathit{w})) \imp  \forall \mathit{w} \, (\mathsf{r}(\mathit{v},\mathit{w}) \imp  \forall \mathit{w_{1}} \, (\mathsf{r}(\mathit{w},\mathit{w_{1}}) \imp  \mathit{p}(\mathit{w_{1}})))).$\\
\noindent Result of elimination:
\[\begin{array}{lllll}
\forall \mathit{x} \forall \mathit{y} \, (\mathsf{r}(\mathit{x},\mathit{y}) \imp  \forall \mathit{z} \, (\mathsf{r}(\mathit{y},\mathit{z}) \imp  \mathsf{r}(\mathit{x},\mathit{z}))).
\end{array}
\]
%
% Doc at position 3481
%
  \section{Craig Interpolation}
%
% Doc at position 3521
%
 A simple propositional example for Craig interpolation:

%
% Statement at position 3586
%
\pplkbBefore
\index{ip1@$\ppldefmacro{ip_{1}}$}$\begin{array}{lllll}
\ppldefmacro{ip_{1}}
\end{array}
$\pplkbBetween
$\begin{array}{lllll}
\mathsf{p} \land  \mathsf{q} \imp  \mathsf{p} \lor  \mathsf{r}.
\end{array}
$\pplkbAfter

\noindent Input: $\pplmacro{ip_{1}}.$\\
\noindent Result of interpolation:
\[\begin{array}{lllll}
\mathsf{p}.
\end{array}
\]
%
% Doc at position 3651
%
 \par\bigskip\noindent A first-order example for Craig interpolation with
    combined universal and existential quantification:

%
% Statement at position 3788
%
\pplkbBefore
\index{ip2@$\ppldefmacro{ip_{2}}$}$\begin{array}{lllll}
\ppldefmacro{ip_{2}}
\end{array}
$\pplkbBetween
$\begin{array}{lllll}
\forall \mathit{x} \, \mathsf{p}(\mathsf{a},\mathit{x}) \land  \mathsf{q} \imp  \exists \mathit{x} \, \mathsf{p}(\mathit{x},\mathsf{b}) \lor  \mathsf{r}.
\end{array}
$\pplkbAfter

\noindent Input: $\pplmacro{ip_{2}}.$\\
\noindent Result of interpolation:
\[\begin{array}{lllll}
\exists \mathit{x} \, \forall \mathit{y} \, \mathsf{p}(\mathit{x},\mathit{y}).
\end{array}
\]
%
% Doc at position 3878
%
 \par\bigskip\noindent
    A simple first-order example for Craig interpolation, with
    displayed underlying tableau:

%
% Statement at position 4006
%
\pplkbBefore
\index{ip3@$\ppldefmacro{ip_{3}}$}$\begin{array}{lllll}
\ppldefmacro{ip_{3}}
\end{array}
$\pplkbBetween
$\begin{array}{lllll}
\forall \mathit{x} \, \mathsf{p}(\mathit{x}) \land  \forall \mathit{x} \, (\mathsf{p}(\mathit{x}) \imp  \mathsf{q}(\mathit{x})) \imp  \mathsf{q}(\mathsf{c}).
\end{array}
$\pplkbAfter

\noindent Input: $\pplmacro{ip_{3}}.$\\
\noindent Result of interpolation:
\[\begin{array}{lllll}
\forall \mathit{x} \, \mathsf{q}(\mathit{x}).
\end{array}
\]
%
% Doc at position 4169
%
The clausal tableau proof underlying interpolant extraction can be visualized,
colors representing the sides with respect to interpolation. The color of the
closing marks indicate the side of the connection partner:

\begin{center}
\includegraphics[width=10em]{/tmp/tmp_pie_dotgraph01}
\end{center}
%
% Doc at position 4478
%
 \section{Definientia, Definability}
%
% Statement at position 4522
%
\pplkbBefore
\index{definiens(G,F,P)@$\ppldefmacro{definiens}(\pplparamplain{G},\pplparamplain{F},\pplparamplain{P})$}$\begin{array}{lllll}
\ppldefmacro{definiens}(\pplparamplain{G},\pplparamplain{F},\pplparamplain{P})
\end{array}
$\pplkbBetween
$\begin{array}{lllll}
\exists \pplparamplain{P} \, (\pplparamplain{F} \land  \pplparamplain{G}) \imp  \forall \pplparamplain{P} \, (\pplparamplain{F} \imp  \pplparamplain{G}).
\end{array}
$\pplkbAfter
%
% Doc at position 4584
%
  The interpolants of the left and right side of this implication are exactly
  the definientia of $G$ in $F$ in terms of all predicates not in $P$.  The
  implication is valid if and only if definability holds.
  \par\bigskip\noindent Her is an example:
%
% Statement at position 4847
%
\pplkbBefore
\index{kb2@$\ppldefmacro{kb_{2}}$}$\begin{array}{lllll}
\ppldefmacro{kb_{2}}
\end{array}
$\pplkbBetween
$\begin{array}{lllll}
\forall \mathit{x} \, (\mathsf{p}(\mathit{x}) \imp  \mathsf{q}(\mathit{x}) \land  \mathsf{s}(\mathit{x})) &&&&\; \land \\
\forall \mathit{x} \, (\mathsf{s}(\mathit{x}) \imp  \mathsf{r}(\mathit{x})) &&&&\; \land \\
\forall \mathit{x} \, (\mathsf{q}(\mathit{x}) \land  \mathsf{r}(\mathit{x}) \imp  \mathsf{p}(\mathit{x})).
\end{array}
$\pplkbAfter
%
% Doc at position 4944
%
 We verify definability of $p$:\smallskip\par
\pplIsValid{\pplmacro{definiens}(\mathsf{p}(\mathsf{a}),\pplmacro{kb_{2}},{[}\mathsf{p},\mathsf{s}{]}).}
%
% Doc at position 5058
%
 \par\medskip\noindent Craig interpolation can now be applied to compute a definiens:\smallskip\par

\noindent Input: $\pplmacro{definiens}(\mathsf{p}(\mathsf{a}),\pplmacro{kb_{2}},{[}\mathsf{p},\mathsf{s}{]}).$\\
\noindent Result of interpolation:
\[\begin{array}{lllll}
\mathsf{q}(\mathsf{a}) \land  \mathsf{r}(\mathsf{a}).
\end{array}
\]
%
% Doc at position 5225
%
 Since $\f{a}$ does not occur free in $\begin{array}{l}
\pplmacro{kb_{2}}
\end{array}
$, it may be
    considered as representing a variable in a definition of $\f{p}$.
    That is, we can verify:\smallskip
  
\pplIsValid{\pplmacro{kb_{2}} \imp  \forall \mathit{a} \, (\mathsf{p}(\mathit{a}) \equi  \mathsf{q}(\mathit{a}) \land  \mathsf{r}(\mathit{a})).}
%
% Doc at position 5484
%
 \section{Graph Colorability}
%
% Statement at position 5523
%
\pplkbBefore
\index{col2(E)@$\ppldefmacro{col_{2}}(\pplparamplain{E})$}$\begin{array}{lllll}
\ppldefmacro{col_{2}}(\pplparamplain{E})
\end{array}
$\pplkbBetween
$\begin{array}{lllll}
\exists \mathit{r} \exists \mathit{g} \, (\forall \mathit{x} \, (\mathit{r}(\mathit{x}) \lor  \mathit{g}(\mathit{x})) &&&&\; \land \\
\hphantom{\exists \mathit{r} \exists \mathit{g} \, (} \forall \mathit{x} \forall \mathit{y} \, (\pplparamplain{E}(\mathit{x},\mathit{y}) \imp  \lnot  (\mathit{r}(\mathit{x}) \land  \mathit{r}(\mathit{y})) \land  \lnot  (\mathit{g}(\mathit{x}) \land  \mathit{g}(\mathit{y})))).
\end{array}
$\pplkbAfter
%
% Doc at position 5657
%
 2-colorability as an existential second-order formula.  The predicate
  describing the graph is exported as a parameter. Predicate parameters may be
  instantiated with a constant or a $\lambda$-expression. For example,
  $\begin{array}{l}
\pplmacro{col_{2}}(\lambda (\mathit{u},\mathit{v}).(\mathit{u}=\mathsf{1} \land  \mathit{v}=\mathsf{2}) \lor  (\mathit{u}=\mathsf{2} \land  \mathit{v}=\mathsf{3}))
\end{array}
$ expands into:
  
\[\begin{array}{lllll}
\exists \mathit{p} \exists \mathit{q} \, (\forall \mathit{x} \, (\mathit{p}(\mathit{x}) \lor  \mathit{q}(\mathit{x})) &&&&\; \land \\
\hphantom{\exists \mathit{p} \exists \mathit{q} \, (} \forall \mathit{x} \forall \mathit{y} \, ((\mathit{x}=\mathsf{1} \land  \mathit{y}=\mathsf{2}) \lor  (\mathit{x}=\mathsf{2} \land  \mathit{y}=\mathsf{3}) \imp  \lnot  (\mathit{p}(\mathit{x}) \land  \mathit{p}(\mathit{y})) \land  \lnot  (\mathit{q}(\mathit{x}) \land  \mathit{q}(\mathit{y})))).
\end{array}
\]
%
% Doc at position 6056
%
  
    \par\bigskip\noindent We now perform some computations by elimination
    based on the inner first-order component of the above specification of
    2-colorability:
%
% Statement at position 6235
%
\pplkbBefore
\index{fo_col2(E)@$\ppldefmacro{fo\_col_{2}}(\pplparamplain{E})$}$\begin{array}{lllll}
\ppldefmacro{fo\_col_{2}}(\pplparamplain{E})
\end{array}
$\pplkbBetween
$\begin{array}{lllll}
\forall \mathit{x} \, (\mathsf{r}(\mathit{x}) \lor  \mathsf{g}(\mathit{x})) &&&&\; \land \\
\forall \mathit{x} \forall \mathit{y} \, (\pplparamplain{E}(\mathit{x},\mathit{y}) \imp  \lnot  (\mathsf{r}(\mathit{x}) \land  \mathsf{r}(\mathit{y})) \land  \lnot  (\mathsf{g}(\mathit{x}) \land  \mathsf{g}(\mathit{y}))).
\end{array}
$\pplkbAfter
%
% Doc at position 6347
%
 We can instantiate the graph with a predicate constant and
    eliminate one of the color predicates:\par\medskip

\noindent Input: $\exists \mathit{g} \, \pplmacro{fo\_col_{2}}(\mathsf{e}).$\\
\noindent Result of elimination:
\[\begin{array}{lllll}
\forall \mathit{x} \forall \mathit{y} \, (\mathsf{e}(\mathit{x},\mathit{y}) \imp  \lnot  (\mathsf{r}(\mathit{y}) \land  \mathsf{r}(\mathit{x})) \land  (\mathsf{r}(\mathit{y}) \lor  \mathsf{r}(\mathit{x}))).
\end{array}
\]
%
% Doc at position 6519
%
 \par\medskip\noindent We can evaluate 2-colorability for a given graph by
    second-order quantifier elimination. In the current version of
    \textit{PIE} this does not suceed in a single call to elimination, but
    works in two steps with different elimination configurations:
%
% Statement at position 6809
%
\pplkbBefore
\index{ex_rg(F)@$\ppldefmacro{ex\_rg}(\pplparamplain{F})$}$\begin{array}{lllll}
\ppldefmacro{ex\_rg}(\pplparamplain{F})
\end{array}
$\pplkbBetween
$\begin{array}{lllll}
\exists \mathit{r} \, \pplparamplainidx{F}{1},
\end{array}
$\pplkbAfter
\pplkbBodyBefore
$
\begin{array}{l}\mathrm{ppl\_elim(ex2([g],F),[elim\_options=[pre=[c6]],printing=false,r=F1])}.

\end{array}$\pplkbBodyAfter

\noindent Input: $\pplmacro{ex\_rg}(\pplmacro{fo\_col_{2}}(\lambda (\mathit{u},\mathit{v}).(\mathit{u}=\mathsf{1} \land  \mathit{v}=\mathsf{2}) \lor  (\mathit{u}=\mathsf{2} \land  \mathit{v}=\mathsf{3}))).$\\
\noindent Result of elimination:
\[\begin{array}{lllll}
\mathsf{1}\neq \mathsf{2} \land  \mathsf{2}\neq \mathsf{3}.
\end{array}
\]
\printindex
\end{document}
